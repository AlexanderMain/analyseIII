\section{Opérateurs adjoints}
\subsection{Adjoints dans les espaces de Hilbert }
\begin{prop}\label{adj:hilb}
  Soient $H$ un espace de Hilbert et $T\in\mathcal L(H)$. Il existe un opérateur
  linéaire $T^*$ tel que pour tous $x$, $y\in H$, on a
  $$\langle Tx, y\rangle = \langle x, T^*y\rangle$$
  et on a $\|T\| = \|T^*\|$.
\end{prop}

\begin{proof}
  Pour $y\in H$, on définit $y^*:H\to\mathbb K: x\mapsto \langle Tx, y\rangle$.
  Le théorème de Riesz-Fréchet assure dès lors qu'il existe un unique
  $z_y\in H$, tel que pour tout $x$ dans $H$, $\langle x, z_y\rangle =
  \langle Tx, y\rangle$. On pose $T^*y = z_y$.

  Montrons que $T^*$ est linéaire. Soient $y$, $z\in H$ et $a$,
  $b\in\mathbb K$. Pour tout $x\in H$,
  \begin{IEEEeqnarray*}{rCl}
    \langle x, T^*(ay + bz)\rangle & = & \langle Tx, ay + bz\rangle \\
    & = & \bar a \langle Tx, y\rangle + \bar b \langle Tx, z\rangle \\
    & = & \bar a \langle x, T^*y\rangle + \bar b \langle x, T^*z\rangle \\
    & = & \langle x, a T^*(y) + bT^*(z)\rangle.
  \end{IEEEeqnarray*}
  Par non-dégénérescence du produit scalaire, on a
  $T^*(ay + bz) = a T^*(y) + bT^*(z)$, ce qui montre la linéarité.

  Montrons maintenant la continuité de $T^*$. Pour tous $x$, $y\in H$,
  l'inégalité de Cauchy-Schwarz assure que
  $|\langle x, T^*y\rangle| = |\langle Tx, y\rangle| \leq
  \|Tx\|\cdot \|y\| \leq \|T\|\cdot \|x\|\cdot\|y\|$. Rappelons que
  $\|T^*y\| = \sup_{\|x\|\leq 1}|\langle x, T^*y\rangle$, et on déduit de
  ces inégalités que $\|T^*y\| \leq \|T\| \cdot\|y\|$. Ceci montre la continuité
  de $T^*$.

  Il reste à démontrer la deuxième inégalité pour clôturer la preuve de la
  proposition. Or, pour tous $x\in H$, on a
  $$\|Tx\| = \sup_{\|y\|\leq 1}\langle Tx, y\rangle =
  \sup_{\|y\|\leq 1}\langle x, T^*y\rangle \leq \|x\|\cdot \|T^*\|$$
  ce qui conclut la preuve.
\end{proof}

\begin{df}
  Soient $H$ un espace de Hilbert et $T\in\mathcal L(H)$. On appelle l'opérateur
  $T^*$ donné par la proposition \ref{adj:hilb} est appelé l'adjoint de $T$.
\end{df}

\begin{rem}
  L'adjoint d'un opérateur linéaire $T$ est unique. L'application
  $\mathcal L(H)\to\mathcal L(H): T\mapsto T^*$ est bijective, car elle
  est son propre inverse, c'est-à-dire $T^{**} = T$.
\end{rem}

\begin{exo}
  Soit $T: \ell^2\to \ell^2: (x_1, x_2, \ldots)\mapsto (0, x_1, x_2, \ldots)$.
  Déterminez $T^*$.
\end{exo}
\subsection{Adjoints dans les espaces de Banach}
\begin{df}
  Soient $X$, $Y$ des espaces de Banach et $T\in \mathcal L(X, Y)$.
  On définit l'adjointe de $T$ (également appelée transposée ou application
  duale) par $$T^*: Y^*\to X^*: y^*\mapsto y^*\circ T.$$
\end{df}

\begin{prop}
  Soient $X$, $Y$ des espaces de Banach et $T\in \mathcal L(X, Y)$.
  L'application $T^*$ est linéaire et continue et on a l'égalité
  $\|T^*\| = \|T\|$.
\end{prop}
\begin{proof}
  La preuve de la linéarité est laissée en exercice.

  Pour tout $y^*\in Y^*$, on a $\|T^*(y^*)\| \leq \|y^*\| \cdot \|T\|$ (par
  sous-multiplicativité de la norme opérateur), ce qui montre que
  $\|T^*\| \leq \|T\|$. En particulier, $T^*$ est continue.

  L'inégalité réciproque se déduit du corollaire \ref{hb:a:norme} du théorème
  de Hahn-Banach; pour tout $x\in X$, on a:
  $$\|Tx\| = \max_{\substack{\|y^*\|\leq 1 \\ y^*\in Y^*}}|y^*(T(x))|
  =\max_{\substack{\|y^*\|\leq 1 \\ y^*\in Y^*}}|T^*(y^*)(x)|
  \leq \|T^*\|\cdot \|x\|.$$
\end{proof}
\section{Théorie ergodique}


\section{Propriété de Blum-Hanson (1960)}
%%% Local Variables:
%%% mode: latex
%%% TeX-master: "../analyse3"
%%% End:
