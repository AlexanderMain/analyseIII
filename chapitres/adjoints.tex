\section{Opérateurs adjoints}
\subsection{Adjoints dans les espaces de Hilbert }
\begin{prop}\label{adj:hilb}
  Soient $H$ un espace de Hilbert et $T\in\mathcal L(H)$. Il existe un opérateur
  linéaire $T^*$ tel que pour tous $x$, $y\in H$, on a
  $$\langle Tx, y\rangle = \langle x, T^*y\rangle$$
  et on a $\|T\| = \|T^*\|$.
\end{prop}

\begin{proof}
  Pour $y\in H$, on définit $y^*:H\to\mathbb K: x\mapsto \langle Tx, y\rangle$.
  Le théorème de Riesz-Fréchet assure dès lors qu'il existe un unique
  $z_y\in H$, tel que pour tout $x$ dans $H$, $\langle x, z_y\rangle =
  \langle Tx, y\rangle$. On pose $T^*y = z_y$.

  Montrons que $T^*$ est linéaire. Soient $y$, $z\in H$ et $a$,
  $b\in\mathbb K$. Pour tout $x\in H$,
  \begin{IEEEeqnarray*}{rCl}
    \langle x, T^*(ay + bz)\rangle & = & \langle Tx, ay + bz\rangle \\
    & = & \bar a \langle Tx, y\rangle + \bar b \langle Tx, z\rangle \\
    & = & \bar a \langle x, T^*y\rangle + \bar b \langle x, T^*z\rangle \\
    & = & \langle x, a T^*(y) + bT^*(z)\rangle.
  \end{IEEEeqnarray*}
  Par non-dégénérescence du produit scalaire, on a
  $T^*(ay + bz) = a T^*(y) + bT^*(z)$, ce qui montre la linéarité.

  Montrons maintenant la continuité de $T^*$. Pour tous $x$, $y\in H$,
  l'inégalité de Cauchy-Schwarz assure que
  $|\langle x, T^*y\rangle| = |\langle Tx, y\rangle| \leq
  \|Tx\|\cdot \|y\| \leq \|T\|\cdot \|x\|\cdot\|y\|$. Rappelons que
  $\|T^*y\| = \sup_{\|x\|\leq 1}|\langle x, T^*y\rangle|$, et on déduit de
  ces inégalités que $\|T^*y\| \leq \|T\| \cdot\|y\|$. Ceci montre la continuité
  de $T^*$.

  Il reste à démontrer la deuxième inégalité pour clôturer la preuve de la
  proposition. Or, pour tous $x\in H$, on a
  $$\|Tx\| = \sup_{\|y\|\leq 1}\langle Tx, y\rangle =
  \sup_{\|y\|\leq 1}\langle x, T^*y\rangle \leq \|x\|\cdot \|T^*\|$$
  ce qui conclut la preuve.
\end{proof}

\begin{df}
  Soient $H$ un espace de Hilbert et $T\in\mathcal L(H)$. On appelle l'opérateur
  $T^*$ donné par la proposition \ref{adj:hilb} l'adjoint de $T$.
\end{df}

\begin{rem}
  L'adjoint d'un opérateur linéaire $T$ est unique. L'application
  $\mathcal L(H)\to\mathcal L(H): T\mapsto T^*$ est bijective, car elle
  est son propre inverse, c'est-à-dire $T^{**} = T$.
\end{rem}

\begin{exo}
  Soit $T: \ell^2\to \ell^2: (x_1, x_2, \ldots)\mapsto (0, x_1, x_2, \ldots)$.
  Déterminez $T^*$.
\end{exo}
\subsection{Adjoints dans les espaces de Banach}
\begin{df}
  Soient $X$, $Y$ des espaces de Banach et $T\in \mathcal L(X, Y)$.
  On définit l'adjointe de $T$ (également appelée transposée ou application
  duale) par $$T^*: Y^*\to X^*: y^*\mapsto y^*\circ T.$$
\end{df}

\begin{prop}
  Soient $X$, $Y$ des espaces de Banach et $T\in \mathcal L(X, Y)$.
  L'application $T^*$ est linéaire et continue et on a l'égalité
  $\|T^*\| = \|T\|$.
\end{prop}
\begin{proof}
  La preuve de la linéarité est laissée en exercice.

  Pour tout $y^*\in Y^*$, on a $\|T^*(y^*)\| \leq \|y^*\| \cdot \|T\|$ (par
  sous-multiplicativité de la norme opérateur), ce qui montre que
  $\|T^*\| \leq \|T\|$. En particulier, $T^*$ est continue.

  L'inégalité réciproque se déduit du corollaire \ref{hb:a:norme} du théorème
  de Hahn-Banach; pour tout $x\in X$, on a:
  $$\|Tx\| = \max_{\substack{\|y^*\|\leq 1 \\ y^*\in Y^*}}|y^*(T(x))|
  =\max_{\substack{\|y^*\|\leq 1 \\ y^*\in Y^*}}|T^*(y^*)(x)|
  \leq \|T^*\|\cdot \|x\|.$$
\end{proof}
\section{Théorie ergodique}
\subsection{Espaces de Hilbert}
\begin{thm}[Théorème ergodique de Von Neumann]
  Soient $H$ un espace de Hilbert et $T\in\mathcal L(H)$ tel que $\|T\|\leq 1$
  (on dit dans ce contexte que $T$ est une contraction).
  Soit
  $$S_n = \frac{1}{n}\sum_{k=0}^{n-1}T^k.$$
  La suite $(S_n)_{n\in\IN}$ converge dans $\mathcal L(H)$, au sens de la
  topologie forte des opérateurs (voir
  \href{https://en.wikipedia.org/wiki/Strong\_operator\_topology}{Wikipédia}),
  et l'opérateur limite est la projection orthogonale $P$ sur
  $\mathrm{Fix}(T) = \{x\in H\mid T(x) = x\}$. Autrement dit,
  $$\forall x \in H, S_n(x)\xrightarrow{\|.\|} P(x).$$
\end{thm}

\begin{proof}
  Remarquons tout d'abord que l'opérateur $P$ est bien défini et continu;
  $\mathrm{Fix}(T)$ est bien un sous-espace vectoriel fermé de $H$ (exercice)
  et le théorème de Pythagore permet d'affirmer que $P$ est borné.

  Posons $F = \mathrm{adh}( (\mathrm{Id}-T)(H) )$. Tout élément $x\in H$
  s'écrit de manière unique $x = y +z$ avec $y\in F$ et $z\in F^\perp$ (théorème
  du supplémentaire orthogonal d'un sous-espace vectoriel fermé).
  On considère trois cas avant de les combiner pour conclure.

  \textbf{Cas 1}: $y\in (\mathrm{Id}-T)(H)$. Alors il existe $w\in H$
  tel que $y = w - T(w)$. Alors, pour tout naturel $k$,
  $T^k(y) = T^k(w) - T^{k+1}(w)$. On en déduit que $S_n(y) = 1/n (w - T^n(w))$.
  Comme $T$ est une contraction, on a l'inégalité
  $$\|S_n(y)\| \leq \frac{1}{n}(\|w\| + \|T^n(w)\|) \leq \frac{2\|w\|}{n}$$
  qui implique $S_n(y) \xrightarrow[n\to+\infty]{}0$.

  \textbf{Cas 2}: $y\in \mathrm{adh}((\mathrm{Id}-T)(H))$. Alors pour tout
  $\epsilon >0$, il existe $z\in (\mathrm{Id}-T)(H)$ tel que
  $\|y - z\|\leq \epsilon$. Du cas précédent, on déduit qu'il existe
  $N$ tel que pour tout $n\geq N$, $\|S_n(z)\| \leq \epsilon$.
  Comme $T$ est une contraction et $S_n$ est linéaire, on peut affirmer que
  \begin{IEEEeqnarray*}{rCl}
    \|S_n(y)\| & \leq & \|S_n(y -z)\| + \|S_n(z)\| \\
    & \leq & \frac{1}{n}\left\|\sum_{k=0}^{n-1}T^n(y - z)\right\| + \epsilon \\
    & \leq & \frac{1}{n}\sum_{k=0}^{n-1}\|y - z\| + \epsilon \\
    & \leq & 2\epsilon.
  \end{IEEEeqnarray*}
  On a donc montré que $S_n(y) \xrightarrow[n\to+\infty]{}0$.

  \textbf{Cas 3}: $z\in \mathrm{adh}((\mathrm{Id}-T)(H))^\perp =
  \mathrm{Ker}(\mathrm{Id} - T^*)$. Alors $z$ est un point fixe de $T^*$.
  Montrons que $z$ est un point fixe de $T$. On a:
  \begin{IEEEeqnarray*}{rCl}
    \|z-T(z)\|^2 & = & \|z\|^2 + \|T(z)\|^2- 2 \Re\langle z, T(z)\rangle \\
    & = & \|z\|^2 + \|T(z)\|^2- 2 \Re\langle T^*(z), z\rangle \\
    & = & \|T(z)\|^2 - \|x\|^2 \leq 0.
  \end{IEEEeqnarray*}
  Cette inégalité implique que $\|z-T(z)\| = 0$, c'est-à-dire $z = T(z)$.
  Il en découle que $S_n(z) = z$.

  On combine maintenant les cas présentés. Soit $x\in H$, alors il existe
  d'uniques $y\in F$, $z\in F^{\perp}$ tels que $x = y + z$. On en déduit que
  $$\lim_{n\to+\infty}S_n(x) =
  \lim_{n\to+\infty}S_n(y) + \lim_{n\to+\infty}S_n(z)=
  0 + z = z = P(x).$$
  Ceci montre que $S_n$ converge ponctuellement vers l'opérateur annoncé,
  ce qui correspond à l'affirmation du théorème.

  % L'application $\sigma: F \times F^\perp\to H: (y, x) \mapsto y + z$
  % est linéaire,  continue. Comme $F$ et $F^\perp$
  % sont fermés dans $H$, ils sont complets. Par le théorème d'isomorphisme
  % de Banach, $sigma^{-1}$ est continue. Sur $F$, on a par continuité de $S_n$
  % que $S_n = (1/n) (\mathrm{Id} - T^n)$. Sur $F^\perp$, $S_n$ est l'identité.
  % Il est clair que sur chacun de ces sous-espaces, $S_n$ converge en norme
  % (respectivement vers l'application nulle et l'identité).

  % Soit $x\in B(0, 1)$, $x = y + z$ avec $(y, z)\in F\times F^\perp$.
  % Alors $$\|S_n(x) - z \| = \|S_n(y)\| \leq ...?$$ % \frac{2}{n}$$
  % car $T$ est une contraction et
  % $\|y\|^2\leq \|y\|^2 + \|z\|^2 = \|x\|^2 \leq 1$. On déduit de cette
  % dernière inégalité que $S_n$ converge uniformément en $x$ vers sa limite
  % ponctuelle sur la boule
  % unité, ce qui implique que $S_n$ converge en norme vers sa limite ponctuelle.
\end{proof}

\subsection{Espaces de Banach}
\begin{thm}[Théorème ergodique en moyenne]
  Soient $X$ un espace de Banach, $T\in\mathcal L(X)$ tel que $\|T\|\leq 1$
  et $x\in X$. Pour $N\in\IN$, $N\neq 0$, on définit
  $A_N = \frac{1}{N}\sum_{k=0}^{N-1}T^k$.

  Si la suite $(A_Nx)_{N\in\IN}$ a une sous-suite faiblement convergente
  $A_{N_k}x\xrightarrow{\omega}z\in X$, alors $Tz = z$ et
  $A_nx \xrightarrow{\|.\|} z$.
\end{thm}
\begin{proof}
  Admis.
\end{proof}
\section{Propriété de Blum-Hanson (1960)}
\begin{df}
  Soit $X$ un espace de Banach réel et $(x_n)_{n\in\IN}$ une suite d'éléments de
  $X$. On dit que $(x_n)_{n\in\IN}$ a la propriété de Blum-Hanson (B-H) si
  pour toute suite de naturels strictement croissante $(n_k)_{k\in\IN}$,
  $$\lim_{K\to+\infty}\frac{1}{K}\sum_{i=1}^Kx_{n_i} = 0.$$
\end{df}

\begin{prop}
  Soit $X$ un espace de Banach réel et $(x_n)_{n\in\IN_0}$ une suite
  d'éléments de
  $X$. Si $(x_n)_{n\in\IN_0}$ a (B-H), alors $(x_n)_{n\in\IN_0}$ est bornée.
\end{prop}
\begin{proof}
  Supposons que $(x_n)_{n\in\IN_0}$ est non bornée, c'est-à-dire pour toute
  constante $C > 0$, il existe $N$ tel que $\|x_N\|> C$. On construit
  une sous-suite $(x_{k_n})_{n\in\IN_0}$ telle que pour tout naturel $n$,
  $1/n \|x_{k_1} + \cdots + x_{k_n}\| > n$.

  On construit la suite par récurrence.
  Comme $(x_n)_{n\in\IN}$ est non bornée, il existe $k_0$ tel que
  $\|x_{k_0}\| >1$. Ceci couvre le cas de base.

  Supposons maintenant avoir construit les $n-1$ premiers éléments de
  la sous-suite, c'est-à-dire on a $k_1$ < $\ldots$ < $k_{n-1}$ tels que
  $1/(n-1) \|x_{k_1} + \cdots + x_{k_{n-1}}\| > n-1$. On applique
  à nouveau l'hypothèse sur la suite avec
  $C = \max(\|x_{k_1} + \cdots + x_{k_{n-1}}\| + n^2,
  \max_{l\leq k_{n-1}}\|x_l\|+1)$. Il existe $k_n > k_{n-1}$\footnote{
    Comme $C$ excède la norme de tout élément de la suite d'indice
    inférieur à $k_{n-1}$, nécessairement le $k_n$ obtenu est
    strictement supérieur à $k_{n-1}$.
  }
  tel que $\|x_{k_n}\| \geq C$. Une application de \og{}l'inégalité
  triangulaire renversée\fg{} assure
  $$ \frac{1}{n} \|x_{k_0} + \cdots + x_{k_n}\| \geq
  \frac{1}{n} (\|x_{k_n}\| - \|x_{k_0} + \cdots + x_{k_{n+1}}\|)
  \geq \frac{n^2}{n} = n.$$
  On voit facilement que la sous-suite construite ci-avant contredit (B-H).
\end{proof}

\begin{prop}
  Soit $X$ un espace de Banach réel et $(x_n)_{n\in\IN_0}$ une suite
  d'éléments de $X$ ayant (B-H). Alors $x_n\xrightarrow{\omega}0$.
\end{prop}
\begin{proof}
  Par l'absurde, on suppose que la suite ne converge pas faiblement
  vers $0$, c'est-à-dire qu'il existe $x^*\in X^*$ tel que
  $(x^*(x_n))_{n\in\IN}$ ne converge pas vers $0$. On peut donc
  extraire une sous-suite de $(x_n)_{n\in\IN}$ qui sera toujours au moins
  $\epsilon$-distante de $0$ (pour un $\epsilon$ donné par l'hypothèse de
  divergence).

  Comme la suite est bornée, il en est de même
  pour $(x^*(x_n))_{n\in\IN}$. Cette sous-suite admet donc une
  sous-suite convergente (car il s'agit d'une suite d'éléments de $\IR$
  bornée, donc contenue dans un compact). Il existe donc un réel $\alpha\neq 0$,
  $(k_n)_{n\in\IN}$ une suite strictement croissante de naturels tels que
  $x^*(x_{k_n})\to\alpha$ (en prenant une sous-suite de la sous-suite
  qui était $\epsilon$-distante de 0 pour tout $n$, on assure $\alpha\neq 0$).

  Cette sous-suite converge en moyenne de Césaro vers $\alpha$. Or, l'hypothèse
  de (B-H) implique que $1/n(x_{k_0} + \cdots + x_{k_{n-1}})\to 0$.
  En appliquant $x^*$ à la précédente suite, implique que
  $1/n(x^*(x_{k_0}) + \cdots + x^*(x_{k_{n-1}}))\to 0$, contradiction avec
  la convergence en moyenne de Césaro précédemment évoquée.
\end{proof}

\begin{df}
  Un opérateur $T$ a (B-H) si pour toute suite $(n_k)_{k\in\IN}$ de nombres
  naturels strictement croissante,
  $$\forall x\in X,
  \lim_{K\to+\infty}\frac{1}{K}\left\|\sum_{n=0}^{K-1}T^{n_k}(x)\right\| = 0.$$
\end{df}

\begin{thm}[Jones-Kuftinec (1971)] Soit $H$ un espace de Hilbert,
  $T\in\mathcal{L}(H)$, $\|T\|\leq 1$. Alors
  $$(\forall x\in H, T^n(x)\xrightarrow{\omega}0) \mbox{ si et seulement si }
  T \mbox{ a (B-H)}.$$
\end{thm}

\begin{df}
  Un espace vectoriel normé $(X, \|.\|)$ a (B-H) si pour tout
  $T\in\mathcal L(X)$ tel que $\|T\|\leq 1$, les assertions suivantes
  sont équivalentes:
  \begin{enumerate}
  \item $\forall x\in X$, $T^n(x)\xrightarrow{\omega} 0$;
  \item $T$ a (B-H).
  \end{enumerate}
\end{df}

\begin{ex}
  Par exemple, tout espace de Hilbert a (B-H). Les espaces
  de suites $c_0$, $\ell^1$ et $\ell^p$, $1 < p < \infty$ ont (B-H).

  Toutefois, tous les opérateurs d'un espace de Hilbert n'ont pas nécessairement
  (B-H); en 2007, il a été démontré qu'il existe un espace de Hilbert $H$ et
  $T\in \mathcal L(H)$ n'ayant pas (B-H).
\end{ex}

\begin{rem}
  On dit qu'un espace a la propriété de Schur si la convergence faible d'une
  suite équivaut à la convergence forte d'une suite. L'espace
  $\ell^1$ a cette propriété (qui simplifie la preuve de (B-H)).
\end{rem}

%%% Local Variables:
%%% mode: latex
%%% TeX-master: "../analyse3"
%%% End:
