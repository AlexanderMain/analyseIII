\begin{df}
Le produit de convolution de deux fonctions $f, g:\IR\to\IK$
(où $\IK = \IR$ ou $\IC$), noté $f * g$ est la fonction définie par
$$(f * g)(x) = \int_{\IR}f(x-t)g(t)\ud t\mbox{, } x\in\IR.$$
\end{df}

Le produit de convolution est commutatif. Il suffit de poser
le changement de variable $u = x - t$ pour le prouver.

\begin{thm}%[Inégalité de Young pour la convolution]
  Soient $1 < p$, $q <\infty$ tels que $\frac{1}{p} + \frac{1}{q} = 1$,
  $f\in L^p(\IR)$ et $g\in L^q(\IR)$.
  Pour tout $x\in \IR$, $(f * g)(x)$ existe et $\|f * g\|_\infty$ est finie
  et $(f * g)\in L^1(\IR)$.
\end{thm}

\begin{proof}
  On admet l'existence et que $f * g$ est intégrable.

  Remarquons que $|(f  * g)(x)| =
  |\int_{\IR}f(x-t)g(t)\ud t | \leq \|f\|_p \|g\|_q$ par l'inégalité de Hölder
  pour tout $x$, ce qui montre que $f * g$ est bornée.
\end{proof}

\begin{thm}
  Soient $f$, $g\in L^1(\IR)$. Pour presque tout $x$, $f * g$ existe et
  $\|f * g \|_1\leq \|f\|_1 \|g\|_1$.
\end{thm}
\begin{proof}
  Soit $F(t, x) = f(x - t)g(t)$. On admet que $F$ est mesurable sur $\IR^2$.
  On a:
  $$\int_{\IR^2}|F(t, x)|\ud(t, x) =
  \int_\IR \left(\int_\IR|f(x - t)| \ud x\right) |g(t)|\ud t =
  \int_\IR \|f\|_1 |g(t)|\ud t = \|f\|_1 \|g\|_1.$$

  On déduit du théorème de Fubini que $\int_\IR f(x - t) g(t) \ud t$ existe
  pour presque tout $x$.
\end{proof}

\begin{df}[Transformation de Fourier sur $L^1(\IR)$]
Soit $f\in L^1(\IR)$. La transformée de Fourier de $f$, notée $\mathcal F(f)$
ou $\widehat{f}$ est définie par:
$$\widehat{f}(x) = \int_\IR f(t) e^{-2i\pi t x}\ud t\mbox{, } x\in \IR.$$
\end{df}

\begin{prop}
  Soit $f\in L^1(\IR)$. Alors $\widehat{f}$ est bien définie,
  continue sur $\IR$ et
  $$\lim_{|x|\to\infty}\widehat{f}(x) = 0.$$
\end{prop}

Avant de démontrer cette proposition, montrons un autre résultat utile à
sa preuve.
\begin{thm}\label{conv:trans}
  Soit $1 \leq p < \infty$ et $f \in L^p(\IR)$. On pose, pour tout réel
  $t$, $f_t: \IR \to \IR: x\mapsto f(x - t)$.
  Alors
  $$f_t \xrightarrow[t\to 0]{\|.\|_p} f.$$
\end{thm}

\begin{proof}[Esquisse de preuve]
  Posons $V = \{f\in L^p(\IR) \mid f_t \to f \mbox{ quand } t\to 0\}$.
  On va montrer qu'il s'agit d'un sous-espace vectoriel fermé de $L^p(\IR)$
  qui contient les fonctions étagées (intégrables au sens de la norme $\|.\|_p$)

  Puisque $f \mapsto f_t$ est linéaire, on déduit facilement que $V$ est
  un espace vectoriel. Montrons maintenant que $V$ est fermé. On choisit
  $f\in L^p(\IR)$ telle qu'il existe une suite $f^{(n)}\in V$ telle que
  $\|f - f^{(n)}\|_p\to 0$.

  Soit $\epsilon >0$, il existe par hypothèse un rang $N$ tel que
  $\|f^{(N)} - f\|_p < \epsilon/3$. Puisque $f^{(N)} \in V$, il existe
  $T > 0$ tel que pour tout $t$ vérifiant $|t| < T$, on a
  $|f^{(N)}_t - f^{(N)}| < \epsilon/3$. On remarque également
  que $\|f^{(N)} - f\|_p = \|f^{(N)}_t - f_t\|_p$ pour tout réel $t$ (car
  la mesure
  de Lebesgue est invariante par translation). On déduit de l'inégalité
  triangulaire que pour tout $t$ tel que $|t| < T$, on a bien
  $\|f_t - f \|_p < \epsilon$.

  On retrouve dans $V$ les indicatrices d'intervalles $[a, b]$ bornés;
  en effet, notons $f$ une telle indicatrice, alors pour $t$ borné,
  on peut borner $f_t$ par une fonction intégrable indépendante de $t$
  (e.g. l'indicatrice de l'intervalle  $[a - 1, b+1]$, alors on prend
  $|t| < 1$) et on déduit alors
  du théorème de convergence dominée que $f_t \xrightarrow{\|.\|_p} f$.

  Puisque $V$ est un sous-espace vectoriel de $L^p(\IR)$, on retrouve
  dans $V$ toutes les combinaisons linéaires d'indicatrices d'intervalles et
  en particulier les indicatrices des unions finies d'intervalles.
  Comme $V$ est fermé, on y retrouve les indicatrices de tout ensemble
  de mesure finie (cf. construction de la mesure de Lebesgue). Cela implique
  que toute fonction étagée à support de mesure finie
  est dans $V$ et par densité de l'ensemble de ces fonctions, on conclut
  $V = L^p(\IR)$.
\end{proof}

Prouvons maintenant la proposition.
\begin{proof}
  On a $|f(t)e^{2i \pi t x}| \leq |f(t)|$ pour tous réels $x$ et $t$. De
  cette majoration, on déduit que $\widehat{f}(x)$ est bien définie pour
  tout $x$ (car $f\in L^1(\IR)$).
  Le théorème de convergence dominée implique la continuité séquentielle
  de $\widehat{f}$ (on utilise toujours la même majoration).

  Montrons maintenant la seconde assertion de la proposition.
  Pour tout $y\in \IR$, on a les égalités:
  \begin{IEEEeqnarray*}{rClL}
    \widehat{f}(y) & = & -\int_\IR f(x) e^{-2i\pi xy} e^{-i\pi} \ud x &
    \mbox{(car $e^{i\pi} = -1$)}\\
    & = & -\int_\IR f(x) e^{-2i\pi \left(xy - \frac{1}{2}\right)} \ud x &\\
    & = & -\int_\IR f\!\left(t-\frac{1}{2y}\right) e^{-2i\pi ty}\ud t \quad &
    \mbox{(en posant $t = x + \frac{1}{2y}$)}.
  \end{IEEEeqnarray*}
  On en déduit alors que pour tout réel $y$ non nul, on a
  \begin{equation*}
    2\widehat{f}(y) = \int_\IR f(x) e^{-2i\pi xy} \ud x -
    \int_\IR f\!\left(t - \frac{1}{2y}\right)e^{-2i\pi t y} \ud t
  \end{equation*}
  et par l'inégalité triangulaire, on a
  \begin{equation*}
    2|\widehat{f}(y)|\leq
    \int_\IR \left|f(x) -f\left(x - \frac{1}{2y}\right)\right|
    \cdot |e^{-2i\pi xy}|\ud x = \|f - f_{\frac{1}{2y}}\|_1.
  \end{equation*}

  Par le théorème \ref{conv:trans}, on a bien que
  $|\widehat{f}(y)| \xrightarrow{|y|\to + \infty} 0$, ce qu'on voulait montrer.
\end{proof}

\begin{exo}
  Reprenons les notations du théorème \ref{conv:trans}. Déduire de ce
  théorème que
  $T: \IR\to L^1(\IR): x\mapsto f_x$ est uniformément continue.
\end{exo}

\begin{thm}
  Soient $f$, $g\in L^1(\IR)$. On a l'égalité $\widehat{f * g} = \widehat{f}
  * \widehat{g}$.
\end{thm}

\begin{proof}
  On admet sans preuve que les hypothèses du théorème de Fubini sont vérifiées.
  \begin{IEEEeqnarray*}{rCl}
    \widehat{f * g} & = & \int_\IR (f * g)(x) e^{-2i\pi xy}\ud x \\
    & = &\int_\IR\left(\int_\IR f(x - t) g(t) \ud t\right) e^{-2i\pi xy}\ud x \\
    & = &\int_\IR\int_\IR f(x - t) e^{-2i\pi (x - t)y} g(t) e^{2i\pi ty}
    \ud(t, x) \\
    & = & \int_\IR \left(
      \int_\IR f(x - t) e^{-2i\pi (x - t) y} g(t) e^{2i\pi ty} \right)
  \end{IEEEeqnarray*}
\end{proof}

%%% Local Variables:
%%% mode: latex
%%% TeX-master: "../analyse3"
%%% End:
