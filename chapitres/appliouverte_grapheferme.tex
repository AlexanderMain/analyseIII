\section{Théorème de l'application ouverte}
On rappelle qu'une application $f(X, \tau_X)\to(Y, \tau_Y)$ est ouverte
si l'image de tout ouvert de $X$ par $f$ est un ouvert de $Y$.
\'{E}nonçons le théorème de l'application ouverte.

\begin{thm}[Théorème de l'application ouverte]
  Soient $E$, $F$ deux espaces de Banach et $T\in\mathscr{L}(E, F)$.

  Si $T$ est une surjection, alors $T$ est ouverte.
\end{thm}

Prouvons tout d'abord un résultat qui nous permet de montrer
qu'un opérateur est ouvert.
\begin{prop}
  Soit $T: E\to F$ une application linéaire. $T$ est ouverte si et seulement
  si il existe $r>0$ tel que $T\left(B_E(0, 1)\right)\supseteq B_F(0, r)$.
\end{prop}

Remarquez que de manière équivalente, $T$ linéaire est ouverte si et seulement
si il existe $s>0$ tel que $T\left(B_E(0, s)\right)\supseteq B_F(0, 1)$.

\begin{proof}
  Supposons $T$ ouverte. Alors l'image par $T$ de $B_E(0, 1)$ est un ouvert
  de $F$ contenant $0$, ce qui fournit le résultat.

  Réciproquement, supposons qu'il existe $r>0$ tel que
  $T\left(B_E(0, 1)\right)\supseteq B_F(0, r)$. Soit $O$ un ouvert de $E$.
  Soit $T(x)$ un élément de $T(O)$. Il existe $s>0$ tel que $B_E(x, s)%
  \subseteq O$. Puisque $B_E(x, s) = x + s B_E(0, 1)$, on a que $T(O)$ contient
  l'ensemble $T(x) + s T(B_E(0, 1))$.
  Par hypothèse, ce sous-ensemble contient $T(x) + s B_F(0, r)$, ce qui
  implique $B_F(T(x), r\cdot{}s)\subseteq T(O)$.
\end{proof}

Nous pouvons désormais prouver le théorème. Nous utilisons les résultats
suivants, implicitement:
\begin{exo}
  Soient $A\subseteq E$ et $T\in\mathscr{L}(E, F)$.
  \begin{enumerate}
  \item On a que $A + A$ comprend $2A$;
  \item Si $A$ est convexe, $A+A = 2A$;
  \item Si $A$ est convexe, alors $T(A)$ est convexe;
  \item Si $A$ est convexe, alors son adhérence aussi.
  \end{enumerate}
\end{exo}
\begin{proof}
  Montrons premièrement qu'il existe $r>0$ tel que $\mathrm{adh}(T(B_E(0, 1)))%
  \supseteq B_F(0, r)$. Afin de faire cela, on utilise le théorème de Baire.

  Pour tout naturel $n\geq 1$, on pose $F_n = n\cdot\mathrm{adh}(T(B_E(0, 1)))$
  qui est un fermé de $F$. Dès lors, l'union des $F_n$ est égale à $F$
  par surjectivité de $T$. En effet, il suffit de constater que
  $F_n\supseteq n\cdot T(B_E(0, 1)) =T(B_E(0, n))$.

  Dès lors le théorème de Baire assure qu'il existe $n_0$ tel que
  l'intérieur de $F_{n_0}$ est non vide. On en déduit que l'intérieur
  de $F_1$ est non vide. Soit $x_0$ un élément de l'intérieur de $F_1$.
  Alors il existe $s>0$ tel que
  $B_F(x_0, s)\subseteq  \mathrm{adh}(T(B_E(0, 1)))$. Puisque
  $\mathrm{adh}(T(B_E(0, 1)))$ est symétrique ($A$ est symétrique
  si pour tout élément $a$ de $A$, $-a$ est dans $A$), $B_F(-x_0, s)$
  est également contenue dans $\mathrm{adh}(T(B_E(0, 1)))$. Par convexité
  de $\mathrm{adh}(T(B_E(0, 1)))$, on déduit que
  $B_F(0, s)\subseteq \mathrm{adh}(T(B_E(0, 1)))$.

  Maintenant, on a montré que $\exists r' > 0$, $\mathrm{adh}(T(B_E(0, 1))%
  \supseteq B_F(0, r')$. On en déduit que $\mathrm{adh}(T(B_E(0, r))%
  \supseteq B_F(0, 1)$ où $r= 1/r'$. Pour conclure, on doit montrer l'existence
  de $s>0$ tel que $B_F(0, 1)\subseteq T(B_E(0, s))$.

  Soit $y\in F$, $\|y\| < 1$. Alors, $y\in\mathrm{adh}(T(B_E(0, r)))$.

  Par définition d'adhérence, il existe $x_0\in B_E(0, r)$ tel que
  $\|y-T(x_0)\| < \frac{1}{2}$. Dès lors, $y-T(x_0)$ appartient
  à $\mathrm{adh}(T(B_E(0, r/2)))$. Il existe donc
  $x_1\in B_E(0, r/2)$ tel que $\|y-T(x_0 + x_1)\| < \frac{1}{2^2}$.

  \`{A} l'étape $n$, on a $y - T(x_0 + \cdots + x_{n-1})\in%
  \mathrm{adh}(T(B(0, \frac{r}{2^n})))$. Il existe donc
  $x_n\in B(0, \frac{r}{2 ^{n}})$ tel que
  $\|y - T(x_1 + \cdots + x_n)\| < \frac{1}{2^{n+1}}$.

  La série des $x_n$ converge car elle est absolument convergente (on a
  pour tout naturel $n$, $\|x_n\|<\frac{r}{2^{n}}$). Soit $x$ la limite
  de cette série. Alors, $\|x\|\leq 2r$ et on a $T(x) = y$.
  D'où $y\in T(B(0, 2r))$.

  Il suffit donc de prendre $s = 2r$ pour conclure.
\end{proof}
\begin{cor}[Théorème d'isomorphisme de Banach]
  Soient $E$, $F$ deux espaces de Banach, $T\in\mathscr L(E, F)$.
  Si $T$ est bijective, alors elle est bicontinue (c'est-à-dire sa
  réciproque est également continue).
\end{cor}

\begin{proof}
  Toute application bijective, continue et ouverte est un
  homéomorphisme. Le résultat est immédiat par le théorème de
  l'application ouverte.
\end{proof}

\begin{cor}\label{ao:de}
  Soient $\|.\|_1$, $\|.\|_2$ deux normes sur un même espace
  vectoriel $E$ telles que $(E, \|.\|_i)$ est un espace de Banach
  pour $i =1, 2$.

  S'il existe $c > 0$ tel que pour tout $x\in E$, $\|x\|_1\leq c\|x\|_2$ 
  (c'est-à-dire que la norme $\|.\|_2$ domine la norme $\|.\|_1$), alors elles
  sont équivalentes.
\end{cor}

\begin{proof}
  L'application $\mathrm{Id}: (E, \|.\|_2)\to (E, \|.\|_1)$ est linéaire,
  continue et bijective. Par le théorème d'isomorphisme de Banach, l'identité
  est bien un homéomorphisme.
\end{proof}

\begin{prop}
  Soient $E$, $F$ des espaces de Banach. L'ensemble des applications
  linéaires et continues ouvertes $O(E, F)$ est un ouvert de $\mathscr L(E, F)$.
\end{prop}

Avant de prouver la proposition, on introduit le lemme suivant:
\begin{lem}
  Soit $\varepsilon \in \left]0, 1\right[$ et $A\subseteq F$ un sous-ensemble
  borné tels que $A\subseteq T(B_E(0, 1)) + \varepsilon A$.
  Alors $A\subseteq \frac{1}{1-\varepsilon} T(B_E[0, 1])$.
\end{lem}
\begin{rem}
  Le lemme ne requiert pas la complétude de $F$.
\end{rem}

\begin{proof}
  Soit $a = a_0$ un élément de $A$. Il existe $x_0$ un élément
  de $B_E(0, 1)$ et $a_1\in A$ tels que $a_0 = T(x_0) + \varepsilon a_1$.
  De même, il existe $x_1$ un élément de $B_E(0, 1)$ et $a_2$ un élément
  de $A$ tels que $a_1 = T(x_1) + \varepsilon a_2$, ce qui implique
  que
  $$a_0 = T(x_0) + \varepsilon (T(x_1) +\varepsilon a_2).$$

  En continuant ainsi, pour $a_n\in A$, il existe $x_n\in B_E(0, 1)$
  et $a_{n+1}\in A$ tels que $a_n = T(x_n) + \varepsilon a_{n+1}$. On en déduit
  $$a_0 = T(x_0 + \varepsilon x_1 + \cdots + \varepsilon^nx_n)+%
  \varepsilon^{n+1}a_{n+1}.$$
  Puisque $A$ est borné et que $0 < \varepsilon < 1$, on a
  $\varepsilon^na_n \to 0$. De plus, on a:
  $$\sum_{k=0}^{\infty}\varepsilon^k\|x_k\| \leq%
  \sum_{k=0}^{\infty}\varepsilon^k = \frac{1}{1-\varepsilon}$$

  Puisque $E$ est complet, la série des $\varepsilon^kx_k$ converge
  vers un élément $x\in E$. Alors $a_0 = T(x)$ et $\|x\|\leq
  \frac{1}{1-\varepsilon}$ implique $(1-\varepsilon)x \in B_E[0, 1]$.
\end{proof}
Prouvons la proposition.
\begin{proof}
  Soit $T$ une application linéaire et continue ouverte de $E$ dans $F$.
  On recherche $r>0$ tel que $B_{\mathscr L(E, F)}(T, r)\subseteq O(E, F)$.
  Puisque $T$ est une application ouverte, il existe $s>0$ tel que
  $B_F(0, s)\subseteq T(B_E(0, 1))$. Soit $S\in B_{\mathscr L(E, F)}(T, s/2)$,
  montrons que $S$ est une application ouverte.

  Par hypothèse sur $S$, $\|T-S\| < s/2$. Cela implique que pour tout
  $x\in B_E(0, 1)$, $\|T(x) - S(x)\|< s/2$, d'où $\|T(x)\|\leq \| S(x)\| + s/2$.

  En termes d'inclusions d'ensembles, on obtient
  $$T(B_E(0, 1)) \subseteq S(B_E(0, 1))+\frac{1}{2}B_F(0, s).$$
  De l'hypothèse, on déduit,
  $$B_F(0, s) \subseteq S(B_E(0, 1))+\frac{1}{2}B_F(0, s).$$

  Par le lemme, on a $B_F(0, s)\subseteq\frac{1}{1-\frac{1}{2}}S(B_E(0, 1))$.
  On en déduit que $B_F(0, \frac{s}{2})\subseteq S(B_E(0, 1))$
\end{proof}

\section{Théorème du graphe fermé}
Soient $E$, $F$ des espaces vectoriels normés, et $T: E\to F$.

\begin{df}
  On note $\mathrm{Gr}(T) = \{(x, T(x))\mid x\in E\}$ le graphe
  de $T$, qui est un sous-ensemble de $E\times F$.
\end{df}

On munit $E\times F$ de la norme $\|(x, y)\| = \max (\|x\|_E, \|y\|_F)$.
Cette norme engendre bien la topologie produit (pour le prouver, vous
pouvez procéder comme suit: dans un premier temps, montrer que les rectangles
ouverts sont bien des ouverts pour cette norme et ensuite montrer que
les boules sont bien des ouverts au sens de la topologie produit).

\begin{exo}
  Si $T$ est continue, alors $\mathrm{Gr}(T)$ est fermé.
\end{exo}

Le théorème du graphe fermé est une réciproque partielle
pour le résultat précédent; il nécessite comme hypothèse additionnelle
la complétude de l'espace de départ et de l'espace d'arrivée.

\begin{thm}
  Supposons que $E$ et $F$ sont complets et que $T$ est linéaire.
  Alors $T$ est continue si et seulement si le graphe de $T$ est
  fermé.
\end{thm}

\begin{proof}
  Supposons que le graphe de $T$ est fermé, c'est-à-dire si
  $(x_n)_n$ est une suite d'éléments de $E$ et
  $(x, y)\in E\times F$, si $(x_n, T(x_n))\to (x, y)$, alors
  $y = T(x)$.

  On définit une nouvelle norme sur $E$, pour $x\in E$, on
  pose $|\!|\!|x |\!|\!|= \|x\|_E + \|T(x)\|_F$, appelée norme du graphe.
  Montrons que $E$ muni de cette norme est complet.

  Soit $(x_n)_n$ une suite de Cauchy au sens de la norme du graphe.
  Alors pour tout $\varepsilon >0$, il existe $n_0$ naturel,
  pour tous $p, q\geq n_0$,
  $$\|x_q-x_p\|_E + \|T(x_q)-T(x_p)\|_F\leq \varepsilon.$$
  Il s'en suit que $(x_n)_n$ et $(T(x_n))_n$ sont de Cauchy (au
  sens de $\|.\|_E$ et de $ \|.\|_F$ respectivement) et convergent
  respectivement vers $x\in E$ et $y\in F$.
  Puisqu'on suppose le graphe de $T$ fermé, on a $y = T(x)$.
  Dès lors $\||x_n -x \|| = \|x_n - x\| + \|T(x_n)-T(x)\|$
  qui converge bien vers $0$.

  De plus, on remarque que la norme du graphe (de $T$)
  domine la norme $\|.\|_E$. Par le corollaire \ref{ao:de},
  les deux normes sont équivalentes. En particulier,
  il existe $c > 0$ tel que pour tout $x$ dans $E$,
  $\|x\|_E + \|T(x)\|_F\leq c\|x\|_E$. Ceci implique
  que $\|T(x)\|_F\leq c\|x\|_E$ donc $T$ est continue.
\end{proof}

\begin{prop}
  Soit $E$ un espace de Banach. Soit $T: E\to E^*$ linéaire
  telle que
  $$\forall x, y \in E, (Tx)(y) = (Ty)(x)$$
  (c'est-à-dire $T$ est symétrique). Alors $T$ est continue.
\end{prop}

\begin{proof}
  Montrons que le graphe de $T$ est fermé (on a bien $E$ et $E^*$
  complets). Soit $(x_n)_n$ une suite d'éléments de $E$, soit
  $(x, S)\in E \times E^*$ tels que $(x_n, Tx_n)\to (x, S)$.
  Soit $z\in E$. Montrons que $S(z) = (Tx)(z)$.
  On a, pour tout $n$, $(Tx_n)(z) = (Tz)(x_n)$. Le premier
  membre converge vers $S(z)$ par hypothèse et le second
  vers $(Tz)(x)$ par continuité. D'où $S = Tx$ (unicité
  de la limite).
\end{proof}

\begin{exo}
  On a utilisé implicitement dans l'exercice précédent
  que si $(T_n)_n$, est une suite d'opérateurs $E\to F$ convergeant
  vers un opérateur $T$ au sens de la norme opérateur,
  alors pour tout $x$ dans $E$, $T_n(x) \to T(x)$. Prouvez
  cette affirmation.
\end{exo}

\begin{exo}
  Soient $E$ un espace de Banach et $T: E\to E^*$ linéaire vérifiant
  $\forall x\in E, \langle Tx, x\rangle\geq 0$. Montrer que $T$
  est continue.
\end{exo}


\begin{exo}
  Soit $(c_{00})$ l'espaces des suites à support fini, c'est-à-dire
  des suites qui ont un nombre fini d'éléments non nuls. On munit
  $(c_{00})$ de la norme du maximum, ie. pour tout $x=(x_n)_n\in c_{00}$,
  $\|x\| = \max_{n}|x_n|$.

  Soit $T:c_{00}\to c_{00}$ définie par $T((x_n)) = ((n+1)x_n)_n$
  pour tout $(x_n)_n$ dans $c_{00}$.
  \begin{enumerate}
  \item $T$ est-elle linéaire?
  \item $T$ est-elle continue?
  \item Qu'en est-il du graphe de $T$?
  \item Qu'en déduit-on?
  \end{enumerate}
\end{exo}

%%% Local Variables:
%%% mode: latex
%%% TeX-master: "../analyse3"
%%% End:
