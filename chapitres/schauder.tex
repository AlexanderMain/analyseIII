%\section{Bases de Schauder}
\section{Introduction et définition}
Soit $X$ un espace de Banach. On a vu précedemment que toute base algébrique
de $X$ est soit finie, soit non dénombrable.
Si on considère $(e_j)_{j\in J}$ une base algébrique de $X$, et $(z_n)_n$ une
suite d'éléments de $X$ convergeant vers un élément $z\in X$, il n'y a aucune
raison (en dimension infinie) d'avoir une convergence composante par composante.
\begin{exo}
  Soit $\ell^\infty$ l'espace des suites de nombres réels bornées.
  Posons $e_i=(\delta_{i, n})_{n\in\IN}$
  et $x=(x_n)_n$, la suite définie par $x_n = 1/2^n$ pour tout naturel $n$.
  \begin{enumerate}
  \item Montrez que la suite $(x^{(k)})_{k\in\IN}$ définie par
    $x^{(k)}=\sum_{n=0}^kx_ne_n$ converge vers $x$ dans $\ell^\infty$.
  \item Montrer que la famille de vecteurs $V:= \{e_i \mid i\in \IN\}\cup \{x\}$
    est linéairement indépendante (\textbf{rappel}: une famille de vecteurs est
    linéairement indépendante si toute sous-famille finie l'est).
  \item Du point précédent et du lemme de Zorn, on déduit qu'il existe une
    base $B$ de $\ell^\infty$ contenant $V$ (cette affirmation ne fait pas
    partie de l'exercice, vous n'êtes pas obligé de le montrer).
    Argumentez que la suite $(x^{(k)})_{k\in\IN}$ introduite ci-avant
    montre qu'il n'y a pas convergence composante par composante
    dans la base $B$.
  \end{enumerate}
\end{exo}

On introduit les bases de Schauder, plus appropriées pour les espaces de Banach.
\begin{df}
  La suite de vecteurs $(e_n)_{n\in\IN}$ est appelée une base de Schauder de $X$
  si pour tout $x\in X$, il existe une unique suite $(x_n)_{n\in\IN}$ de
  scalaires telle que
  $$x = \sum_{n=0}^{+\infty}x_ne_n.$$
\end{df}

\begin{ex}
  Les espaces de suites $c_0$ et $\ell^p$, $1\leq p < \infty$ admettent
  des bases de Schauder; il s'agit de la famille des suites
  $e_i=(\delta_{i, n})_{n\in\IN}$, $i\in\IN$.
\end{ex}

\begin{prop}
  Si $X$ admet une base de Schauder, alors $X$ est séparable.
\end{prop}
\begin{proof}
  Soit $(e_n)_{n\in\IN}$ une base de Schauder de $X$. Alors l'espace
  vectoriel engendré par cette base est dense dans $X$. Posons
  $V = \langle e_i \mid i\in\IN\rangle_\IK$. Il suffit de trouver
  un sous-ensemble dénombrable de $X$ dont l'adhérence contient $V$.

  Soit $\mathbb L = \IQ$ si $\IK = \IR$ et $\IQ(i)$ sinon. Alors
  $\langle e_i \mid i\in\IN\rangle_\mathbb{L}$ est un sous-ensemble
  dénombrable de $X$ dont l'adhérence contient $V$ car $\mathbb L$ est
  dense dans $\IK$. Ceci termine la preuve.
\end{proof}
\begin{ex}
  L'espace de suites $\ell^\infty$ n'admet pas de base de Schauder, car
  il n'est pas séparable.
\end{ex}

La réciproque de cette dernière proposition n'est pas vraie. Il existe un espace
de Banach séparable qui n'admet pas de base de Schauder. Per Enflo a en
1974 construit un sous-espace de $L^p[0, 1]$, $1\leq p < \infty$ qui
n'a pas de base de Schauder. L'espace $L^p[0, 1]$ admet une base de Schauder
appelée système de Haar
(\href{https://fr.wikipedia.org/wiki/Ondelette_de_Haar#Le_syst\%C3%
  \%A8me_de_Haar}{Wikipédia}).

\section{Projections et convergence composante par composante}
Soit $X$ un espace de Banach admettant une base de Schauder et $(e_n)_{n\in\IN}$
une base de Schauder normalisée de $X$.
Pour $x = \sum_{n\geq 0}x_ne_n\in X$ et $j\geq 0$ naturel, on pose
$P_j(x)=\sum_{n=0}^jx_ne_n$. Les applications $P_j$ sont linéaires.

On introduit une nouvelle norme sur $X$ en
posant $||x||' = \sup_{j\geq 0} \|P_j(x)\|$ pour tout $x\in X$. Elle est
bien définie car $P_j(x)\xrightarrow[j\to+\infty]{} x$ (car $(e_n)_{n\in\IN}$
est une base de Schauder de $X$).
\begin{exo}
  Montrer que l'application $\|.\|'$ introduite ci-avant est une norme.
\end{exo}
\begin{prop}\label{pj:prop}
  $(X, \|.\|')$ est un espace de Banach.
\end{prop}
\begin{proof}
  Soit $(x^{(n)})_{n\in\IN}$ une suite de Cauchy dans $(X, \|.\|')$, $x^{(n)}=
  \sum_{k=0}^{+\infty}x^{(n)}_ke_k$ pour tout naturel $n$. Montrons d'abord que
  pour $k$ fixé, $(x^{(n)}_k)_{n\in\IN}$ est une suite de Cauchy.

  Soit $\epsilon > 0$, il existe $N_\epsilon$ tel que pour tous
  $n, m\geq N_\epsilon$,
  \begin{equation}\label{pj:pv:1}
    \|x^{(n)} - x^{(m)}\|' =
    \sup_{j\geq 0}\left\|\sum_{l = 1}^j(x^{(n)}_l-x^{(m)}_l)e_l\right\|
    \leq \epsilon.
  \end{equation}
  En utilisant l'inégalité triangulaire, on a pour tous $n, m\geq N_\epsilon$:
  $$|x^{(n)}_k - x^{(m)}_k|=\|(x^{(n)}_k - x^{(m)}_k)e_k\|\leq
  \left\|\sum_{l = 1}^{k}(x^{(n)}_l-x^{(m)}_l)e_l\right\| +
  \left\|\sum_{l = 1}^{k-1}(x^{(n)}_l-x^{(m)}_l)e_l\right\|\leq 2\epsilon.$$

  Comme $\IK$ est complet, il existe donc $x_k\in\IK$ tel que
  $x_k^{(n)}\to x_k$.

  Montrons maintenant que la série $\sum_{k=0}^{+\infty} x_k e_k$ converge
  dans $(X, \|.\|)$. Il suffit de montrer que la suite des sommes partielles
  est de Cauchy. Soit $\epsilon > 0$, alors la série $\sum_{k=0}^{+\infty}
  x_k^{(N_\epsilon)}e_k$ converge au sens de la norme $\|.\|$ (comme $(e_k)_k$
  est une base de Schauder et qu'on considère l'écriture dans cette base
  de $x^{(N_\epsilon)}$). Il existe donc $M_\epsilon$ tel que
  pour tous $m, r\geq \max\{N_\epsilon, M_\epsilon\}$, $m< r$,
  \begin{equation*}
    \left\|\sum_{k=m+1}^rx_k^{(N_\epsilon)}e_k\right\|\leq \epsilon.
  \end{equation*}
  Soient $m, r\geq \max\{N_\epsilon, M_\epsilon\}$, $m < r$. On a,
  en utilisant (\ref{pj:pv:1}) avec $m\to+\infty$ et $n=N_\epsilon$:
  \begin{IEEEeqnarray*}{rCl}
    \left\|\sum_{k=m+1}^rx_ke_k\right\| &
    \leq & \left\|\sum_{k=m+1}^r(x_k-x^{(N_\epsilon)}_k)e_k\right\|
    + \left\|\sum_{k=m+1}^rx_k^{(N_\epsilon)}e_k\right\| \\
    & \leq & \left\|\sum_{k=1}^r(x_k-x^{(N_\epsilon)}_k)e_k\right\|
    + \left\|\sum_{k=1}^m(x_k-x^{(N_\epsilon)}_k)e_k\right\|
    + \epsilon \leq 3\epsilon.
  \end{IEEEeqnarray*}
  Ceci montre que $\sum_{k=0}^{+\infty} x_k e_k$ converge vers $x\in X$,
  au sens de la norme $\|.\|$, par complétude de $(X, \|.\|)$
  (il fallait s'assurer que cette série converge bien pour
  poser $x$).

  Montrons que cette série converge vers $x$ au sens de $\|.\|'$.
  Il faut montrer que
  $$\sup_{j\geq 0}\left\|P_j\left(x - \sum_{k=0}^nx_ke_k\right)\right\|\to 0.$$
  Soit $\epsilon > 0$. Il existe $N$ tel que tout $n\geq N$ vérifie
  $$\left\|\sum_{k=n}^{+\infty}x_ke_k\right\|< \varepsilon.$$
  Soient $j\geq 0$ et $n\geq N$. Posons
  $v_{j, n}= \|P_j(x - \sum_{k=0}^nx_ke_k)\|$. Si $j\leq n$, alors $v_{j, n}=0$.
  Sinon, on a l'inégalité
  $$ v_{j, n} =
  \left\|\sum_{k=n+1}^jx_ke_k\right\| \leq
  \left\|\sum_{k=n+1}^{+\infty}x_ke_k\right\| +
  \left\|\sum_{k=j+1}^{+\infty}x_ke_k\right\|\leq 2\epsilon.$$

  Comme cette majoration est uniforme en $j$, on a bien montré que la
  série $\sum_{k=0}^{+\infty}x_ke_k$ converge vers $x$ au sens de $\|.\|'$.

  Soient $\varepsilon > 0$ et $N_\varepsilon$ associé. En utilisant
  (\ref{pj:pv:1}) avec $m\to\infty$, on déduit que pour tout $n\geq N_\epsilon$,
  pour $j$ fixé, on a:
  $$ \|P_j(x^{(n)} - x)\| =
  \left\|\sum_{l = 1}^j(x^{(n)}_l-x_l)e_l\right\|
  \leq \epsilon.$$
  Ceci montre que $\|x^{(n)}-x\|'\leq \epsilon$ (la majoration ci-avant
  est uniforme en $j$), ce qui termine la preuve.
\end{proof}

\begin{cor}
  Les normes $\|.\|$ et $\|.\|'$ sont équivalentes sur $X$.
\end{cor}
\begin{proof}
  Comme les deux normes induisent une structure d'espace de Banach sur $X$,
  il suffit de montrer au vu du corollaire \ref{ao:de} du théorème
  d'isomorphisme de Banach qu'une norme domine l'autre.

  Soit $x\in X$. Comme $P_j(x)\xrightarrow[j\to+\infty]{}x$,
  on a $\|x\|'\geq \|x\|$. Ceci montre l'équivalence des normes.
\end{proof}

\begin{cor}
  Les application $P_j$ sont continues et $\sup_{j\geq 0}\|P_j\| < +\infty$.
\end{cor}
\begin{proof}
  Soit $K > 0$ tel que pour tout $x\in X$, $\|x\|'\leq K\|x\|$.
  Soit $j\geq 0$. Alors pour tout $x\in X$, $\|P_j(x)\|\leq K\|x\|$, ce qui
  montre la continuité de $P_j$.

  La seconde assertion est une conséquence directe du théorème de
  Banach-Steinhaus, car pour tout $x\in X$, $\sup_{j\geq 0}\|P_j(x)\|=\|x\|'$
  est fini.
\end{proof}
\begin{prop}
  Soit $n$ un naturel. La forme linéaire $e^*_n$ définie par
  $e^*_n(e_j) = \delta_{n, j}$ est continue.
\end{prop}
\begin{proof}
  Soit $K > 0$ tel que pour tout $x\in X$, $\|x\|'\leq K\|x\|$.
  Soit $x=\sum_{k=0}^{+\infty}x_ke_k\in X$. Alors:
  $$|e^*_n(x)| = |x_n| = \|x_ke_k\| \leq \left\|\sum_{k=0}^{n}x_ke_k\right\|
  + \left\|\sum_{k=0}^{n-1}x_ke_k\right\|\leq 2K\|x\|.$$
\end{proof}
\begin{cor}[Convergence composante par composante]
  Soient $(z_k)_{k\in\IN}$ une suite d'éléments de $X$ et $z\in X$ tels que
  $z_k\to z$. Alors $e^*_n(z_k)\to e^*_n(z)$.
\end{cor}

\begin{prop}
  Il existe $C > 0$, $\forall p$, $q\in\IN$, $p\leq q$,
  $\forall a_0$, $\ldots$, $a_q$,
  $$\left\| \sum_{k=0}^p a_ke_k\right\| \leq
  C \left\| \sum_{k=0}^q a_ke_k\right\|$$
\end{prop}

\begin{proof}
  Soit $F = \{(a_n)_{n\in\IN}\mid \sum_{n=0}^\infty a_ne_n \mbox{ converge}\}$.
  On munit $F$ de la norme $\|(a_n)_{n\in\IN}\|_F =
  \sup_{j\geq 0}\|\sum_{n=0}^j a_ne_n\|$; $F$ est complet pour cette norme
  (se référer à la preuve de la proposition \ref{pj:prop}).
  L'application
  $$T:F\to X: (a_n)_n\mapsto\sum_{n=0}^{+\infty}a_nx_n$$
  est linéaire, bijective et continue car $\|\sum_{n=0}^{+\infty}a_nx_n\|\leq
  \|(a_n)_n\|_F$. Le théorème d'isomorphisme de Banach permet d'affirmer
  que $T^{-1}: X\to F$ est continue, c'est-à-dire qu'il existe une constante
  $C$ telle que pour tout $x= \sum_{n=0}^{+\infty}a_nx_n$, $\|T^{-1}(x)\|_F
  \leq C \|x\|$.

  Soient $a_0, \ldots, a_q\in\IK$. En appliquant la dernière inégalité à
  $x = a_0e_0 + \cdots + a_qe_q$, on a:
  $$\left\| \sum_{k=0}^p a_ke_k \right\| \leq
  \left\| (a_0, \ldots, a_q, 0, \ldots) \right\|_F \leq
  C \left\| \sum_{k=0}^q a_ke_k \right\|.$$
\end{proof}
%%% Local Variables:
%%% mode: latex
%%% TeX-master: "../analyse3"
%%% End:
