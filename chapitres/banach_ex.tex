Dans ce chapitre nous présentons plusieurs exemples d'espaces de Banach,
notamment les espaces $\ell^p$ de suites et deux exemples d'espaces de fonctions.

La lettre $\mathbb{K}$ est utilisée pour représenter le
corps des nombres réels $\mathbb{R}$ ou celui des nombres
complexes $\mathbb{C}$.
\section{Exemples tirés du cours d'Analyse II}

Nous avons vu l'année passée que tous les espaces vectoriels
de dimension finie étaient complets.
\begin{prop}
  Soit $(E, \|.\|)$ un espace vectoriel sur $\mathbb{K}$
  de dimension finie. Il s'agit d'un espace de Banach.
\end{prop}
\begin{proof}(Esquisse de preuve)

  \textbf{Remarque}: un point intéressant de la dimension finie est qu'on
  peut travailler, si on le souhaite, avec sa norme
  préférée étant donné qu'elles sont toutes équivalentes.

  Etant donné une suite de Cauchy, il est évident (via l'inégalité
  triangulaire) que chaque composante (en se fixant au préalable
  une base) est de Cauchy et donc converge dans $\mathbb{K}$

  Ceci fournit un candidat limite et la vérification se fait
  facilement en prenant par exemple la norme $\|.\|_1$.
\end{proof}

Les espaces de Hilbert sont un autre type d'espace de Banach
que nous avons étudié en analyse II. Ils sont complets par
définition.

\section{Espaces de suites $\ell^p$}
\begin{df}
  Soit $1\leq p<\infty$ un réel. On appelle $\ell^p$ (noté aussi
  $\ell^p(\mathbb{N})$) l'ensemble  de suites suivant:
  \begin{equation*}
    \left\{(x_n)_{n\in\mathbb{N}}\mid
       \forall  n\in\mathbb{N}, x_n\in\mathbb{K} \mbox{ et }
      \sum_{n=0}^\infty|x_n|^p<+\infty\right\}
  \end{equation*}

  On définit l'application $\|.\|_p:\ell^p\to\left[0, +\infty\right[$
  par
  $$ \forall  x=(x_n)_{n\in\mathbb{N}}\in \ell^p, \|x\|_p=
  \left(\sum_{n=0}^\infty|x_n|^p\right)^{1/p}$$
\end{df}

Il n'est pas clair à première vue que $\ell^p$ est un espace
vectoriel, lorsque $p>1$. Efforçons de montrer des résultats
permettant une preuve courte de cela, tout en délivrant
des résultats simplifiant la démonstration du fait
que l'application $\|.\|_p$ définit une
norme sur $\ell^p$.
Nous aurons besoin d'une première inégalité célèbre afin
de prouver ce résultat.

\begin{thm}[Inégalité de Hölder]
  Soient $n\geq 1$ un nombre naturel, $p, q>1$ deux réels
  conjugués, c'est-à-dire vérifiant l'égalité:
  $$\frac{1}{p}+\frac{1}{q}=1$$

  Pour tous $x=(x_1, \ldots, x_n), y=(y_1, \ldots, y_n)\in
  \mathbb{K}^n$, l'inégalité suivante est vérifiée:
  $$\sum_{k=1}^n|x_k\cdot y_k|\leq
  \left(\sum_{k=1}^n|x_k|^p\right)^{\frac{1}{p}}
  \left(\sum_{k=1}^n|y_k|^q\right)^{\frac{1}{q}}$$
\end{thm}
\begin{proof}
  Prouvons préalablement que pour tous réels
  $a, b >0$, l'inégalité suivante est vérifiée:
  \begin{equation}\label{hold:pv1}
    a\cdot b\leq\frac{a^p}{p}\cdot\frac{b^q}{q}
  \end{equation}

  Il est aisé de vérifier, à l'aide de la dérivation, que
  la fonction $\ln:\left]0, +\infty\right[\mapsto\mathbb{R}$
  est concave, c'est-à-dire:

  $$  \forall  t\in [0, 1],  \forall  x, y >0,
  \ln(tx+(1-t)y)\geq t\ln(x)+ (1-t)\ln(y)$$

  Puisque $p, q >1$, on a:
  \begin{IEEEeqnarray*}{rCl}
    \ln\left(\frac{a^p}{p}+\frac{b^q}{q}\right)
    & \geq & \frac{1}{p}\ln(a^p)+\frac{1}{q}\ln(b^q) \\
    & = & \ln(a)+\ln(b) \\
    & = & \ln(a\cdot b)
  \end{IEEEeqnarray*}

  Etant donné que la fonction exponentielle est croissante,
  on a bien l'inégalité \ref{hold:pv1}.


Soient $x=(x_k)_{1\leq k\leq n}, y=(y_k)_{1\leq k\leq n}\in\mathbb{K}^n$.
Supposons les tous les deux non nuls, puisque dans ce cas, l'inégalité
du théorème est claire.

On pose, pour $j\in\{1, \ldots, n\}$:
$$A_j = \frac{|x_j|}{\left(\sum_{k=1}^n|x_k|^p\right)^{\frac{1}{p}}}
\mbox{ et }
B_j = \frac{|y_j|}{\left(\sum_{k=1}^n|y_k|^q\right)^{\frac{1}{q}}}$$

Ces quotients sont bien définis car $x$ et $y$ sont non nuls.
Par l'inégalité (\ref{hold:pv1}), on a que
\begin{IEEEeqnarray*}{rCl}
  \left(\sum_{j=1}^n
    \frac{|x_j|}{\left(\sum_{k=1}^n|x_k|^p\right)^{\frac{1}{p}}}
  \cdot
    \frac{|y_j|}{\left(\sum_{k=1}^n|y_k|^q\right)^{\frac{1}{q}}}\right)
  & = &\sum_{j=1}^nA_j\cdot B_j
  \\&\leq& \frac{1}{p}\sum_{j=1}^nA_j^p +\frac{1}{q}\sum_{j=1}^nB_j^q \\
  &=& \frac{1}{p} + \frac{1}{q} = 1
\end{IEEEeqnarray*}

Ce qui implique le résultat.
\end{proof}

Introduisons maintenant une autre inégalité célèbre qui nous
permettra de montrer que $\ell^p$ est un espace vectoriel.

\begin{thm}[Inégalité de Minkowski (dimension finie)]
  Soient $n\geq 1$  un nombre naturel,
  $p\in\left[1, +\infty\right[$ un nombre réel,
  $x=(x_k)_{1\leq k\leq n}, y=(y_k)_{1\leq k\leq n}\in \mathbb{K}^n$.

  L'inégalité suivante est vérifiée:
  $$\|x+y\|_p\leq \|x\|_p + \|y\|_p$$
\end{thm}
\begin{proof}

  Supposons $x, y$ non nuls, sinon le résultat est
  clair.

  Si $p=1$, il suffit d'itérer l'inégalité triangulaire
  pour le module afin de prouver l'assertion. Supposons
  par conséquent que $p>1$.

  \textbf{Remarque}: Cette preuve est fort calculatoire.
  Elle n'a pas vraiment d'intérêt au niveau des idées.

  \begin{IEEEeqnarray*}{rCl}
    \|x+y\|_p^p & = &\sum_{k=1}^n|x_k+y_k|^p \\
    & \leq & \sum_{k=1}^n |x_k+y_k|^{p-1}(|x_k|+|y_k|) \\
    & = & \sum_{k=1}^n |x_k\|x_k+y_k|^{p-1} +
    \sum_{k=1}^n |y_k\|x_k+y_k|^{p-1}
  \end{IEEEeqnarray*}

  Soit $q$ le réel conjugué de $p$. En appliquant l'inégalité
  de Hölder au dernier terme de l'inégalité précédente, on
  obtient:

  \begin{IEEEeqnarray*}{rCl}
    \|x+y\|_p^p & \leq &\left(\sum_{k=1}^n|x_k|^p\right)^\frac{1}{p}
    \cdot \left(\sum_{k=1}^n|x_k+y_k|^{(p-1)q}\right)^\frac{1}{q} +
    \left(\sum_{k=1}^n|y_k|^p\right)^\frac{1}{p}
    \cdot \left(\sum_{k=1}^n|x_k+y_k|^{(p-1)q}\right)^\frac{1}{q} \\
    & = & \left(\sum_{k=1}^n|x_k+y_k|^{p}\right)^\frac{1}{q}
    (\|x\|_p+\|y\|_p)\\
    & = & \left(\|x + y\|_p^{p}\right)^\frac{1}{q}(\|x\|_p+\|y\|_p)
  \end{IEEEeqnarray*}

  Ceci implique l'inégalité nous permettant de conclure:
  $$ \|x\|_p+\|y\|_p\geq \left(\|x + y\|_p^{p}\right)^{1-\frac{1}{q}}
  = \|x + y\|_p$$
\end{proof}

Le résultat se généralise facilement aux espaces $\ell^p$, et
nous avons donc en même temps l'argument le plus important
de la preuve que $\ell^p$ est un espace vectoriel, mais également
l'inégalité triangulaire pour la norme $\|.\|_p$.

\begin{thm}[Inégalité de Minkowski (Généralisation à $\ell^p$)]
  Soient $p\in\left[1, +\infty\right[$ un nombre réel,
  $x=(x_n)_{n\in\mathbb{N}}, y=(y_n)_{n\in\mathbb{N}}\in \ell^p$.

  On a:
  $\|x+y\|_p\leq \|x\|_p + \|y\|_p$
\end{thm}
\begin{proof}
  Etant donné qu'on a le résultat pour toutes les suites de
  sommes partielles, il suffit de passer à la limite
  pour conclure.
\end{proof}

\begin{exo} Rédiger en détail la preuve des affirmations suivantes:
  \begin{itemize}
  \item L'ensemble $\ell^p$ est un espace vectoriel sur $\mathbb{K}$.
  \item L'application $\|.\|_p$ définit bien une norme sur $\ell^p$.
  \end{itemize}
\end{exo}


Il est légitime de se demander pourquoi parler de ces espaces
dans un chapitre réservé aux espaces de Banach. C'est parce
qu'il s'agit d'exemples classiques:

\begin{prop}\label{lp:cpl}
  Soit $p> 1$ un nombre réel. L'espace vectoriel normé $(\ell^p, \|.\|_p)$
  est un espace de Banach.
\end{prop}

\begin{exo}
  Effectuer la preuve de la proposition \ref{lp:cpl}
\end{exo}

Nous n'effectuerons pas la preuve dans cette section, elle
est laissée à titre d'exercice. Nous allons toutefois
montrer le cas $p=1$. Les idées mises
en oeuvre dans cette preuve sont intéressantes et il est
conseillé de les revoir.

\begin{prop}
  L'espace vectoriel normé $(\ell^1, \|.\|_1)$ est un espace de Banach.
\end{prop}

\begin{proof}
  Soit $\left(x^{(n)}\right)_{n\in\mathbb{N}}=
  \left(\left(x_k^{(n)}\right)_k\right)_n$\footnote{
    On note en indice la ``composante'' dans $\ell^1$ et en
    exposant l'indice de la suite} une suite de
  Cauchy dans $(\ell^1, \|.\|_1)$. Montrons qu'elle est convergente
  au sens de la norme $\|.\|_1$.

  En traduisant l'hypothèse sur la suite, il est aisé de déduire que
  pour tout $k\in\mathbb{N}$, la suite $(x_k^{(n)})_n$ est de Cauchy dans
  $\mathbb{K}$ qui est complet. Soit $x_k\in\mathbb{K}$ la limite
  de cette suite dans $\mathbb{K}$ .

  Nous avons ainsi obtenu un candidat limite $x=(x_k)_k$. Nous devons
  montrer qu'il s'agit bien d'un élément de $\ell^1$ et qu'il s'agit de
  la limite de la suite $(x^{(n)})_n$.

  Soit $\varepsilon>0$. Puisque la suite $(x^{(n)})$ est de
  Cauchy, il est vrai que:
%  $$ \exists  N \forall  p, q\geq N, \|x_p-x_q\|_1<\varepsilon$$
  $$ \exists  N\geq 0, \forall  p, q\geq N, \forall  n\geq 0,
  \sum_{k=0}^n| x^{(p)}_k - x^{(q)}_k|\leq\varepsilon$$

  En faisant tendre $q\to+\infty$, on en déduit que:
  $$ \exists  N\geq 0, \forall  p\geq N, \forall  n\geq 0,
  \sum_{k=0}^n| x^{(p)}_k - x_k|\leq\varepsilon$$
  Ce qui permet facilement de conclure que
  $x^{(n)}\xrightarrow[n\to+\infty]{\|.\|_1}x$

  Pour montrer que $x\in\ell^1$, il suffit d'observer l'inégalité
  suivante, valable pour tous naturels $n, N$ considérés.
  $$\sum_{k=0}^N|x_k|
  \leq \sum_{k=0}^N|x_k-x_k^ {(n)}|+ \sum_{k=0}^N|x_k^{(n)}|
  \leq \|x-x^{(n)}\|_1+\|x^{(n)}\|_1$$
  Au vu de la convergence que nous venons d'établir, le premier terme
  du membre de droite converge vers $0$. Le second quant à lui est
  borné par une constante indépendante de $n$ car il s'agit d'un
  élément d'une suite de Cauchy. Ceci implique que la série
  définissant $\|x\|_1$ est bien convergente.

\end{proof}

Il reste encore un espace lié aux espaces $\ell^p$ que
nous n'avons pas introduit: l'espace des suites $\ell^\infty$.

\begin{df}
  L'espace des suites bornées, noté $\ell^\infty$
  correspond à l'ensemble suivant:
  $$\ell^\infty=\{x\in\mathbb{K}^{\mathbb{N}}\mid
  \sup_{n\in\mathbb{N}}|x_n|<\infty\}$$
\end{df}

\begin{rem}
  Il est utile de remarquer que $1$ et $\infty$ sont
  conjugués (par exemple pour les espaces duaux, cf.
  exemples \ref{dual:ex:r2n1} et \ref{dual:ex:r2ninfty}
  de la section \ref{dual:ex:dimf}).
\end{rem}

\begin{exo}
  Montrer que l'application suivante définit bien une
  norme sur $\ell^\infty$:
  $$\|.\|_\infty:\ell^\infty\to\left[0,+\infty\right[:
  (x_n)_{n\in\mathbb{N}}\mapsto\sup_{n\in\mathbb{N}}|x_n|$$
\end{exo}

Tout comme ses confrères, il s'agit d'un espace de
Banach. Cela fait l'objet du résultat suivant:
\begin{prop}
  L'espace vectoriel $(\ell^\infty, \|.\|_\infty)$ est
  un espace de Banach.
\end{prop}
La preuve est fortement similaire à celle pour l'espace
de suites $\ell^1$. Elle sera tout de même explicitée
afin de permettre la lecture d'une version adaptée de
la preuve ci-dessus.
\begin{proof}
  Soit $\left(x^{(n)}\right)_{n\in\mathbb{N}}=
  \left(\left(x_k^{(n)}\right)_k\right)_n$ une suite de
  Cauchy dans $(\ell^\infty, \|.\|_\infty)$. Montrons qu'elle est convergente
  au sens de la norme $\|.\|_\infty$.

  En traduisant l'hypothèse sur la suite, il est aisé de déduire que
  pour tout $k\in\mathbb{N}$, la suite $(x_k^{(n)})_n$ est de Cauchy dans
  $\mathbb{K}$ qui est complet. Soit $x_k\in\mathbb{K}$ la limite
  de cette suite dans $\mathbb{K}$ .

  Nous avons ainsi obtenu un candidat limite $x=(x_k)_k$. Nous devons
  montrer qu'il s'agit bien d'un élément de $\ell^\infty$ et qu'il s'agit de
  la limite de la suite $(x^{(n)})_n$.

  Soit $\varepsilon>0$. Puisque la suite $(x^{(n)})$ est de
  Cauchy, il est vrai que:
%  $$ \exists  N \forall  p, q\geq N, \|x_p-x_q\|_1<\varepsilon$$
  $$ \exists  N\geq 0, \forall  p, q\geq N,
   \forall {k\geq0},| x^{(p)}_k - x^{(q)}_k|\leq\varepsilon$$

  En faisant tendre $q\to+\infty$, on en déduit que:
  $$ \exists N\geq 0, \forall  p\geq N,
   \forall {k\geq0} , | x^{(p)}_k - x_k| \leq \varepsilon$$
  Ce qui permet facilement de conclure que
  $x^{(n)}\xrightarrow[n\to+\infty]{\|.\|_\infty}x$

  Pour montrer que $x\in\ell^\infty$, il suffit d'observer l'inégalité
  suivante, valable pour tous naturels $n, k$ considérés.
  $$|x_k|
  \leq |x_k-x_k^ {(n)}|+ |x_k^{(n)}|
  \leq \|x-x^{(n)}\|_\infty+\|x^{(n)}\|_\infty$$
  Au vu de la convergence que nous venons d'établir, le premier terme
  du membre de droite converge vers $0$. Le second quant à lui est
  borné par une constante indépendante de $n$ car il s'agit d'un
  élément d'une suite de Cauchy. Ceci implique que la suite des
  $(x_k)_{k\in\mathbb{N}}$ est bornée et donc que $x\in\ell^\infty$.
\end{proof}

\section{Espaces de fonctions continues}
Un autre exemple d'espace de Banach dont nous avons
les outils pour montrer la complétude est l'espace
des fonctions continues sur l'intervalle $[0, 1]$.
Tout ce qui suit dans cette section peut être
généralisé aux fonctions définie sur un intervalle
fermé borné quelconque.

\begin{df}
  On note $$\mathscr{C}[0, 1]=\left\{
    f:[0, 1]\to\mathbb{K}\mbox{ continue}\right\}$$
  l'espace vectoriel de fonctions continues sur
  l'intervalle $[0, 1].$
  On le munit de la norme $\|.\|_\infty$ définie par:
  $$\|.\|_\infty:\mathscr{C}[0, 1]\to \left[0, +\infty\right[:
  f\mapsto \|f\|_\infty=\max_{x\in [0, 1]}|f(x)|$$
\end{df}

Le maximum apparaissant dans la norme est bien défini,
car les fonctions considérées sont continues sur un
compact et atteignent donc leurs bornes.

\begin{prop}
  L'espace des fonctions continues sur l'intervalle [0, 1]
  est complet au sens de la norme $\|.\|_\infty$.
\end{prop}

\begin{proof}
  Soit $(f_n)_{n\in\mathbb{N}}$ une suite de Cauchy dans
  $\mathscr{C}[0, 1]$. En retraduisant cette hypothèse
  en remplaçant le maximum définissant $\|.\|_\infty$
  par un quantificateur universel, on obtient la phrase
  quantifiée suivante:
  \begin{equation}\label{cpl:fn}
    \forall \varepsilon>0, \exists N\in\mathbb{N},
  \forall p, q\geq N , \forall x\in [0, 1],
  |f_p(x)-f_q(x)|\leq\varepsilon
  \end{equation}

  On remarque que pour tout $x\in[0, 1]$, la suite
  $(f_n(x))_{n\in\mathbb{N}}$ est de Cauchy dans
  $\mathbb{K}$ qui est complet. Il existe donc
  $f$ limite ponctuelle de $(f_n)_{n\in\mathbb{N}}$.

  Pour conclure, il suffit de montrer que
  la suite $(f_n)_{n\in\mathbb{N}}$ converge
  au sens de $\|.\|_\infty$ vers $f$. Cela
  garantira la continuité de $f$, car $f$
  sera la limite uniforme de la suite.

  En faisant $q\to\infty$ dans la phrase quantifiée
  \ref{cpl:fn}, on a le résultat.

\end{proof}

\section{Espaces d'applications linéaires}

Soient $(E, \|.\|_E)$ et $(F, \|.\|_F)$ deux espaces
vectoriels normés sur $\mathbb{K}$. Rappelons certaines
propriétés et définitions vues l'année passée.
\begin{df}
  On note $\mathscr{L}(E, F)$ l'espace vectoriel des
  applications linéaires continues de $E$
  dans $F$. Il s'agit d'un espace vectoriel normé,
  muni de la norme opérateur définie par:

  $$\|.\|:\mathscr{L}(E, F)\to \left[0, +\infty\right[:
  T\mapsto \|T\| = \sup_{\substack{x\in E\\\|x\|_\leq 1}}\|T(x)\|_F$$
\end{df}

On a également vu plusieurs équivalences
pour montrer qu'une application linéaire
est continue. Rappelons les.
\begin{prop}\label{cont:lin}
  Soit $T: E\to F$ linéaire. Les assertions suivantes sont équivalentes:
  \begin{itemize}
  \item $T$ est continue.
  \item $T$ est continue en $0$.
  \item $T$ est bornée sur la boule unité de $E$.
  \item Il existe une constante $C>0$ telle que
    pour tout $x\in E$, $\|T(x)\|_F\leq C\|x\|$
    \footnote{La constante optimale, s'il en
      existe une, est $\|T\|$.}.
  \end{itemize}
\end{prop}



Remarquez également qu'il existe plusieurs
formulations équivalentes pour la norme
opérateur:
\begin{prop}\label{cont:norm}
  Soit $T\in\mathscr{L}(E, F)$. Les égalités suivantes
  sont satisfaites:
  \begin{IEEEeqnarray*}{rCl}
    \|T\| & = & \sup_{\substack{x\in E\\\|x\|_E = 1}}\|T(x)\|_F \\
    & = & \sup_{\substack{x\in E\\\|x\|_E < 1}}\|T(x)\|_F
  \end{IEEEeqnarray*}

\end{prop}

Il est utile, si vous avez des difficultés avec ces objets,
d'essayer de les manipuler en résolvant les exercices:
\begin{exo}
  Prouver les propositions \ref{cont:lin} et \ref{cont:norm}.
\end{exo}
\begin{exo}[\og Sous-multiplicativité de la norme\fg]
  Soit $(G, \|.\|_G)$ un espace vectoriel normé sur $\mathbb{K}$.
  Soient $f: E\to F$, $g:F\to G$ des applications linéaires
  continues. Montrer l'inégalité
  $$\|g\circ f\|\leq \|g\|\cdot \|f\|$$
\end{exo}

L'espace $\mathscr{L}(E, F)$ n'est \emph{pas} un espace de Banach
en général. Il faut ajouter une hypothèse sur $F$ pour avoir
la complétude de cet espace.

\begin{prop}\label{lin:cpl:imp}
  Si $F$ est complet, alors $\mathscr{L}(E, F)$ est un
  espace de Banach.
\end{prop}
\textbf{Remarque}: il n'y a pas de condition supplémentaire
à imposer à l'espace $E$. En particulier, le résultat n'exige
pas sa complétude.

La réciproque de ce résultat est également vraie, nous
l'énoncerons dans un futur chapitre.

\begin{proof}
  Soit $(T_n)_{n\in\mathbb{N}}$ une suite de Cauchy dans
  $\mathscr{L}(E, F)$, c'est-à-dire:
  $$\forall \varepsilon>0, \exists N\geq 0, \forall p, q\geq N,
  \|T_p-T_q\|\leq \varepsilon$$

  Alors, on a l'assertion suivante:
  \begin{IEEEeqnarray*}{rl}
    \IEEEeqnarraymulticol{2}{l}{
    \forall \varepsilon>0, \exists N\geq 0, \forall p, q\geq N,
    \forall x\in E,}\\ \qquad & \|T_p(x)-T_q(x)\|\leq
    \|T_p-T_q\|\cdot \|x\|_E\leq \varepsilon\cdot\|x\|_E
  \end{IEEEeqnarray*}

  Cela implique que pour tout $x\in E$, la suite $(T_n(x))_{n\in\mathbb{N}}$
  est de Cauchy dans $F$ qui par hypothèse est complet. Soit $T(x)$
  limite de cette suite dans $F$.

  Montrons que $T$ est la limite, au sens de la norme opérateur, de la
  suite $(T_n)_{n\in\mathbb{N}}$.

  En reprenant la phrase quantifiée ci-dessus, en considérant $x\in E$
  de norme inférieure à $1$ et en faisant tendre $q$ vers l'infini,
  on est assuré de la convergence.

  L'application $T$ est bien linéaire car elle est limite simple
  d'applications
  linéaires. Elle est bornée sur la boule unité car pour tout naturel
  $n$ on a l'inégalité:
  $$\|T\|\leq \|T-T_n\| + \|T_n\|$$
\end{proof}

\section{Sous-espaces vectoriels et complétude}
Lorsqu'on considère un espace de Banach et un sous-espace vectoriel de ce
dernier, on peut se poser la question suivante: \og Ce sous-espace est-il
complet pour la norme induite?\fg.
Le résultat suivant permet de donner un critère pour vérifier cela.

\begin{thm}
  Soit $(E, \|.\|)$ un espace de Banach et $F$ un sous-espace vectoriel de $E$.
  Alors $(F, \|.\|)$ est un espace de Banach si et seulement si $F$ est fermé
  dans $E$.
\end{thm}
\begin{proof}
  Supposons que $F$ est complet et montrons qu'il s'agit d'un fermé au
  sens de $E$. Soient $(x_n)_{n\in\mathbb{N}}$ une suite d'éléments de $F$
  et $x\in E$ tels que $x_n\to x$. Puisque la suite $(x_n)_{n\in\mathbb{N}}$
  est convergente, elle est de Cauchy. Par complétude de $F$, elle admet
  une limite $y\in F$. Par unicité de la limite, $x = y\in F$.

  Réciproquement supposons que $F$ est fermé. Soit $(x_n)_{n\in\mathbb{N}}$
  une suite de Cauchy dans $F$. Alors c'est une suite de Cauchy d'éléments
  de $E$; elle admet une limite $x$ dans $E$. Puisque $F$ est fermé, la
  suite considérée converge bien dans $F$.
\end{proof}

Le résultat est également vrai dans le cadre des espaces métriques (la même
preuve fonctionne).

Fournissons une application de ce résultat à un exemple. On introduit
un sous-espace d'un des exemples présentés ci-avant.

\begin{df}
  L'espace $c_0$ est l'espace des suites dans $\mathbb{K}$
  convergeant vers $0$. On le munit de la norme $\|.\|_\infty$.
\end{df}

Etant donné que toutes les suites convergentes sont bornées,
$c_0$ est un sous-espace vectoriel de $\ell^\infty$.

\begin{prop}
  $(c_0, \|.\|_\infty)$ est un espace de Banach.
\end{prop}

\begin{proof}
  Au vu du résultat ci-avant, il suffit de montrer que $c_0$ est
  fermé dans $\ell^\infty$.

  Soit $(x^{(n)})_{n\in\mathbb N}$ une suite d'éléments de $c_0$
  convergeant vers $x$ un élément de $\ell^\infty$. Montrons que
  $x\in c_0$.

  Soit $\varepsilon > 0$. Il existe $N$ tel que pour tout naturel $n\geq N$,
  $\|x^{(n)}-x\|_\infty\leq \varepsilon/2$. \'{E}tant donné que $x^{(N)}$
  est un élément de $c_0$, il existe $k_0$ tel que pour tout $k\geq k_0$,
  $|x^{(N)}_k|\leq \varepsilon /2$. Pour tout $k\geq k_0$, on a
  $$|x_k|\leq |x_k - x^{(N)}_k| + |x^{(N)}_k|
  \leq \|x - x^{(N)}\|_\infty + |x^{(N)}_k| \leq \varepsilon$$
  ce qui montre bien que $x$ est élément de $c_0$.
\end{proof}



%%% Local Variables:
%%% mode: latex
%%% TeX-master: "../analyse3"
%%% End:
