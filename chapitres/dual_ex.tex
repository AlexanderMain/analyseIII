\section{Généralités}
Soit $\mathbb{K}$ un corps, avec $\mathbb{K}= \mathbb{R}$
ou $\mathbb{K}=\mathbb{C}$.
Soit $(E, \|.\|)$ un espace vectoriel sur $\mathbb{K}$.
\begin{df}
  Le dual de $E$, noté $E^*$ correspond à $\mathscr{L}(E, \mathbb{K})$ ie.
  l'ensemble des formes linéaires continues.
\end{df}

Par la proposition \ref{lin:cpl:imp}, étant donné que $\mathbb{K}$ est complet,
on a que $(E^*, \|.\|)$ est un espace de Banach (la norme considérée est
la norme opérateur).

Dans ce chapitre plutôt que de s'attarder sur des propriétés générales
des espaces duaux, nous étudierons des exemples et nous identifierons
les espaces duaux d'espaces connus à d'autres espaces que nous connaissons.

\textbf{Remarque}: les intuitions sur comment trouver à quel espace s'identifie
le dual seront omises.

\section{Exemples en dimension finie}\label{dual:ex:dimf}
En dimension finie, il est connu que le dual algébrique est de même dimension
que l'espace vectoriel considéré. De plus, comme toutes les applications
linéaires définies sur un espace vectoriel de dimension finie
sont continues, il suffit de vérifier qu'on a une isométrie entre
les espaces considérés pour avoir l'identification.

\begin{exo}
Soient $(X, d_X),(Y, d_Y)$ deux espaces métriques, $i:X\to Y$ une isométrie,
c'est-à-dire \og $i$ préserve les distances\fg, c'est-à-dire:
$$\forall x, y\in X, d_X(x, y) = d_Y(i(x), i(y))$$

Montrer que $i$ est injective.
\end{exo}

Dans les calculs relatifs à tous les exemples
suivants, il n'est jamais montré que l'application
considérée est linéaire. Cela est laissé comme exercice.

\begin{ex}
  On a $(\mathbb{R}^2, \|.\|_2)^*\equiv (\mathbb{R}^2, \|.\|_2)$
  via l'isométrie:
  \begin{IEEEeqnarray*}{rrClrCl}
    i: & (\mathbb{R}^2, \|.\|_2) & \to & \IEEEeqnarraymulticol{3}{l}{(\mathbb{R}^2, \|.\|_2)^*} \\
    & (x_1, x_2)&\mapsto& i(x_1, x_2):&(\mathbb{R}^2, \|.\|_2)&\to&\mathbb{R} \\
    &&&&(x, y)&\mapsto& x_1x + x_2 y
  \end{IEEEeqnarray*}
\end{ex}

Il suffit de prouver que $i$ est une isométrie étant donné que
l'espace considéré est de  dimension finie.

Soit $(x_1, x_2)\in\mathbb{R}^2$. On a (en utilisant l'inégalité
de Cauchy-Schwarz):
\begin{IEEEeqnarray*}{rCl}
  \|i(x_1, x_2)\| = \sup_{x^2+y^2\leq 1}|x_1 x + x_2 y|
  & \leq & \sup_{x^2+y^2\leq 1}|x_1x| + |x_2 y| \\
  & \leq & \sup_{x^2+y^2\leq 1} (x_1^2+x_2^2)^{1/2}(x^2+y^2)^{1/2} \\
  & \leq & (x_1^2+x_2^2)^{1/2}\sup_{x^2+y^2\leq 1}(x^2+y^2)^{1/2}
  = \|(x_1, x_2)\|_2
\end{IEEEeqnarray*}

Réciproquement, il suffit de considérer $z = \frac{(x_1, x_2)}{\|(x_1, x_2)\|_2}$;
on a $i(x_1, x_2)(z) = \|(x_1, x_2)\|_2$ ce qui fournit l'inégalité nous
permettant de conclure.

\begin{ex}\label{dual:ex:r2n1}
  On a $(\mathbb{R}^2, \|.\|_1)^*\equiv (\mathbb{R}^2, \|.\|_\infty)$
  via l'isométrie:
  \begin{IEEEeqnarray*}{rrClrCl}
    i: & (\mathbb{R}^2, \|.\|_\infty) & \to & \IEEEeqnarraymulticol{3}{l}{(\mathbb{R}^2, \|.\|_1)^*} \\
    & (x_1, x_2)&\mapsto& i(x_1, x_2):&(\mathbb{R}^2, \|.\|_1)&\to&\mathbb{R} \\
    &&&&(x, y)&\mapsto& x_1x + x_2 y
  \end{IEEEeqnarray*}
\end{ex}

Soit $(x_1, x_2)\in\mathbb{R}^2$. On a:
\begin{IEEEeqnarray*}{rCl}
  \|i(x_1, x_2)\| = \sup_{|x|+|y|\leq 1} |x_1 x + x_2 y|
  & \leq & \sup_{|x|+|y|\leq 1}|x_1 x| + |x_2 y| \\
  & \leq & \sup_{|x|+|y|\leq 1} \max(|x_1|, |x_2|) (|x|+|y|) \\
  & \leq & \|(x_1, x_2)\|_\infty \sup_{|x|+|y|\leq 1}(|x|+|y|)
  = \|(x_1, x_2)\|_\infty
\end{IEEEeqnarray*}

Si $\max(|x_1|, |x_2|) = |x_1|$, alors $i(x_1, x_2)$ atteint
$\|(x_1, x_2)\|_\infty$ en $(\mathrm{sign}(x_1), 0)$ qui est bien un élément de
la boule unité de $(\mathbb{R}^2, \|.\|_1)$. L'autre cas
est analogue. On peut donc conclure que $i$ est bien une isométrie.

\begin{ex}\label{dual:ex:r2ninfty}
  On a $(\mathbb{R}^2, \|.\|_\infty)^*\equiv (\mathbb{R}^2, \|.\|_1)$
  via l'isométrie:
  \begin{IEEEeqnarray*}{rrClrCl}
    i: & (\mathbb{R}^2, \|.\|_1) & \to & \IEEEeqnarraymulticol{3}{l}{(\mathbb{R}^2, \|.\|_\infty)^*}\\
    & (x_1, x_2)&\mapsto& i(x_1, x_2):&(\mathbb{R}^2, \|.\|_\infty)&\to&\mathbb{R} \\
    &&&&(x, y)&\mapsto& x_1x + x_2 y
  \end{IEEEeqnarray*}
\end{ex}

Soit $(x_1, x_2)\in\mathbb{R}^2$. On a:
\begin{IEEEeqnarray*}{rCl}
  \|i(x_1, x_2)\| = \sup_{\max(|x|,|y|)\leq 1} |x_1 x + x_2 y|
  & \leq & \sup_{\max(|x|,|y|)\leq 1}|x_1 x| + |x_2 y| \\
  & \leq & \sup_{\max(|x|,|y|)\leq 1} (|x_1|+|x_2|)\max(|x|, |y|)  \\
  & \leq & \|(x_1, x_2)\|_1 \sup_{\max(|x|,|y|)\leq 1}\max(|x|,|y|)
  = \|(x_1, x_2)\|_1
\end{IEEEeqnarray*}

Il est facile de vérifier que $i(x_1, x_2)$ atteint $\|(x_1, x_2)\|_1$
au point $(\mathrm{sign}(x_1), \mathrm{sign}(x_2))$ de la boule unité
de $(\mathbb{R}^2, \|.\|_\infty)$, ce qui fournit l'autre inégalité
désirée.

\begin{exo}
  Soient $p > 1$, $q>1$ le conjugué de $p$. Montrer qu'on a
  que $(\mathbb{R}^2, \|.\|_q)^*\equiv (\mathbb{R}^2, \|.\|_p)$.
\end{exo}

Notez que tous ces exemples peuvent se généraliser
en dimension supérieure à 2.

\section{Exemples en dimension infinie}
Contrairement aux exemples de la section précédente, il ne
suffit plus de vérifier que l'application considérée est
une isométrie car la surjectivité n'est plus garantie par
l'injectivité. Les exemples requièrent donc plus de travail.

Le premier exemple considéré est celui de l'espace $c_0$
introduit au chapitre précédent.

\begin{ex}
  On a $(c_0, \|.\|_\infty)^*\equiv (\ell^1, \|.\|_1)$
  via l'isométrie:
  \begin{IEEEeqnarray*}{rrClrCl}
    i: & (\ell^1, \|.\|_1) & \to & \IEEEeqnarraymulticol{3}{l}{(c_0, \|.\|_\infty)^*}\\
    & x=(x_n)_{n\in\mathbb{N}}&\mapsto& i(x): & (c_0, \|.\|_\infty)&\to&\mathbb{R} \\
    &&&&(y_n)_{n\in\mathbb{N}}&\mapsto& \sum_{k=0}^\infty x_ky_k
  \end{IEEEeqnarray*}
\end{ex}

\textbf{L'application $i$ est une isométrie}:

Soit $x=(x_n)_{n\in\mathbb{N}}\in\ell^1$. La fonction $i(x)$
est bien définie car pour tout $y\in c_0$, on a l'inégalité
$$|i(x)(y)|\leq \sum_{k=0}^\infty |x_ky_k|\leq \|y\|_\infty \|x\|_1$$

Cela implique en particulier que $\|i(x)\|\leq\|x\|_1$. Montrons l'inégalité
réciproque.

Soit, pour tout naturel $N$, la suite $(y^{(N)}_n)_{n\in\mathbb{N}}$
définie par:
$$\forall n\in\mathbb{N},  y_n^{(N)}=\begin{cases} \mathrm{sign}(x_n)\mbox{ si $n\leq N$}
  \\ 0 \mbox{ sinon}\end{cases}$$

Alors $(y^{(N)}_n)_{n\in\mathbb{N}}$ appartient à la boule unité
de $c_0$ et on a $|i(x)(y^{(N)})|\to\|x\|_1$ quand $N$ tend vers
l'infini. Ceci implique que $\|i(x)\|\geq \|x\|_1$, ce qu'on voulait. \newline

\textbf{L'application $i$ est surjective}:

Vérifions la surjectivité de $i$.
Soit $x^*\in (c_0)^*$. On pose $x = (x^*(e_n))_{n\in\mathbb{N}}$ où $e_n$
est la suite de terme général $(\ind_{\{n\}}(k))_{k\in\mathbb{N}}$.

Il reste à montrer que $i(x) = x^*$, ce qui se vérifie par
un simple calcul (il suffit d'observer les sommes partielles
puis de passer à la limite), et que $x$ est bien élément de $\ell^1$.

Pour tout naturel $N$, on a:
$$\sum_{k=0}^N |x_k| = x^*(y^{(N)})\leq \|x^*\|$$
Ce qui implique que la série des termes de $x$ est
absolument convergente, ce qu'on voulait montrer.

\begin{ex}
  On a $(\ell^1, \|.\|_1)^*\equiv (\ell^\infty, \|.\|_\infty)$
  via l'isométrie:
  \begin{IEEEeqnarray*}{rrClrCl}
    i: & (\ell^\infty, \|.\|_\infty) & \to & \IEEEeqnarraymulticol{3}{l}{(\ell^1, \|.\|_1)^*}\\
    & x=(x_n)_{n\in\mathbb{N}}&\mapsto& i(x): & (\ell^1, \|.\|_1)&\to&\mathbb{R} \\
    &&&&(y_n)_{n\in\mathbb{N}}&\mapsto& \sum_{k=0}^\infty x_ky_k
  \end{IEEEeqnarray*}
\end{ex}

\textbf{L'application $i$ est une isométrie}:

Soit $x=(x_n)_{n\in\mathbb{N}}\in\ell^\infty$. La fonction $i(x)$
est bien définie car pour tout $y\in \ell^1$, on a l'inégalité
$$|i(x)(y)|\leq \sum_{k=0}^\infty |x_ky_k|\leq \|x\|_\infty \|y\|_1$$

Cela implique en particulier que $\|i(x)\|\leq\|x\|_\infty$. Montrons l'inégalité
réciproque.

Soit, pour tout naturel $N$, la suite $(y^{(N)}_n)_{n\in\mathbb{N}}$
définie par:
$$\forall n\in\mathbb{N} , y_n^{(N)}= \ind_{\{N\}}(n)\mathrm{sign}(x_n)$$

Alors $(y^{(N)}_n)_{n\in\mathbb{N}}$ est élément de la boule unité
de $\ell^1$ et on a $|i(x)(y^{(N)})| = |x_N|$ quel que soit
le naturel $N$ considéré.
Ceci implique que $\|i(x)\|\geq \|x\|_\infty$, ce qu'on voulait. \newline

\textbf{L'application $i$ est surjective}:

Vérifions la surjectivité de $i$.
Soit $x^*\in (\ell^1)^*$. On pose $x = (x^*(e_n))_{n\in\mathbb{N}}$ où $e_n$
est la suite de terme général $(\ind_{\{n\}}(k))_{k\in\mathbb{N}}$.

Il reste à montrer que $i(x) = x^*$, ce qui se vérifie par
un simple calcul (il suffit d'observer les sommes partielles
puis de passer à la limite), et que $x$ est bien élément de $\ell^\infty$.

Pour tout naturel $N$, on a:
$$|x_k| = x^*(y^{(N)})\leq \|x^*\|$$
Ce qui implique que $x$ est
une suite bornée, ce que l'on voulait montrer.


%%% Local Variables:
%%% mode: latex
%%% TeX-master: "../analyse3"
%%% End:
