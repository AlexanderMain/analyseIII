\chapter{Solutions des exercices}
Tous les exercices ne seront pas corrigés.

\section{Hahn-Banach (Formes analytiques)}
\textbf{Solution de l'exercice \ref{hb:a:exo1}}

\begin{enumerate}
\item Supposons que $x_0\in \mathrm{adh(F)}$.
  Etant donné que le noyau de toute forme linéaire
  continue est fermé (le noyau est l'image réciproque du singleton
  0 qui est fermé, donc est fermé par continuité), et que le passage
  à l'adhérence conserve les inclusions, il est aisé de conclure.

  Réciproquement, supposons que $x_0\notin \mathrm{adh(F)}$.
  Par le corollaire \ref{hb:a:cor4} des formes analytiques de
  Hahn-Banach, il existe une forme s'annulant sur $F$ et pas
  en $x_0$, ce que l'on voulait montrer.
\item Supposons $F$ dense dans $E$. Soit $x^*\in E^*$ tel
  que $F\subseteq\mathrm{Ker}(x^*)$. Etant donné que ce
  noyau est fermé, il contient la fermeture de $F$ qui
  est $E$. Ceci implique que $x^*$ est l'application
  constante nulle.

  Réciproquement, supposons $F$ non dense dans $E$.
  Il suffit de considérer un élément du complémentaire
  de l'adhérence de $F$ et d'appliquer le résultat \ref{hb:a:cor4}
  pour obtenir une forme linéaire dont le noyau contient
  $F$ mais non identiquement nulle.
\end{enumerate}

\section{Hahn-Banach (Formes géométriques)}
\textbf{Solution de l'exercice \ref{hb:g:j2}}

On vérifie facilement que si $x\in \alpha C$, alors pour tout
$\beta >\alpha$, $x\in \beta C$;
avoir que $\alpha^{-1}x\in C$ implique que le segment joignant $0$ et ce
point est contenu dans $C$, et $\beta^{-1}x$ est dans le segment car
$\beta^{-1} <\alpha^{-1}$.

Par définition de la jauge, $j_C(x)= \inf\{\alpha > 0 \mid x\in \alpha C\}$.
Soit $\varepsilon >0$, alors il existe $\alpha > 0$ tel que $j_C(x)\leq
\alpha \leq j_C(x) + \varepsilon$ et $x\in\alpha C$. Par ce qui précède,
on peut conclure.

%%% Local Variables:
%%% mode: latex
%%% TeX-master: "analyse3"
%%% End:
