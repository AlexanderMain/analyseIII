\chapter{Solutions des exercices}
Tous les exercices ne seront pas corrigés.

\section{Hahn-Banach (Formes analytiques)}
\subsubsection*{Solution de l'exercice \ref{hb:a:exo1}}

\begin{enumerate}
\item Supposons que $x_0\in \mathrm{adh(F)}$.
  Etant donné que le noyau de toute forme linéaire
  continue est fermé (le noyau est l'image réciproque du singleton
  0 qui est fermé, donc est fermé par continuité), et que le passage
  à l'adhérence conserve les inclusions, il est aisé de conclure.

  Réciproquement, supposons que $x_0\notin \mathrm{adh(F)}$.
  Par le corollaire \ref{hb:a:cor4} des formes analytiques de
  Hahn-Banach, il existe une forme s'annulant sur $F$ et pas
  en $x_0$, ce que l'on voulait montrer.
\item Supposons $F$ dense dans $E$. Soit $x^*\in E^*$ tel
  que $F\subseteq\mathrm{Ker}(x^*)$. Etant donné que ce
  noyau est fermé, il contient la fermeture de $F$ qui
  est $E$. Ceci implique que $x^*$ est l'application
  constante nulle.

  Réciproquement, supposons $F$ non dense dans $E$.
  Il suffit de considérer un élément du complémentaire
  de l'adhérence de $F$ et d'appliquer le résultat \ref{hb:a:cor4}
  pour obtenir une forme linéaire dont le noyau contient
  $F$ mais non identiquement nulle.
\end{enumerate}

\section{Hahn-Banach (Formes géométriques)}
\subsubsection*{Solution de l'exercice \ref{hb:g:j2}}
On vérifie facilement que si $x\in \alpha C$, alors pour tout
$\beta >\alpha$, $x\in \beta C$;
avoir que $\alpha^{-1}x\in C$ implique que le segment joignant $0$ et ce
point est contenu dans $C$, et $\beta^{-1}x$ est dans le segment car
$\beta^{-1} <\alpha^{-1}$.

Par définition de la jauge, $j_C(x)= \inf\{\alpha > 0 \mid x\in \alpha C\}$.
Soit $\varepsilon >0$, alors il existe $\alpha > 0$ tel que $j_C(x)\leq
\alpha \leq j_C(x) + \varepsilon$ et $x\in\alpha C$. Par ce qui précède,
on peut conclure.

\section{Théorème du graphe fermé}
\subsubsection*{Solution de l'exercice \ref{grp:ferme:exo:pos}}
Par le théorème du graphe fermé, il suffit de montrer que le graphe
de $T$ est fermé. Soit $(x_n, Tx_n)$ une suite du graphe de $T$ convergeant
vers $(x, x^*)\in E\times E^*$ et montrons que $x^* = Tx$.

Posons $u_n = x_n - x$, il suffit de montrer (par linéarité de $T$) que
$Tu_n \to 0$. Notons $f = \lim_{n\to+\infty}Tu_n$ (elle existe car on suppose
la convergence préalable de la suite $(Tx_n)_n$). Quel que soit $y\in E$, on a
$$\langle T(y - u_n), y - u_n\rangle \geq 0.$$
En passant à la limite quand $n\to+\infty$, on en déduit que pour tout $y\in E$,
$$\langle Ty - f, y\rangle \geq 0 \iff
\langle Ty, y\rangle \geq f(y).$$
Soit $y\in E$. Pour tout réel $R$, cela implique que
(par linéarité de $T$, de $Ty$ et de $f$)
$$R^2\langle Ty, y\rangle \geq R f(y).$$
En prenant $R\to 0$, avec $R> 0$, on déduit que $f(y)\leq 0$, et en prenant
$R \to$ avec $R < 0$, on a $f(y) \geq 0$, d'où $f(y) = 0$. Ceci montre que
$f$ est identiquement nulle, ce qu'on voulait montrer.

\section{Topologie faible}
\subsubsection*{Solution de l'exercice \ref{faib:ex:prod}}
\begin{lem}
  Soient $(E, \|.\|_E)$ et $(F, \|.\|_F)$ deux espaces vectoriels normés.
  Notons $\pi_E$ (resp. $\pi_F$) la projection sur $E$ (resp. $F$).
  Pour tout élément $z^*$ de $(E\times F)^*$, il existe $x^*\in E^*$,
  $y^*\in F^*$ tels que $z^* = (x^*\circ \pi_E) + (y^*\circ\pi_F)$.
\end{lem}
\begin{proof}
  Soient les applications linéaires
  $\sigma_E:E\to E\times F: x\mapsto (x, 0)$ et
  $\sigma_F:F\to E\times F: y\mapsto (0, y)$. Elles sont clairement
  continues (sur le produit on considère la norme du maximum).
  Posons $x^* = z^*\circ \sigma_E$ et $y^* = z^*\circ \sigma_F$. Puisque
  $\pi_E\circ \sigma_E =\mathrm{Id}_E$ et  $\pi_F\circ \sigma_F =\mathrm{Id}_F$,
  on a bien l'égalité ci-dessus.
\end{proof}
Notons $\mathcal{T}_\pi$ la topologie sur le produit
$(E, \omega_E)\times(F, \omega_F)$ et $\mathcal{T}_\omega$ la topologie
faible sur $(E, \|.\|_E)\times(F, \|.\|_F)$.

En utilisant ce lemme, on peut résoudre l'exercice. Il suffit de montrer
$\mathcal{T}_\pi$ rend les formes linéaires de $(E\times F)^*$
continues et que $\mathcal{T}_\omega$ rend continues les projections
$\pi_E$ et $\pi_F$.

Soit $z^*\in(E\times F)^*$. Soit $x^*\in E^*$ et $y^*\in F^*$ donnés par
le lemme ci-dessus. Comme les projections sont continues pour la topologie
produit, et que la continuité est préservée par la somme et la composition,
on en déduit que $z^*$ est continue au sens de la topologie produit. Ceci
montre que la topologie produit est plus fine que la topologie faible,
c'est-à-dire $\mathcal{T}_\pi\supseteq \mathcal{T}_\omega$.

Soit $U$ un ouvert élémentaire de $\mathcal{T}_\pi$.
On veut montrer qu'il s'agit d'un ouvert au sens de la topologie faible
sur le produit. Soient $V_i$ et $W_i$ des ouverts de $(E, \omega_E)$ et
$(F, \omega_F)$, $i\in I$, tels que
$$U=\bigcup_{i\in I} V_i\times W_i.$$

Pour conclure que $U\in\mathcal{T}_\omega$, il suffit de montrer que chaque
$V_i\times W_i$ est ouvert au sens de $\mathcal{T}_\omega$. Soit $i\in I$.
Alors $V_i$ est une union d'ouverts éléments, qui sont
des intersections de $(x_{i, j}^*)^{-1}(O_{i, j})$
avec $O_{i, j}$ ouverts de $\mathbb R$ et $x_{i, j}^*\in E^*$ pour $j = 1, \ldots, m$.
En considérant $z_{i, j}^*=x^*_{i, j}\circ\pi_E\in (E\times F)^*$, on a
$$V_i\times F = \bigcup\bigcap_{j=1}^m (z_{i, j}^*)^{-1}(O_{i, j})$$
qui montre que $V_i\times F$ est un ouvert au sens de $\mathcal{T}_\omega$.

De la même manière, on montre que $E\times W_i$ est un ouvert de
la topologie $\mathcal{T}_\omega$ pour chaque $i\in I$ et on conclut
par l'égalité
$$V_i\times W_i = (V_i\times F)\cap(E\times W_i).$$
%%% Local Variables:
%%% mode: latex
%%% TeX-master: "analyse3"
%%% End:
