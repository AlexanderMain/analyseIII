\documentclass[a4paper, 12pt]{book}

\usepackage[utf8]{inputenc}
\usepackage[T1]{fontenc}
\usepackage[french]{babel}
\usepackage{amsmath, amssymb, amsthm, verbatim}
\usepackage{IEEEtrantools}
\usepackage[margin=1in]{geometry}
\usepackage[colorlinks, linkcolor=blue]{hyperref}
\usepackage{epigraph}
\usepackage{mathrsfs}
\usepackage[toc,page]{appendix}
\usepackage{tikz-cd}
\usepackage{xcolor}
%Header stuff
\usepackage{fancyhdr}
\pagestyle{fancy}
\renewcommand{\chaptermark}[1]{\markboth{#1}{}}
\renewcommand{\sectionmark}[1]{\markright{\thesection\ #1}}
\fancyhf{}
\fancyhead[LE,RO]{\bfseries\thepage}
\fancyhead[LO]{\bfseries\rightmark}
\fancyhead[RE]{\bfseries\leftmark}
\renewcommand{\headrulewidth}{0.5pt}
\renewcommand{\footrulewidth}{0pt}
\setlength{\headheight}{14.85pt} % space for the rule
\fancypagestyle{plain}{%
  \fancyhead{} % get rid of headers on plain pages
  \renewcommand{\headrulewidth}{0pt} % and the line
}

\usepackage{dsfont}
\newcommand{\ind}{\mathds{1}}
\newcommand{\ssi}{si et seulement si}
\renewcommand{\Re}{\mathrm{Re}}
\renewcommand{\Im}{\mathrm{Im}}
\renewcommand{\phi}{\varphi}
\renewcommand{\epsilon}{\varepsilon}


\title{Analyse Mathématique III}
\author{James Campbell Alexander Main}
\date{\today}

\theoremstyle{definition} \newtheorem{df}{D\'{e}finition}[chapter]
\theoremstyle{definition} \newtheorem{ex}[df]{Exemple}
\theoremstyle{definition} \newtheorem{thm}[df]{Th\'{e}or\`{e}me}
\theoremstyle{definition} \newtheorem{cor}[df]{Corollaire}
\theoremstyle{definition} \newtheorem{lem}[df]{Lemme}
\theoremstyle{definition} \newtheorem{prop}[df]{Proposition}
\theoremstyle{definition} \newtheorem{rem}[df]{Remarque}

\theoremstyle{definition} \newtheorem{exo}{Exercice}[chapter]


\usepackage{marvosym} %for smileys

\begin{document}

\frontmatter

\maketitle

\tableofcontents
\chapter{Un petit mot avant de commencer}
Ce document a pour but de créer des notes de cours
pour le cours d'Analyse Mathématique III dispensé
à l'UMONS afin d'altérer la structure dans un format
plus clair que les notes prises au cours.

Parmi les références citées, vous pouvez trouver des exercices
à réaliser dans les suivantes: \cite{hb:exo1}.

\section*{Remerciements}

Merci beaucoup à Damien Galant et Trudel Nguepi pour leur relecture du document.

\mainmatter
\part{Premier quadrimestre}
\chapter{Espaces de Banach: exemples}
Dans ce chapitre nous présentons plusieurs exemples d'espaces de Banach,
notamment les espaces $\ell^p$ de suites et deux exemples d'espaces de fonctions.

La lettre $\mathbb{K}$ est utilisée pour représenter le
corps des nombres réels $\mathbb{R}$ ou celui des nombres
complexes $\mathbb{C}$.
\section{Exemples tirés du cours d'Analyse II}

Nous avons vu l'année passée que tous les espaces vectoriels
de dimension finie étaient complets.
\begin{prop}
  Soit $(E, \|.\|)$ un espace vectoriel sur $\mathbb{K}$
  de dimension finie. Il s'agit d'un espace de Banach.
\end{prop}
\begin{proof}(Esquisse de preuve)

  \textbf{Remarque}: un point intéressant de la dimension finie est qu'on
  peut travailler, si on le souhaite, avec sa norme
  préférée étant donné qu'elles sont toutes équivalentes.

  Etant donné une suite de Cauchy, il est évident (via l'inégalité
  triangulaire) que chaque composante (en se fixant au préalable
  une base) est de Cauchy et donc converge dans $\mathbb{K}$

  Ceci fournit un candidat limite et la vérification se fait
  facilement en prenant par exemple la norme $\|.\|_1$.
\end{proof}

Les espaces de Hilbert sont un autre type d'espace de Banach
que nous avons étudié en analyse II. Ils sont complets par
définition.

\section{Espaces de suites $\ell^p$}
\begin{df}
  Soit $1\leq p<\infty$ un réel. On appelle $\ell^p$ (noté aussi
  $\ell^p(\mathbb{N})$) l'ensemble  de suites suivant:
  \begin{equation*}
    \left\{(x_n)_{n\in\mathbb{N}}\mid
       \forall  n\in\mathbb{N}, x_n\in\mathbb{K} \mbox{ et }
      \sum_{n=0}^\infty|x_n|^p<+\infty\right\}
  \end{equation*}

  On définit l'application $\|.\|_p:\ell^p\to\left[0, +\infty\right[$
  par
  $$ \forall  x=(x_n)_{n\in\mathbb{N}}\in \ell^p, \|x\|_p=
  \left(\sum_{n=0}^\infty|x_n|^p\right)^{1/p}$$
\end{df}

Il n'est pas clair à première vue que $\ell^p$ est un espace
vectoriel, lorsque $p>1$. Efforçons de montrer des résultats
permettant une preuve courte de cela, tout en délivrant
des résultats simplifiant la démonstration du fait
que l'application $\|.\|_p$ définit une
norme sur $\ell^p$.
Nous aurons besoin d'une première inégalité célèbre afin
de prouver ce résultat.

\begin{thm}[Inégalité de Hölder]
  Soient $n\geq 1$ un nombre naturel, $p, q>1$ deux réels
  conjugués, c'est-à-dire vérifiant l'égalité:
  $$\frac{1}{p}+\frac{1}{q}=1$$

  Pour tous $x=(x_1, \ldots, x_n), y=(y_1, \ldots, y_n)\in
  \mathbb{K}^n$, l'inégalité suivante est vérifiée:
  $$\sum_{k=1}^n|x_k\cdot y_k|\leq
  \left(\sum_{k=1}^n|x_k|^p\right)^{\frac{1}{p}}
  \left(\sum_{k=1}^n|y_k|^q\right)^{\frac{1}{q}}$$
\end{thm}

Prouvons préalablement le lemme suivant:
\begin{lem}[Inégalité de Young]
  Pour tous réels
  $a, b \geq 0$, l'inégalité suivante est vérifiée:
  \begin{equation}\label{hold:pv1}
    a\cdot b\leq\frac{a^p}{p}\cdot\frac{b^q}{q}.
  \end{equation}
\end{lem}

\begin{proof}
  Si $a$ ou $b$ est nul, l'inégalité est immédiate. Supposons
  maintenant que $a$ et $b$ sont non nuls.

  Il est aisé de vérifier, à l'aide de la dérivation, que
  la fonction $\ln:\left]0, +\infty\right[\mapsto\mathbb{R}$
  est concave, c'est-à-dire:

  $$  \forall  t\in [0, 1],  \forall  x, y >0,
  \ln(tx+(1-t)y)\geq t\ln(x)+ (1-t)\ln(y)$$

  Puisque $p, q >1$, on a:
  \begin{IEEEeqnarray*}{rCl}
    \ln\left(\frac{a^p}{p}+\frac{b^q}{q}\right)
    & \geq & \frac{1}{p}\ln(a^p)+\frac{1}{q}\ln(b^q) \\
    & = & \ln(a)+\ln(b) \\
    & = & \ln(a\cdot b)
  \end{IEEEeqnarray*}

  Etant donné que la fonction exponentielle est croissante,
  on a bien l'inégalité \ref{hold:pv1}.
\end{proof}

Prouvons l'inégalité de Hölder:
\begin{proof}
Soient $x=(x_k)_{1\leq k\leq n}, y=(y_k)_{1\leq k\leq n}\in\mathbb{K}^n$.
Supposons les tous les deux non nuls, puisque dans ce cas, l'inégalité
du théorème est claire.

On pose, pour $j\in\{1, \ldots, n\}$:
$$A_j = \frac{|x_j|}{\left(\sum_{k=1}^n|x_k|^p\right)^{\frac{1}{p}}}
\mbox{ et }
B_j = \frac{|y_j|}{\left(\sum_{k=1}^n|y_k|^q\right)^{\frac{1}{q}}}$$

Ces quotients sont bien définis car $x$ et $y$ sont non nuls.
Par l'inégalité (\ref{hold:pv1}), on a que
\begin{IEEEeqnarray*}{rCl}
  \left(\sum_{j=1}^n
    \frac{|x_j|}{\left(\sum_{k=1}^n|x_k|^p\right)^{\frac{1}{p}}}
  \cdot
    \frac{|y_j|}{\left(\sum_{k=1}^n|y_k|^q\right)^{\frac{1}{q}}}\right)
  & = &\sum_{j=1}^nA_j\cdot B_j
  \\&\leq& \frac{1}{p}\sum_{j=1}^nA_j^p +\frac{1}{q}\sum_{j=1}^nB_j^q \\
  &=& \frac{1}{p} + \frac{1}{q} = 1
\end{IEEEeqnarray*}

Ce qui implique le résultat.
\end{proof}

Introduisons maintenant une autre inégalité célèbre qui nous
permettra de montrer que $\ell^p$ est un espace vectoriel.

\begin{thm}[Inégalité de Minkowski (dimension finie)]
  Soient $n\geq 1$  un nombre naturel,
  $p\in\left[1, +\infty\right[$ un nombre réel,
  $x=(x_k)_{1\leq k\leq n}, y=(y_k)_{1\leq k\leq n}\in \mathbb{K}^n$.

  L'inégalité suivante est vérifiée:
  $$\|x+y\|_p\leq \|x\|_p + \|y\|_p$$
\end{thm}
\begin{proof}

  Supposons $x, y$ non nuls, sinon le résultat est
  clair.

  Si $p=1$, il suffit d'itérer l'inégalité triangulaire
  pour le module afin de prouver l'assertion. Supposons
  par conséquent que $p>1$.

  \textbf{Remarque}: Cette preuve est fort calculatoire.
  Elle n'a pas vraiment d'intérêt au niveau des idées.

  \begin{IEEEeqnarray*}{rCl}
    \|x+y\|_p^p & = &\sum_{k=1}^n|x_k+y_k|^p \\
    & \leq & \sum_{k=1}^n |x_k+y_k|^{p-1}(|x_k|+|y_k|) \\
    & = & \sum_{k=1}^n |x_k\|x_k+y_k|^{p-1} +
    \sum_{k=1}^n |y_k\|x_k+y_k|^{p-1}
  \end{IEEEeqnarray*}

  Soit $q$ le réel conjugué de $p$. En appliquant l'inégalité
  de Hölder au dernier terme de l'inégalité précédente, on
  obtient:

  \begin{IEEEeqnarray*}{rCl}
    \|x+y\|_p^p & \leq &\left(\sum_{k=1}^n|x_k|^p\right)^\frac{1}{p}
    \cdot \left(\sum_{k=1}^n|x_k+y_k|^{(p-1)q}\right)^\frac{1}{q} +
    \left(\sum_{k=1}^n|y_k|^p\right)^\frac{1}{p}
    \cdot \left(\sum_{k=1}^n|x_k+y_k|^{(p-1)q}\right)^\frac{1}{q} \\
    & = & \left(\sum_{k=1}^n|x_k+y_k|^{p}\right)^\frac{1}{q}
    (\|x\|_p+\|y\|_p)\\
    & = & \left(\|x + y\|_p^{p}\right)^\frac{1}{q}(\|x\|_p+\|y\|_p)
  \end{IEEEeqnarray*}

  Ceci implique l'inégalité nous permettant de conclure:
  $$ \|x\|_p+\|y\|_p\geq \left(\|x + y\|_p^{p}\right)^{1-\frac{1}{q}}
  = \|x + y\|_p$$
\end{proof}

Le résultat se généralise facilement aux espaces $\ell^p$, et
nous avons donc en même temps l'argument le plus important
de la preuve que $\ell^p$ est un espace vectoriel, mais également
l'inégalité triangulaire pour la norme $\|.\|_p$.

\begin{thm}[Inégalité de Minkowski (Généralisation à $\ell^p$)]
  Soient $p\in\left[1, +\infty\right[$ un nombre réel,
  $x=(x_n)_{n\in\mathbb{N}}, y=(y_n)_{n\in\mathbb{N}}\in \ell^p$.

  On a:
  $\|x+y\|_p\leq \|x\|_p + \|y\|_p$
\end{thm}
\begin{proof}
  Etant donné qu'on a le résultat pour toutes les suites de
  sommes partielles, il suffit de passer à la limite
  pour conclure.
\end{proof}

\begin{exo} Rédiger en détail la preuve des affirmations suivantes:
  \begin{itemize}
  \item L'ensemble $\ell^p$ est un espace vectoriel sur $\mathbb{K}$.
  \item L'application $\|.\|_p$ définit bien une norme sur $\ell^p$.
  \end{itemize}
\end{exo}


Il est légitime de se demander pourquoi parler de ces espaces
dans un chapitre réservé aux espaces de Banach. C'est parce
qu'il s'agit d'exemples classiques:

\begin{prop}\label{lp:cpl}
  Soit $p> 1$ un nombre réel. L'espace vectoriel normé $(\ell^p, \|.\|_p)$
  est un espace de Banach.
\end{prop}

\begin{exo}
  Effectuer la preuve de la proposition \ref{lp:cpl}
\end{exo}

Nous n'effectuerons pas la preuve dans cette section, elle
est laissée à titre d'exercice. Nous allons toutefois
montrer le cas $p=1$. Les idées mises
en oeuvre dans cette preuve sont intéressantes et il est
conseillé de les revoir.

\begin{prop}
  L'espace vectoriel normé $(\ell^1, \|.\|_1)$ est un espace de Banach.
\end{prop}

\begin{proof}
  Soit $\left(x^{(n)}\right)_{n\in\mathbb{N}}=
  \left(\left(x_k^{(n)}\right)_k\right)_n$\footnote{
    On note en indice la ``composante'' dans $\ell^1$ et en
    exposant l'indice de la suite} une suite de
  Cauchy dans $(\ell^1, \|.\|_1)$. Montrons qu'elle est convergente
  au sens de la norme $\|.\|_1$.

  En traduisant l'hypothèse sur la suite, il est aisé de déduire que
  pour tout $k\in\mathbb{N}$, la suite $(x_k^{(n)})_n$ est de Cauchy dans
  $\mathbb{K}$ qui est complet. Soit $x_k\in\mathbb{K}$ la limite
  de cette suite dans $\mathbb{K}$ .

  Nous avons ainsi obtenu un candidat limite $x=(x_k)_k$. Nous devons
  montrer qu'il s'agit bien d'un élément de $\ell^1$ et qu'il s'agit de
  la limite de la suite $(x^{(n)})_n$.

  Soit $\varepsilon>0$. Puisque la suite $(x^{(n)})$ est de
  Cauchy, il est vrai que:
%  $$ \exists  N \forall  p, q\geq N, \|x_p-x_q\|_1<\varepsilon$$
  $$ \exists  N\geq 0, \forall  p, q\geq N, \forall  n\geq 0,
  \sum_{k=0}^n| x^{(p)}_k - x^{(q)}_k|\leq\varepsilon$$

  En faisant tendre $q\to+\infty$, on en déduit que:
  $$ \exists  N\geq 0, \forall  p\geq N, \forall  n\geq 0,
  \sum_{k=0}^n| x^{(p)}_k - x_k|\leq\varepsilon$$
  Ce qui permet facilement de conclure que
  $x^{(n)}\xrightarrow[n\to+\infty]{\|.\|_1}x$

  Pour montrer que $x\in\ell^1$, il suffit d'observer l'inégalité
  suivante, valable pour tous naturels $n, N$ considérés.
  $$\sum_{k=0}^N|x_k|
  \leq \sum_{k=0}^N|x_k-x_k^ {(n)}|+ \sum_{k=0}^N|x_k^{(n)}|
  \leq \|x-x^{(n)}\|_1+\|x^{(n)}\|_1$$
  Au vu de la convergence que nous venons d'établir, le premier terme
  du membre de droite converge vers $0$. Le second quant à lui est
  borné par une constante indépendante de $n$ car il s'agit d'un
  élément d'une suite de Cauchy. Ceci implique que la série
  définissant $\|x\|_1$ est bien convergente.

\end{proof}

Il reste encore un espace lié aux espaces $\ell^p$ que
nous n'avons pas introduit: l'espace des suites $\ell^\infty$.

\begin{df}
  L'espace des suites bornées, noté $\ell^\infty$
  correspond à l'ensemble suivant:
  $$\ell^\infty=\{x\in\mathbb{K}^{\mathbb{N}}\mid
  \sup_{n\in\mathbb{N}}|x_n|<\infty\}$$
\end{df}

\begin{rem}
  Il est utile de remarquer que $1$ et $\infty$ sont
  conjugués (par exemple pour les espaces duaux, cf.
  exemples \ref{dual:ex:r2n1} et \ref{dual:ex:r2ninfty}
  de la section \ref{dual:ex:dimf}).
\end{rem}

\begin{exo}
  Montrer que l'application suivante définit bien une
  norme sur $\ell^\infty$:
  $$\|.\|_\infty:\ell^\infty\to\left[0,+\infty\right[:
  (x_n)_{n\in\mathbb{N}}\mapsto\sup_{n\in\mathbb{N}}|x_n|$$
\end{exo}

Tout comme ses confrères, il s'agit d'un espace de
Banach. Cela fait l'objet du résultat suivant:
\begin{prop}
  L'espace vectoriel $(\ell^\infty, \|.\|_\infty)$ est
  un espace de Banach.
\end{prop}
La preuve est fortement similaire à celle pour l'espace
de suites $\ell^1$. Elle sera tout de même explicitée
afin de permettre la lecture d'une version adaptée de
la preuve ci-dessus.
\begin{proof}
  Soit $\left(x^{(n)}\right)_{n\in\mathbb{N}}=
  \left(\left(x_k^{(n)}\right)_k\right)_n$ une suite de
  Cauchy dans $(\ell^\infty, \|.\|_\infty)$. Montrons qu'elle est convergente
  au sens de la norme $\|.\|_\infty$.

  En traduisant l'hypothèse sur la suite, il est aisé de déduire que
  pour tout $k\in\mathbb{N}$, la suite $(x_k^{(n)})_n$ est de Cauchy dans
  $\mathbb{K}$ qui est complet. Soit $x_k\in\mathbb{K}$ la limite
  de cette suite dans $\mathbb{K}$ .

  Nous avons ainsi obtenu un candidat limite $x=(x_k)_k$. Nous devons
  montrer qu'il s'agit bien d'un élément de $\ell^\infty$ et qu'il s'agit de
  la limite de la suite $(x^{(n)})_n$.

  Soit $\varepsilon>0$. Puisque la suite $(x^{(n)})$ est de
  Cauchy, il est vrai que:
%  $$ \exists  N \forall  p, q\geq N, \|x_p-x_q\|_1<\varepsilon$$
  $$ \exists  N\geq 0, \forall  p, q\geq N,
   \forall {k\geq0},| x^{(p)}_k - x^{(q)}_k|\leq\varepsilon$$

  En faisant tendre $q\to+\infty$, on en déduit que:
  $$ \exists N\geq 0, \forall  p\geq N,
   \forall {k\geq0} , | x^{(p)}_k - x_k| \leq \varepsilon$$
  Ce qui permet facilement de conclure que
  $x^{(n)}\xrightarrow[n\to+\infty]{\|.\|_\infty}x$

  Pour montrer que $x\in\ell^\infty$, il suffit d'observer l'inégalité
  suivante, valable pour tous naturels $n, k$ considérés.
  $$|x_k|
  \leq |x_k-x_k^ {(n)}|+ |x_k^{(n)}|
  \leq \|x-x^{(n)}\|_\infty+\|x^{(n)}\|_\infty$$
  Au vu de la convergence que nous venons d'établir, le premier terme
  du membre de droite converge vers $0$. Le second quant à lui est
  borné par une constante indépendante de $n$ car il s'agit d'un
  élément d'une suite de Cauchy. Ceci implique que la suite des
  $(x_k)_{k\in\mathbb{N}}$ est bornée et donc que $x\in\ell^\infty$.
\end{proof}

\section{Espaces de fonctions continues}
Un autre exemple d'espace de Banach dont nous avons
les outils pour montrer la complétude est l'espace
des fonctions continues sur l'intervalle $[0, 1]$.
Tout ce qui suit dans cette section peut être
généralisé aux fonctions définie sur un intervalle
fermé borné quelconque.

\begin{df}
  On note $$\mathscr{C}[0, 1]=\left\{
    f:[0, 1]\to\mathbb{K}\mbox{ continue}\right\}$$
  l'espace vectoriel de fonctions continues sur
  l'intervalle $[0, 1].$
  On le munit de la norme $\|.\|_\infty$ définie par:
  $$\|.\|_\infty:\mathscr{C}[0, 1]\to \left[0, +\infty\right[:
  f\mapsto \|f\|_\infty=\max_{x\in [0, 1]}|f(x)|$$
\end{df}

Le maximum apparaissant dans la norme est bien défini,
car les fonctions considérées sont continues sur un
compact et atteignent donc leurs bornes.

\begin{prop}
  L'espace des fonctions continues sur l'intervalle [0, 1]
  est complet au sens de la norme $\|.\|_\infty$.
\end{prop}

\begin{proof}
  Soit $(f_n)_{n\in\mathbb{N}}$ une suite de Cauchy dans
  $\mathscr{C}[0, 1]$. En retraduisant cette hypothèse
  en remplaçant le maximum définissant $\|.\|_\infty$
  par un quantificateur universel, on obtient la phrase
  quantifiée suivante:
  \begin{equation}\label{cpl:fn}
    \forall \varepsilon>0, \exists N\in\mathbb{N},
  \forall p, q\geq N , \forall x\in [0, 1],
  |f_p(x)-f_q(x)|\leq\varepsilon
  \end{equation}

  On remarque que pour tout $x\in[0, 1]$, la suite
  $(f_n(x))_{n\in\mathbb{N}}$ est de Cauchy dans
  $\mathbb{K}$ qui est complet. Il existe donc
  $f$ limite ponctuelle de $(f_n)_{n\in\mathbb{N}}$.

  Pour conclure, il suffit de montrer que
  la suite $(f_n)_{n\in\mathbb{N}}$ converge
  au sens de $\|.\|_\infty$ vers $f$. Cela
  garantira la continuité de $f$, car $f$
  sera la limite uniforme de la suite.

  En faisant $q\to\infty$ dans la phrase quantifiée
  \ref{cpl:fn}, on a le résultat.

\end{proof}

\section{Espaces d'applications linéaires}

Soient $(E, \|.\|_E)$ et $(F, \|.\|_F)$ deux espaces
vectoriels normés sur $\mathbb{K}$. Rappelons certaines
propriétés et définitions vues l'année passée.
\begin{df}
  On note $\mathscr{L}(E, F)$ l'espace vectoriel des
  applications linéaires continues de $E$
  dans $F$. Il s'agit d'un espace vectoriel normé,
  muni de la norme opérateur définie par:

  $$\|.\|:\mathscr{L}(E, F)\to \left[0, +\infty\right[:
  T\mapsto \|T\| = \sup_{\substack{x\in E\\\|x\|_\leq 1}}\|T(x)\|_F$$
\end{df}

On a également vu plusieurs équivalences
pour montrer qu'une application linéaire
est continue. Rappelons les.
\begin{prop}\label{cont:lin}
  Soit $T: E\to F$ linéaire. Les assertions suivantes sont équivalentes:
  \begin{itemize}
  \item $T$ est continue.
  \item $T$ est continue en $0$.
  \item $T$ est bornée sur la boule unité de $E$.
  \item Il existe une constante $C>0$ telle que
    pour tout $x\in E$, $\|T(x)\|_F\leq C\|x\|$
    \footnote{La constante optimale, s'il en
      existe une, est $\|T\|$.}.
  \end{itemize}
\end{prop}



Remarquez également qu'il existe plusieurs
formulations équivalentes pour la norme
opérateur:
\begin{prop}\label{cont:norm}
  Soit $T\in\mathscr{L}(E, F)$. Les égalités suivantes
  sont satisfaites:
  \begin{IEEEeqnarray*}{rCl}
    \|T\| & = & \sup_{\substack{x\in E\\\|x\|_E = 1}}\|T(x)\|_F \\
    & = & \sup_{\substack{x\in E\\\|x\|_E < 1}}\|T(x)\|_F
  \end{IEEEeqnarray*}

\end{prop}

Il est utile, si vous avez des difficultés avec ces objets,
d'essayer de les manipuler en résolvant les exercices:
\begin{exo}
  Prouver les propositions \ref{cont:lin} et \ref{cont:norm}.
\end{exo}
\begin{exo}[\og Sous-multiplicativité de la norme\fg]
  Soit $(G, \|.\|_G)$ un espace vectoriel normé sur $\mathbb{K}$.
  Soient $f: E\to F$, $g:F\to G$ des applications linéaires
  continues. Montrer l'inégalité
  $$\|g\circ f\|\leq \|g\|\cdot \|f\|$$
\end{exo}

L'espace $\mathscr{L}(E, F)$ n'est \emph{pas} un espace de Banach
en général. Il faut ajouter une hypothèse sur $F$ pour avoir
la complétude de cet espace.

\begin{prop}\label{lin:cpl:imp}
  Si $F$ est complet, alors $\mathscr{L}(E, F)$ est un
  espace de Banach.
\end{prop}
\textbf{Remarque}: il n'y a pas de condition supplémentaire
à imposer à l'espace $E$. En particulier, le résultat n'exige
pas sa complétude.

La réciproque de ce résultat est également vraie, nous
l'énoncerons dans un futur chapitre.

\begin{proof}
  Soit $(T_n)_{n\in\mathbb{N}}$ une suite de Cauchy dans
  $\mathscr{L}(E, F)$, c'est-à-dire:
  $$\forall \varepsilon>0, \exists N\geq 0, \forall p, q\geq N,
  \|T_p-T_q\|\leq \varepsilon$$

  Alors, on a l'assertion suivante:
  \begin{IEEEeqnarray*}{rl}
    \IEEEeqnarraymulticol{2}{l}{
    \forall \varepsilon>0, \exists N\geq 0, \forall p, q\geq N,
    \forall x\in E,}\\ \qquad & \|T_p(x)-T_q(x)\|\leq
    \|T_p-T_q\|\cdot \|x\|_E\leq \varepsilon\cdot\|x\|_E
  \end{IEEEeqnarray*}

  Cela implique que pour tout $x\in E$, la suite $(T_n(x))_{n\in\mathbb{N}}$
  est de Cauchy dans $F$ qui par hypothèse est complet. Soit $T(x)$
  limite de cette suite dans $F$.

  Montrons que $T$ est la limite, au sens de la norme opérateur, de la
  suite $(T_n)_{n\in\mathbb{N}}$.

  En reprenant la phrase quantifiée ci-dessus, en considérant $x\in E$
  de norme inférieure à $1$ et en faisant tendre $q$ vers l'infini,
  on est assuré de la convergence.

  L'application $T$ est bien linéaire car elle est limite simple
  d'applications
  linéaires. Elle est bornée sur la boule unité car pour tout naturel
  $n$ on a l'inégalité:
  $$\|T\|\leq \|T-T_n\| + \|T_n\|$$
\end{proof}

\section{Sous-espaces vectoriels et complétude}
Lorsqu'on considère un espace de Banach et un sous-espace vectoriel de ce
dernier, on peut se poser la question suivante: \og Ce sous-espace est-il
complet pour la norme induite?\fg.
Le résultat suivant permet de donner un critère pour vérifier cela.

\begin{thm}
  Soit $(E, \|.\|)$ un espace de Banach et $F$ un sous-espace vectoriel de $E$.
  Alors $(F, \|.\|)$ est un espace de Banach si et seulement si $F$ est fermé
  dans $E$.
\end{thm}
\begin{proof}
  Supposons que $F$ est complet et montrons qu'il s'agit d'un fermé au
  sens de $E$. Soient $(x_n)_{n\in\mathbb{N}}$ une suite d'éléments de $F$
  et $x\in E$ tels que $x_n\to x$. Puisque la suite $(x_n)_{n\in\mathbb{N}}$
  est convergente, elle est de Cauchy. Par complétude de $F$, elle admet
  une limite $y\in F$. Par unicité de la limite, $x = y\in F$.

  Réciproquement supposons que $F$ est fermé. Soit $(x_n)_{n\in\mathbb{N}}$
  une suite de Cauchy dans $F$. Alors c'est une suite de Cauchy d'éléments
  de $E$; elle admet une limite $x$ dans $E$. Puisque $F$ est fermé, la
  suite considérée converge bien dans $F$.
\end{proof}

Le résultat est également vrai dans le cadre des espaces métriques (la même
preuve fonctionne).

Fournissons une application de ce résultat à un exemple. On introduit
un sous-espace d'un des exemples présentés ci-avant.

\begin{df}
  L'espace $c_0$ est l'espace des suites dans $\mathbb{K}$
  convergeant vers $0$. On le munit de la norme $\|.\|_\infty$.
\end{df}

Etant donné que toutes les suites convergentes sont bornées,
$c_0$ est un sous-espace vectoriel de $\ell^\infty$.

\begin{prop}
  $(c_0, \|.\|_\infty)$ est un espace de Banach.
\end{prop}

\begin{proof}
  Au vu du résultat ci-avant, il suffit de montrer que $c_0$ est
  fermé dans $\ell^\infty$.

  Soit $(x^{(n)})_{n\in\mathbb N}$ une suite d'éléments de $c_0$
  convergeant vers $x$ un élément de $\ell^\infty$. Montrons que
  $x\in c_0$.

  Soit $\varepsilon > 0$. Il existe $N$ tel que pour tout naturel $n\geq N$,
  $\|x^{(n)}-x\|_\infty\leq \varepsilon/2$. \'{E}tant donné que $x^{(N)}$
  est un élément de $c_0$, il existe $k_0$ tel que pour tout $k\geq k_0$,
  $|x^{(N)}_k|\leq \varepsilon /2$. Pour tout $k\geq k_0$, on a
  $$|x_k|\leq |x_k - x^{(N)}_k| + |x^{(N)}_k|
  \leq \|x - x^{(N)}\|_\infty + |x^{(N)}_k| \leq \varepsilon$$
  ce qui montre bien que $x$ est élément de $c_0$.
\end{proof}



%%% Local Variables:
%%% mode: latex
%%% TeX-master: "../analyse3"
%%% End:


\chapter{Espaces duaux: exemples}
\section{Généralités}
Soit $\mathbb{K}$ un corps, avec $\mathbb{K}= \mathbb{R}$
ou $\mathbb{K}=\mathbb{C}$.
Soit $(E, \|.\|)$ un espace vectoriel sur $\mathbb{K}$.
\begin{df}
  Le dual de $E$, noté $E^*$ correspond à $\mathscr{L}(E, \mathbb{K})$ ie.
  l'ensemble des formes linéaires continues.
\end{df}

Par la proposition \ref{lin:cpl:imp}, étant donné que $\mathbb{K}$ est complet,
on a que $(E^*, \|.\|)$ est un espace de Banach (la norme considérée est
la norme opérateur).

Dans ce chapitre plutôt que de s'attarder sur des propriétés générales
des espaces duaux, nous étudierons des exemples et nous identifierons
les espaces duaux d'espaces connus à d'autres espaces que nous connaissons.

\textbf{Remarque}: les intuitions sur comment trouver à quel espace s'identifie
le dual seront omises.

\section{Exemples en dimension finie}\label{dual:ex:dimf}
En dimension finie, il est connu que le dual algébrique est de même dimension
que l'espace vectoriel considéré. De plus, comme toutes les applications
linéaires définies sur un espace vectoriel de dimension finie
sont continues, il suffit de vérifier qu'on a une isométrie entre
les espaces considérés pour avoir l'identification.

\begin{exo}
Soient $(X, d_X),(Y, d_Y)$ deux espaces métriques, $i:X\to Y$ une isométrie,
c'est-à-dire \og $i$ préserve les distances\fg, c'est-à-dire:
$$\forall x, y\in X, d_X(x, y) = d_Y(i(x), i(y))$$

Montrer que $i$ est injective.
\end{exo}

Dans les calculs relatifs à tous les exemples
suivants, il n'est jamais montré que l'application
considérée est linéaire. Cela est laissé comme exercice.

\begin{ex}
  On a $(\mathbb{R}^2, \|.\|_2)^*\equiv (\mathbb{R}^2, \|.\|_2)$
  via l'isométrie:
  \begin{IEEEeqnarray*}{rrClrCl}
    i: & (\mathbb{R}^2, \|.\|_2) & \to & \IEEEeqnarraymulticol{3}{l}{(\mathbb{R}^2, \|.\|_2)^*} \\
    & (x_1, x_2)&\mapsto& i(x_1, x_2):&(\mathbb{R}^2, \|.\|_2)&\to&\mathbb{R} \\
    &&&&(x, y)&\mapsto& x_1x + x_2 y
  \end{IEEEeqnarray*}
\end{ex}

Il suffit de prouver que $i$ est une isométrie étant donné que
l'espace considéré est de  dimension finie.

Soit $(x_1, x_2)\in\mathbb{R}^2$. On a (en utilisant l'inégalité
de Cauchy-Schwarz):
\begin{IEEEeqnarray*}{rCl}
  \|i(x_1, x_2)\| = \sup_{x^2+y^2\leq 1}|x_1 x + x_2 y|
  & \leq & \sup_{x^2+y^2\leq 1}|x_1x| + |x_2 y| \\
  & \leq & \sup_{x^2+y^2\leq 1} (x_1^2+x_2^2)^{1/2}(x^2+y^2)^{1/2} \\
  & \leq & (x_1^2+x_2^2)^{1/2}\sup_{x^2+y^2\leq 1}(x^2+y^2)^{1/2}
  = \|(x_1, x_2)\|_2
\end{IEEEeqnarray*}

Réciproquement, il suffit de considérer $z = \frac{(x_1, x_2)}{\|(x_1, x_2)\|_2}$;
on a $i(x_1, x_2)(z) = \|(x_1, x_2)\|_2$ ce qui fournit l'inégalité nous
permettant de conclure.

\begin{ex}\label{dual:ex:r2n1}
  On a $(\mathbb{R}^2, \|.\|_1)^*\equiv (\mathbb{R}^2, \|.\|_\infty)$
  via l'isométrie:
  \begin{IEEEeqnarray*}{rrClrCl}
    i: & (\mathbb{R}^2, \|.\|_\infty) & \to & \IEEEeqnarraymulticol{3}{l}{(\mathbb{R}^2, \|.\|_1)^*} \\
    & (x_1, x_2)&\mapsto& i(x_1, x_2):&(\mathbb{R}^2, \|.\|_1)&\to&\mathbb{R} \\
    &&&&(x, y)&\mapsto& x_1x + x_2 y
  \end{IEEEeqnarray*}
\end{ex}

Soit $(x_1, x_2)\in\mathbb{R}^2$. On a:
\begin{IEEEeqnarray*}{rCl}
  \|i(x_1, x_2)\| = \sup_{|x|+|y|\leq 1} |x_1 x + x_2 y|
  & \leq & \sup_{|x|+|y|\leq 1}|x_1 x| + |x_2 y| \\
  & \leq & \sup_{|x|+|y|\leq 1} \max(|x_1|, |x_2|) (|x|+|y|) \\
  & \leq & \|(x_1, x_2)\|_\infty \sup_{|x|+|y|\leq 1}(|x|+|y|)
  = \|(x_1, x_2)\|_\infty
\end{IEEEeqnarray*}

Si $\max(|x_1|, |x_2|) = |x_1|$, alors $i(x_1, x_2)$ atteint
$\|(x_1, x_2)\|_\infty$ en $(\mathrm{sign}(x_1), 0)$ qui est bien un élément de
la boule unité de $(\mathbb{R}^2, \|.\|_1)$. L'autre cas
est analogue. On peut donc conclure que $i$ est bien une isométrie.

\begin{ex}\label{dual:ex:r2ninfty}
  On a $(\mathbb{R}^2, \|.\|_\infty)^*\equiv (\mathbb{R}^2, \|.\|_1)$
  via l'isométrie:
  \begin{IEEEeqnarray*}{rrClrCl}
    i: & (\mathbb{R}^2, \|.\|_1) & \to & \IEEEeqnarraymulticol{3}{l}{(\mathbb{R}^2, \|.\|_\infty)^*}\\
    & (x_1, x_2)&\mapsto& i(x_1, x_2):&(\mathbb{R}^2, \|.\|_\infty)&\to&\mathbb{R} \\
    &&&&(x, y)&\mapsto& x_1x + x_2 y
  \end{IEEEeqnarray*}
\end{ex}

Soit $(x_1, x_2)\in\mathbb{R}^2$. On a:
\begin{IEEEeqnarray*}{rCl}
  \|i(x_1, x_2)\| = \sup_{\max(|x|,|y|)\leq 1} |x_1 x + x_2 y|
  & \leq & \sup_{\max(|x|,|y|)\leq 1}|x_1 x| + |x_2 y| \\
  & \leq & \sup_{\max(|x|,|y|)\leq 1} (|x_1|+|x_2|)\max(|x|, |y|)  \\
  & \leq & \|(x_1, x_2)\|_1 \sup_{\max(|x|,|y|)\leq 1}\max(|x|,|y|)
  = \|(x_1, x_2)\|_1
\end{IEEEeqnarray*}

Il est facile de vérifier que $i(x_1, x_2)$ atteint $\|(x_1, x_2)\|_1$
au point $(\mathrm{sign}(x_1), \mathrm{sign}(x_2))$ de la boule unité
de $(\mathbb{R}^2, \|.\|_\infty)$, ce qui fournit l'autre inégalité
désirée.

\begin{exo}
  Soient $p > 1$, $q>1$ le conjugué de $p$. Montrer qu'on a
  que $(\mathbb{R}^2, \|.\|_q)^*\equiv (\mathbb{R}^2, \|.\|_p)$.
\end{exo}

Notez que tous ces exemples peuvent se généraliser
en dimension supérieure à 2.

\section{Exemples en dimension infinie}
Contrairement aux exemples de la section précédente, il ne
suffit plus de vérifier que l'application considérée est
une isométrie car la surjectivité n'est plus garantie par
l'injectivité. Les exemples requièrent donc plus de travail.

Introduisons le premier espace sur lequel nous allons nous
concentrer, qui est un espace de suites.
\begin{df}
  L'espace $c_0$ est l'espace des suites dans $\mathbb{K}$
  convergeant vers $0$. On le munit de la norme $\|.\|_\infty$.
\end{df}

Etant donné que toutes les suites convergeantes sont bornées,
on a que $c_0$ est un sous-espace vectoriel de $\ell^\infty$.

\begin{ex}
  On a $(c_0, \|.\|_\infty)^*\equiv (\ell^1, \|.\|_1)$
  via l'isométrie:
  \begin{IEEEeqnarray*}{rrClrCl}
    i: & (\ell^1, \|.\|_1) & \to & \IEEEeqnarraymulticol{3}{l}{(c_0, \|.\|_\infty)^*}\\
    & x=(x_n)_{n\in\mathbb{N}}&\mapsto& i(x): & (c_0, \|.\|_\infty)&\to&\mathbb{R} \\
    &&&&(y_n)_{n\in\mathbb{N}}&\mapsto& \sum_{k=0}^\infty x_ky_k
  \end{IEEEeqnarray*}
\end{ex}

\textbf{L'application $i$ est une isométrie}:

Soit $x=(x_n)_{n\in\mathbb{N}}\in\ell^1$. La fonction $i(x)$
est bien définie car pour tout $y\in c_0$, on a l'inégalité
$$|i(x)(y)|\leq \sum_{k=0}^\infty |x_ky_k|\leq \|y\|_\infty \|x\|_1$$

Cela implique en particulier que $\|i(x)\|\leq\|x\|_1$. Montrons l'inégalité
réciproque.

Soit, pour tout naturel $N$, la suite $(y^{(N)}_n)_{n\in\mathbb{N}}$
définie par:
$$\forall n\in\mathbb{N},  y_n^{(N)}=\begin{cases} \mathrm{sign}(x_n)\mbox{ si $n\leq N$}
  \\ 0 \mbox{ sinon}\end{cases}$$

Alors $(y^{(N)}_n)_{n\in\mathbb{N}}$ appartient à la boule unité
de $c_0$ et on a $|i(x)(y^{(N)})|\to\|x\|_1$ quand $N$ tend vers
l'infini. Ceci implique que $\|i(x)\|\geq \|x\|_1$, ce qu'on voulait. \newline

\textbf{L'application $i$ est surjective}:

Vérifions la surjectivité de $i$.
Soit $x^*\in (c_0)^*$. On pose $x = (x^*(e_n))_{n\in\mathbb{N}}$ où $e_n$
est la suite de terme général $(\ind_{\{n\}}(k))_{k\in\mathbb{N}}$.

Il reste à montrer que $i(x) = x^*$, ce qui se vérifie par
un simple calcul (il suffit d'observer les sommes partielles
puis de passer à la limite), et que $x$ est bien élément de $\ell^1$.

Pour tout naturel $N$, on a:
$$\sum_{k=0}^N |x_k| = x^*(y^{(N)})\leq \|x^*\|$$
Ce qui implique que la série des termes de $x$ est
absolument convergente, ce qu'on voulait montrer.

\begin{ex}
  On a $(\ell^1, \|.\|_1)^*\equiv (\ell^\infty, \|.\|_\infty)$
  via l'isométrie:
  \begin{IEEEeqnarray*}{rrClrCl}
    i: & (\ell^\infty, \|.\|_\infty) & \to & \IEEEeqnarraymulticol{3}{l}{(\ell^1, \|.\|_1)^*}\\
    & x=(x_n)_{n\in\mathbb{N}}&\mapsto& i(x): & (\ell^1, \|.\|_1)&\to&\mathbb{R} \\
    &&&&(y_n)_{n\in\mathbb{N}}&\mapsto& \sum_{k=0}^\infty x_ky_k
  \end{IEEEeqnarray*}
\end{ex}

\textbf{L'application $i$ est une isométrie}:

Soit $x=(x_n)_{n\in\mathbb{N}}\in\ell^\infty$. La fonction $i(x)$
est bien définie car pour tout $y\in \ell^1$, on a l'inégalité
$$|i(x)(y)|\leq \sum_{k=0}^\infty |x_ky_k|\leq \|x\|_\infty \|y\|_1$$

Cela implique en particulier que $\|i(x)\|\leq\|x\|_\infty$. Montrons l'inégalité
réciproque.

Soit, pour tout naturel $N$, la suite $(y^{(N)}_n)_{n\in\mathbb{N}}$
définie par:
$$\forall n\in\mathbb{N} , y_n^{(N)}= \ind_{\{N\}}(n)\mathrm{sign}(x_n)$$

Alors $(y^{(N)}_n)_{n\in\mathbb{N}}$ est élément de la boule unité
de $\ell^1$ et on a $|i(x)(y^{(N)})| = |x_N|$ quel que soit
le naturel $N$ considéré.
Ceci implique que $\|i(x)\|\geq \|x\|_\infty$, ce qu'on voulait. \newline

\textbf{L'application $i$ est surjective}:

Vérifions la surjectivité de $i$.
Soit $x^*\in (\ell^1)^*$. On pose $x = (x^*(e_n))_{n\in\mathbb{N}}$ où $e_n$
est la suite de terme général $(\ind_{\{n\}}(k))_{k\in\mathbb{N}}$.

Il reste à montrer que $i(x) = x^*$, ce qui se vérifie par
un simple calcul (il suffit d'observer les sommes partielles
puis de passer à la limite), et que $x$ est bien élément de $\ell^\infty$.

Pour tout naturel $N$, on a:
$$|x_k| = x^*(y^{(N)})\leq \|x^*\|$$
Ce qui implique que $x$ est
une suite bornée, ce que l'on voulait montrer.


%%% Local Variables:
%%% mode: latex
%%% TeX-master: "../analyse3"
%%% End:


\chapter{Théorèmes de Hahn-Banach}
Dates (pour votre culture générale): Hahn (1879-1934), Banach (1892-1945).

Les théorèmes de Hahn-Banach sont des résultats d'analyse fonctionnelle
très importants. Ces théorèmes sont vus ici en deux formes:
les formes analytiques du théorème assurent qu'une forme linéaire
peut être étendue à tout l'espace en préservant certaines contraintes
iniatiales; les formes géométriques quant à elles sont des résultats
assurant qu'on peut séparer par des hyperplans deux convexes vérifiant
certaines hypothèses dans des espaces vectoriels normés.

\section{Formes analytiques}
\subsection{Espaces vectoriels sur $\mathbb{R}$}

\begin{thm}[Hahn-Banach -- Forme analytique (cas réel)] \label{hb:a1}
Soient $E$ un espace vectoriel sur $\mathbb{R}$, $G$ un sous-espace
vectoriel de $E$, $f: G \to \mathbb{R}$ une forme linéaire
et $p: E\to \mathbb{R}$ une application vérifiant les hypothèses suivantes:
\begin{itemize}
\item $\forall\lambda >0, \forall x \in E, p(\lambda x)=\lambda p(x)$
  (p est dite positivement homogène)
\item $\forall x,y \in E, p(x+y)\leq p(x)+p(y)$
  (p est dite sous-additive)
\item$\forall x\in G, f(x)\leq p(x)$
\end{itemize}
Alors il existe $g:E\to\mathbb{R}$ forme linéaire étendant $f$ et telle
que $\forall x \in E, g(x)\leq p(x)$.
\end{thm}

\begin{proof}
  Supposons que $G\neq E$, sinon le résultat est immédiat.

  Supposons donc qu'il existe $x_0\in E\setminus G$. Montrons
  qu'il est possible d'étendre $f$ à $V = G\oplus \mathbb{R}x_0$
  comme annoncé dans le théorème.

  On doit montrer qu'il existe un réel $g(x_0)$ tel que pour
  tout $x = y+tx_0\in V$, $g(x)\leq p(x)$, c'est-à-dire:
  $$f(y) + t g(x_0)\leq p(y + tx_0)$$

  Analysons les différents cas, selon le signe de $t$:

  \textbf{Cas 1, $t>0$}: la condition devient, par positive
  homogénéité de $p$ et linéarité de $f$
  $$\forall y\in G, g(x_0)\leq p\left(\frac{y}{t} + x_0\right)
  - f\left(\frac{y}{t}\right)$$
  Puisque $G$ est un sous-espace vectoriel, on peut réécrire la
  condition:
  $$\forall z\in G, g(x_0)\leq p(z + x_0) - f(z)$$

  \textbf{Cas 2, $t>0$}: la condition devient, par positive
  homogénéité de $p$ et linéarité de $f$
  $$\forall y\in G, g(x_0)\geq  f\left(\frac{-y}{t}\right)
  - p\left(\frac{-y}{t} - x_0\right)$$
  Puisque $G$ est un sous-espace vectoriel, on peut réécrire la
  condition:
  $$\forall w\in G, g(x_0)\leq f(w) - p(w - x_0) $$

  La question qu'on se posait revient donc à se demander s'il existe
  un réel satisfaisant les inégalités:
  $$\forall w, z\in G, f(w) - p(w - x_0)\leq g(x_0)\leq p(z + x_0) - f(z)$$

  Il suffit de montrer l'assertion:
  $$\forall w, z\in G, f(w) - p(w - x_0)\leq p(z + x_0) - f(z)$$

  Soient $z, w\in G$. On a $f(w + z)\leq p(w+z)$ par hypothèse
  sur $p$. Or puisque $$p(w + z)= p(w - x_0 + z +x_0)
  \leq p(w - x_0) + p(z + x_0)$$ par sous-additivité de $p$,
  et que $f$ est linéaire, on a l'inégalité.

  On peut donc poser, par exemple,
  $g(x_0)=\inf_{z\in G}p(z+x_0)-f(z)$.

  Pour finir la preuve, on utilise le lemme de Zorn, en considérant
  l'ensemble suivant, avec comme ordre l'inclusion (en considérant
  qu'une fonction $h_1$ prolonge une fonction $h_2$ \ssi{} le
  graphe de $h_1$ est inclus à celui de $h_2$):
  $$\mathcal{M}= \left\{(V, h)\mid G\subseteq V\mbox{ sev. de $E$},
    h: V\to\mathbb{R} \mbox{ linéaire, $h$ prolongement de $f$}\right\}$$
\end{proof}

\subsection{Espaces vectoriels sur $\mathbb{C}$}

\begin{thm}[Hahn-Banach -- Forme analytique (cas complexe)]\label{hb:a2}
Soient $E$ un espace vectoriel sur $\mathbb{C}$, $G$ un sous-espace
vectoriel de $E$, $f: G \to \mathbb{C}$ une forme linéaire
et $p: E\to \left[0,+\infty\right[ $ une application vérifiant
les hypothèses suivantes:
\begin{itemize}
\item $\forall\lambda \in\mathbb{C}, \forall x \in E, p(\lambda x)=|\lambda| p(x)$
\item $\forall x,y \in E, p(x+y)\leq p(x)+p(y)$
  (p est dite sous-additive)
\item$\forall x\in G, |f(x)|\leq p(x)$
\end{itemize}
Alors il existe $g:E\to\mathbb{C}$ forme linéaire étendant $f$ et telle
que $\forall x \in E, |g(x)|\leq p(x)$.
\end{thm}

\begin{proof}
  Remarquez que quel que soit l'élément $x\in G$ considéré,
  on a les égalités suivantes:
  \begin{IEEEeqnarray*}{rCl}
    i f(x) &=& - \Im(f)(x) + i\Re(f)(x)\\
    f(ix) &=& \Re(f)(ix) + i \Im(f)(ix)
  \end{IEEEeqnarray*}

  Par définition de l'égalité de deux nombres complexes
  (leurs parties réelles et imaginaires doivent être
  égales),
  on en déduit que $\Re(f)(ix) = -\Im(f)(x)$ et que
  $\Im(f)(ix) = \Re(f)(x)$. On peut donc exprimer $f$
  uniquement à l'aide de $\Re(f)$ de la manière suivante:
  $$f(x) = \Re(f)(x) - i \Re(f)(ix)$$

  L'application $\Re(f):G \to\mathbb{R}$ est une application
  $\mathbb{R}$-linéaire. Par le théorème de Hahn-Banach, cas
  réel, elle s'étend en une fonction $\mathbb{R}$-linéaire
  $h:E\to\mathbb{R}$.

  On pose $g$ l'application définie par
  $g(x)=h(x)- i h(ix)$ pour tout $x$ dans $E$.
  Donc $h$ correspond à la partie réelle de $g$.
  Il est aisé de vérifier qu'elle est $\mathbb{C}$-linéaire
  par calcul.

  Il reste à montrer que $g$ vérifie bien $|g|\leq p$.
  Soit $x\in E$. On a:
  \begin{IEEEeqnarray*}{rClr}
    |g(x)| & = & e^{-i\arg(g(x))} g(x) & \\
    & = & g(e^{-i\arg(g(x))}x) & \quad (\mbox{$g$ linéaire}) \\
    & = & \Re(g)(e^{-i\arg(g(x))}x) & \\
    & = & h(e^{-i\arg(g(x))}x) & \\
    & \leq & p(e^{-i\arg(g(x))}x) = p(x) & (\mbox{Thm. \ref{hb:a1}})
  \end{IEEEeqnarray*}

\end{proof}

\subsection{Espaces vectoriels normés}

\begin{thm}[Théorème de Hahn-Banach -- Forme analytique]\label{hb:a3}
  Soient $(E, \|.\|)$ un espace vectoriel normé sur $\mathbb{K}
  =\mathbb{R}$ ou $\mathbb{C}$, $G$ un sous-espace vectoriel
  de $E$ et $f: G\to\mathbb{K}$ linéaire continue.

  Alors il existe une fonction $g:E\to\mathbb{K}$ linéaire
  continue prolongeant $f$ telle que $\|f\| = \|g\|$.
\end{thm}

\begin{proof}
  On prend pour fonction $p: E\to\left[0, +\infty\right[$ la fonction
  $x\mapsto \|f\|\cdot\|x\|$. Elle vérifie toutes les hypothèses
  des théorèmes \ref{hb:a1} et \ref{hb:a2}.

  On peut donc prolonger $f$ en une forme linéaire $g$ telle que
  $|g(x)|\leq p(x)$ quel que soit $x\in E$. En particulier, cela
  implique que $g$ est continue car bornée sur la boule unité
  par la norme de $f$.

  Pour conclure il nous reste à montrer l'inégalité réciproque.
  Or pour tout $x\in G$ dans la boule unité, on a
  $|f(x)| = |g(x)|\leq \|g\|$, ce qui fournit l'inégalité
  désirée.
\end{proof}

\begin{rem}
  \textcolor{red}{
  Remarquez que le théorème n'exige \emph{pas} la complétude
  de l'espace vectoriel normé considéré.
}
\end{rem}


\subsection{Hahn-Banach et dualité}
Prouvons plusieurs corollaires des formes analytiques
du théorème de Hahn-Banach liés à la dualité.  Ils
constituent de petits exercices d'application des
théorèmes.

On considère un corps $\mathbb{K}$ qui est soit
celui des réels ou des complexes.

\begin{cor}\label{hb:a:cor1}
  Soient $(E, \|.\|)$ un espace vectoriel normé et
  $x_0\in E\setminus\{0\}$.

  Alors il existe $x^*\in E^*$, tel que $x^*(x_0)=1$
  et $\|x^*\|= \frac{1}{\|x_0\|}$.
\end{cor}

\begin{proof}
  Soit $G = \mathbb{K}x_0$. On considère l'application
  linéaire continue $f: G\to\mathbb{K}:tx_0\mapsto t$
  (la continuité est assurée car l'application est
  définie sur un espace de dimension 1).

  Calculons la norme de $f$:
  $$\|f\|=\sup_{\substack{t\in\mathbb{K}\\ \|tx_0\|\leq 1}}|f(tx_0)|=
  \sup_{\substack{t\in\mathbb{K}\\ |t|\leq\frac{1}{\|x_0\|}}}|t| =
  \frac{1}{\|x_0\|}$$

  Pour conclure, il suffit d'utiliser le théorème \ref{hb:a3}.
\end{proof}

\begin{cor}\label{hb:a:cor2}
  Soient $(E, \|.\|)$ un espace vectoriel normé et
  $x_0\in E\setminus\{0\}$.

  Alors il existe $x^*\in E^*$, tel que $x^*(x_0)=\|x_0\|$
  et $\|x^*\|= 1$.
\end{cor}
\begin{proof}
  Soit $y^*$ fournie par le corollaire \ref{hb:a:cor1}.
  Il suffit de prendre $x^*= \|x_0\|y^*$.
\end{proof}

\begin{rem}\label{hb:a:norme}
  Soit $(E, \|.\|)$ un espace vectoriel normé. Soit $x\in E$.

  Quelle que soit la forme considérée $x^*\in E^*$, on a toujours
  que $|x^*(x)|\leq \|x^*\|\cdot \|x\|$.

  Par le corollaire \ref{hb:a:cor2}, on obtient l'égalité
  $$\|x\|=\max_{\substack{\|x^*\|\leq 1 \\ x^*\in E^*}}|x^*(x)|$$
\end{rem}

\begin{cor}\label{hb:a:cor4}
  Soient $(E, \|.\|)$ un espace vectoriel normé, $F$
  un sous-espace vectoriel de $E$ qui n'est pas
  dense dans $E$ et $x_0\in (E\setminus\mathrm{adh}(F))$.

  Il existe $x^*\in E^*$ tel que $F\subseteq \mathrm{Ker}(x^*)$,
  $x^*(x_0)=1$ et $\|x^*\|=\frac{1}{d}$ où $d = d(x_0, F)$.
\end{cor}

\begin{proof}
  Soit $f: F\oplus \mathbb{K}x_0\to\mathbb{K}:y+tx_0\mapsto t$.
  Cette application est linéaire et on a $F\subseteq \mathrm{Ker}(f)$
  et $f(x_0)=1$.

  Rappelons que $d\neq 0$ car $x_0\notin \mathrm{adh}(F)$.

  Montrons que $f$ est continue.
  Soit $y + tx_0\in F\oplus \mathbb{K}x_0$ tel que
  $\|y + t x_0\|\leq 1$. On a, pout $t\neq 0$:
  $$\|y + tx_0\| = |t|\cdot \left\|x_0 - \left(-\frac{y}{t}\right)\right\|\geq d\cdot |t|$$
  D'où $|f(y + t x_0)| = |t|\leq \frac{1}{d}$. Ceci montre que $f$
  est continue.

  En utilisant le théorème \ref{hb:a3}, on peut étendre $f$
  en une forme linéaire continue $x^*$ sur $E$ de même norme.
  Pour achever la preuve de l'assertion, il faut montrer l'inégalité
  $\|f\|\geq \frac{1}{d}$.

  Soit $\varepsilon >0$. Il existe $y\in F$ tel que
  $|d-\|y-x_0\||<\varepsilon$ (définition d'infimum)
  . Cela implique $\| y - x_0\|\leq d + \varepsilon$.
  On a également l'inégalité
  $$\|f\|\geq \left|f\left(\frac{y - x_0}{\| y - x_0\|}\right)\right|
  =\frac{1}{\| y - x_0\|}
  \geq \frac{1}{d+\varepsilon}$$
  On conclut en faisant tendre $\varepsilon$ vers 0.
\end{proof}

Pour conclure cette section, les deux exercices suivants sont laissés:
\begin{exo}\label{hb:a:exo1}
  Soit $(E, \|.\|)$ un espace vectoriel normé sur $\mathbb{K}$.
  Soit $F$ un sous-espace vectoriel de $E$. Montrer les équivalences
  suivantes:
  \begin{enumerate}
  \item ($x_0\in\mathrm{adh}(F)$) \ssi{} ($\forall x^*\in E^*,
    F\subseteq \mathrm{Ker}(x^*)\implies x^*(x_0) = 0$)
  \item ($\mathrm{adh}(F) = E$)  \ssi{}  ($\forall x^*\in E^*,
    F\subseteq \mathrm{Ker}(x^*)\implies x^* = 0$)
  \end{enumerate}
\end{exo}

Il existe une notation pour l'ensemble des formes linéaires
s'annulant sur un sous-espace vectoriel donné. Il est donc
possible de réécrire la question ci-dessus de manière plus
concise à l'aide de cette notation.

\begin{df}
  Soit $(E, \|.\|)$ un espace vectoriel normé sur $\mathbb{K}$.
  Soit $F$ un sous-espace vectoriel de $E$. On appelle
  annulateur de $F$ le sous-espace vectoriel suivant
  de $E^*$:
  $$F^\perp = \left\{x^*\in E^*\mid F\subseteq \mathrm{Ker}(x^*)\right\}$$
\end{df}

\subsection{Bidualité}
Soit $(E, \|.\|)$ un espace vectoriel normé sur $\mathbb{K}$.
Il est possible de considérer les formes linéaires définies
sur le dual de $E$, étant donné qu'il s'agit d'un espace vectoriel
normé. On appelle cet espace le bidual de $E$, et
on le note $E^{**}$.

\begin{df}[Injection canonique]
  On appelle injection canonique l'application
  \begin{IEEEeqnarray*}{rCrCl}
    i & : & E & \to & E^{**} \\
    & & x & \mapsto & i(x)
  \end{IEEEeqnarray*}
  où pour tout $x\in E$, $i(x)$ est l'application définie par
   \begin{IEEEeqnarray*}{rCrCl}
      i(x) & : & E & \to & \mathbb{K} \\
      & & x^* & \mapsto & i(x)(x^*) = x^*(x)
    \end{IEEEeqnarray*}
\end{df}

Il est simple de vérifier que l'injection canonique est
bien définie (c'est-à-dire qu'elle est bien à image dans
$E^{**}$). De plus, l'injection canonique a les propriétés
suivantes:

\begin{prop}
  L'injection canonique est linéaire, continue et préserve
  la norme.
\end{prop}

\begin{proof}
  La linéarité est claire, car le dual est un espace
  d'applications linéaires, ce qui fournit le résultat.

  Quant à la préservation des normes, il suffit de constater
  les égalités suivantes (par le corollaire \ref{hb:a:cor2}):

  $$\|i(x)\|=\sup_{\substack{\|x^*\|\leq1\\ x^*\in E^*}}|x^*(x)|
  = \|x\|$$
  Par cet argument, on a $\|i\| = 1$, ce qui implique
  que $i$ est continue.
\end{proof}

Les espaces qui s'identifient à leur bidual via l'injection
canonique sont dits réflexifs. Il existe des espaces qui sont
isomorphes à leur bidual, mais pas via l'injection canonique;
ces espaces sont appelés espaces de James.

\begin{prop}
  Si $E$ est réflexif, toute forme $x^*\in E^*$ atteint
  sa norme.
\end{prop}

\begin{proof}
  Soit $x^*\in E^*$. Il existe $x^{**}\in E^{**}$ de
  norme 1
  telle que $x^{**}(x^*)=\|x^*\|$, par le corollaire
  \ref{hb:a:cor2}. Par sujectivité de l'injection
  canonique, il existe $x\in E$ tel que $i(x)=x^{**}$.
  Il s'ensuit que $x^*$ atteint sa norme en $x$.
\end{proof}

\textbf{Remarque}: la réciproque est également vraie,
mais nous n'allons pas nous en préoccuper.

Dans le cas où l'espace n'est pas réflexif, il
existe donc des formes linéaires continues n'atteignant
pas leur norme. Donnons-en un:
\begin{ex}
  On considère $E = (c_0, \|.\|_\infty)$. Alors
  son dual $E^*$ s'identifie à $\ell^1$. Soit
  la suite $x = (\frac{1}{2^n})_{n\in \mathbb{N}}$
  qui est bien élément de $\ell^1$. Alors
  l'application suivante est bien élément de $(c_0)^*$:
  $$x^*:c_0\to\mathbb{K}:(y_n)_{n\in\mathbb{N}}\mapsto \sum_{n=0}^\infty x_ny_n$$

  Montrons que $x^*$ n'atteint pas sa norme. Soit $y = (y_n)_{n\in\mathbb{N}}$
  un élément de la boule unité de $c_0$. Par définition, il existe
  $N$ un naturel tel que pour tout $n > N$, $|y_n|\leq \frac{1}{2}$. On a:
  $$|x^*(y)| \leq \sum_{n=0}^\infty \frac{|y_n|}{2^n} =
  \sum_{n=0}^{N} \frac{|y_n|}{2^n} + \sum_{n=N+1}^\infty \frac{|y_n|}{2^n}
  \leq \sum_{n=0}^{N} \frac{1}{2^n} + \sum_{n=N+1}^\infty \frac{1}{2^{n+1}}
  \leq 2 - \frac{1}{2^{N+1}}<2$$
\end{ex}

Montrons que tous les espaces de Hilbert sont réflexifs.
Pour ce faire, rappelons le théorème de représentation
de Riesz-Fréchet:
\begin{thm}[Théorème de représentation de Riesz-Fréchet]
  Soit $H$ un espace de Hilbert. Soit $x^*$ élément du dual de $H$.
  Il existe un unique élément $x\in H$ tel que pour tout $y\in H$,
  $x^*(y) = \langle y, x\rangle$.
\end{thm}

\textbf{Remarque}: Soit $H$ un espace de Hilbert.
L'application $H\to H^*: y \mapsto \langle$ $.$ $, y\rangle$ est
bijective (par le théorème de Riesz-Fréchet), une isométrie
et \emph{antilinéaire} (exercice).

Nous sommes maintenant armés pour montrer que tous les espaces
de Hilbert sont réflexifs.

\begin{thm}
  Tout espace de Hilbert $H$ est réflexif.
\end{thm}

\begin{proof}
  Munissons le bidual d'une structure d'espace de Hilbert.
  Pour ce faire, faisons d'abord de même pour le dual.

  Etant donné deux formes linéaires continues $u^*, v^*$ sur $H$,
  il existe $u, v$ deux éléments de $H$ leur correspondant (via
  le théorème de représentation de Riesz-Fréchet). On pose
  alors:
  $$\langle u^*, v^*\rangle = \langle v, u \rangle$$

  Il s'agit bien d'un produit scalaire; les détails sont
  laissés à titre d'exercice (l'échange est important pour
  assurer la linéarité par rapport à la première composante).
  Ce produit scalaire engendre la même norme que la norme
  opérateur
  (puisque les représentants sont de même norme que
  les formes considérées). La complétude de $H^*$
  est assurée par la proposition \ref{lin:cpl:imp}.

  De la même manière, on munit le bidual de $H$ d'un
  produit scalaire, pour tous $u^{**}, v^{**}$ du bidual,
  soient $u^*, v^*$ leur représentants du dual, on pose:
    $$\langle u^{**}, v^{**}\rangle = \langle v^*, u^* \rangle$$

  Soit $y^{**}\in H^{**}$. Montrons qu'il existe un élément
  de $y$ de $H$ tel que $y^{**}$ est l'image de $y$ par
  l'injection canonique.

  On considère $y$ le représentant de la forme linéaire continue
  $y^*$, elle-même représentante de $y^{**}$. Soient $x^*$ une forme
  linéaire continue et $x$ son représentant dans $H$. Alors:
  $$i(y)(x^*) = x^*(y) = \langle y, x\rangle$$
  et
  $$y^{**}(x^*) = \langle x^*, y^{*} \rangle = \langle y, x\rangle$$

  Les deux correspondent ce qui  conclut la preuve.

\end{proof}

Vous pouvez consulter une autre preuve dans le document
\cite[p.~49]{refl:hilbert}.

\section{Formes géométriques}
\subsection{Hyperplans}
Soit $E$ un espace vectoriel sur $\mathbb{K}$
(le corps des réels ou des complexes).

\begin{df}
  Un sous-espace vectoriel $H$ de $E$ est dit hyperplan
  vectoriel s'il existe $e\in E$ non nul tel que
  $E = H\oplus \mathbb{K}e$.
\end{df}

Il existe une formulation équivalente en termes de formes linéaires:
\begin{prop}
  Soit $H$ un sous-espace vectoriel de $H$.
  Il s'agit d'un hyperplan vectoriel \ssi{} il existe une
  forme linéaire $f$ non identiquement nulle dont $H$ est
  le noyau.
\end{prop}

\begin{proof}
  Supposons que $H$ est un hyperplan vectoriel. Alors tout
  vecteur de $E$ s'écrit de manière unique sous la forme
  $x + te$ avec $t\in\mathbb{K}$ et $x\in H$.
  Il suffit de poser $f(x + te) = t$ pour avoir l'affirmation.

  Réciproquement supposons que $H$ est le noyau d'une application
  linéaire $f$ non nulle. Soit $e\in E$ un vecteur n'appartenant pas
  au noyau de $f$. Montrons que pour tout $x$ dans $E$, il existe
  $\lambda\in\mathbb{K}$ tel que $x-\lambda e\in H$;
  ceci impliquera $E = H + \mathbb{K}e$. Il suffit de prendre
  $\lambda = \frac{f(x)}{f(e)}$. On a bien $x-\lambda e\in H$.

  Il faut maintenant montrer l'unicité de cette écriture. Soient
  $a, b\in\mathbb{K}$, $u, v \in H$ tels que $x = u + ae = v+be$.
  En appliquant $f$, on obtient $af(e) = bf(e)$, d'où $a = b$
  et $u = v$.
\end{proof}

En plus des hyperplans vectoriels, on définit également les
hyperplans dits affines. Il s'agit de translatés d'hyperplans
vectoriels.
\begin{df}
  Un sous-ensemble $H$ de $E$ est dit hyperplan affine s'il
  existe $f$ une forme linéaire sur $E$, un scalaire $\alpha$
  tels que $H = f^{-1}(\{\alpha\})$.
\end{df}

Supposons désormais que $(E, \|.\|)$ est un espace vectoriel
normé. On a le résultat suivant, qui lie continuité d'une
forme linéaire et si le noyau est fermé ou non.

\begin{prop}
  Soit $f: E \to\mathbb{K}$ une forme linéaire
  non identiquement nulle.

  Son noyau est fermé si et seulement si $f$ est continue.
\end{prop}

\textbf{Remarque}: l'application nulle est continue et son
noyau est fermé. Le cas est écarté de la proposition car
il est trivial.

\begin{proof}
  Si l'application est supposée continue, alors son noyau est
  fermé car il s'agit de l'image réciproque par $f$ du singleton
  $\{0\}$ qui est fermé.

  Réciproquement, supposons son noyau fermé. \'{E}tant donné que
  $f$ est non nulle, il existe $x_0$ dans le complémentaire du
  noyau, qui est ouvert (par hypothèse). On peut supposer $f(x_0)=1$
  (en normalisant le vecteur).

  Soit $r>0$ tel que $B(x_0, r)\subseteq E\setminus\mathrm{Ker}(f)$.
  On a donc $\forall z\in B(0, 1)$, $f(x_0 + rz )\neq 0$. Par
  linéarité de $f$, on obtient
  $\forall z\in B(0, 1)$, $f(z)\neq r^{-1}$.

  Pour conclure, il suffit de montrer que $|f(z)|< r^{-1}$ pour tout
  élément $z$ de la boule unité de $E$, une application linéaire
  étant continue si et seulement si elle est bornée sur la boule
  unité. Montrons que le
  cas complémentaire est impossible par l'absurde.

  S'il existe un élément $z$ de la boule unité de $E$ tel que
  $|f(z)|\geq r^{-1}$, alors il existe $y$ dans la boule unité
  tel que son image par $f$ est réelle positive et $f(y)\geq r^{-1}$
  (prendre $y = e^{-i\arg(f(z))}z$). En multipliant $y$ par une
  constante appropriée, on obtient un élément de la boule
  unité d'image $r^{-1}$ ce qui constitue une contradiction.

\end{proof}

\subsection{Introduction aux formes géométriques du
  théorème de Hahn-Banach}

Tout au long de cette section on considère un espace vectoriel
normé $(E, \|.\|)$ sur $\mathbb{R}$.

Le but de cette section est d'introduire les bases nécessaires
pour prouver les formes géométriques du théorème de Hahn-Banach.
Le théorème affirme qu'étant donnés deux convexes de $E$
vérifiant certaines hypothèses, il est
possible de les séparer ces derniers par un hyperplan, comme
illustré à la figure \ref{sep:ill}. Plus formellement:

% Figure pour illustrer la séparation au sens large de deux convexes
% par un hyperplan
\begin{figure}[!h]
  \begin{center}
    \caption{Séparation de deux convexes par des hyperplans dans $\mathbb{R}^2$}%
    \label{sep:ill}
    \begin{tikzpicture}
      \begin{scope}[rotate=315]
      \draw (1, -1) -- (1, 5.5) node[right] {D};
      \draw [red] (0 , 2) ellipse (1 cm and 2 cm) node[below left] {A};
      \draw [blue] (2 , 2) ellipse (1 cm and 2 cm) node[below right] {B};
    \end{scope}
    \end{tikzpicture}
\end{center}
\end{figure}

\begin{df}
  Soient $A$, $B$ deux sous-ensembles disjoints de $E$,
  $f:E\to\mathbb{R}$ une forme linéaire continue, $\alpha$ un
  réel et $H = f^{-1}(\{\alpha\})$ un hyperplan affine.

  $H$ sépare $A$ et $B$ au sens large si $\forall x\in A,
  f(x)\leq \alpha$ et $\forall x\in B, f(x)\geq \alpha$.

  $H$ sépare $A$ et $B$ au sens strict s'il existe
  $\varepsilon > 0$ tel que $\forall x\in A, f(x)\leq \alpha
  - \varepsilon$ et $\forall x\in B, f(x)\geq \alpha + \varepsilon$.
\end{df}

Pour parvenir à nos fins, on introduit une fonction appelée jauge
d'un convexe $C$:

\begin{df}
  Soit $C\subseteq E$ un convexe contenant 0 et ouvert.
  Pour tout $x$ dans $E$, on définit la jauge de $C$ en $x$ comme
  suit:
  $$j_C(x) = \inf\left\{\alpha > 0\mid x\in\alpha C\right\} =
  \inf\left\{\alpha > 0\mid \alpha^{-1}x\in C\right\}$$
\end{df}


Pour se familiariser avec cette fonction, vous êtes invité
à effectuer les exercices suivants:
\begin{exo}
  On considère le convexe $C = \left]\frac{-1}{2}, 2\right[$
  de $\mathbb{R}$.
  Effectuez les calculs suivants:
  \begin{IEEEeqnarray*}{rClCrCl}
    j_C\left(\frac{1}{2}\right) & = & \fbox{\phantom{AAAAAA}}
    &\qquad & j_C\left(-1\right) & = & \fbox{\phantom{AAAAAA}} \\
    j_C\left(\frac{-1}{2}\right) & = & \fbox{\phantom{AAAAAA}}
    &\qquad & j_C\left(2\right) & = & \fbox{\phantom{AAAAAA}} \\
  \end{IEEEeqnarray*}
\end{exo}

\begin{exo} \label{hb:g:j2}
  Soit $C$ un ouvert convexe de $E$ contenant $0$. Soit $x$ un élément
  de $E$. Montrer que quel que soit $\varepsilon > 0$,
  $$\frac{1}{j_C(x)+\varepsilon}x\in C$$
\end{exo}

Pour prouver les formes géométriques du théorème de Hahn-Banach,
nous allons utiliser le résultat suivant.

\begin{prop}\label{hb:g:l}
Soient $C$ un ouvert convexe non vide de $E$ et $x_0\in E\setminus
 C$. Alors il existe un hyperplan fermé qui sépare $\{x_0\}$ et $C$
au sens large, c'est-à-dire:
$$\exists x^{*}\in E^*, \forall x \in C, x^*(x) < x^{*}(x_0)$$
\end{prop}

Prouvons plusieurs propriétés de la jauge d'un convexe qui serviront
dans la preuve de la proposition \ref{hb:g:l}.

\begin{lem} \label{lem:jc}
  Soit $C$ un ouvert convexe de $E$ contenant 0. Alors:
  \begin{enumerate}
  \item $\exists M> 0$, $\forall x\in E$, $0\leq j_C(x)\leq M\|x\|$.
    \label{lem:jc1}
  \item $C = \left\{ x\in E \mid j_C(x)<1\right\}$. \label{lem:jc2}
  \item La jauge d'un convexe est positivement homogène. \label{lem:jc3}
  \item L'application $j_C$ est sous-additive. \label{lem:jc4}
  \end{enumerate}
\end{lem}

\begin{proof}
  Montrons chaque point un à un.
  \begin{enumerate}
  \item Soit $r>0$ tel que $B(0, r)\subseteq C$. Posons
    $M = 2r^{-1}$. Soit $x\in E$ non nul. On vérifie facilement
    que $r\frac{x}{2\|x\|}\in B(0, r)\subseteq C$, d'où $x \in
    2r^{-1}\|x\|\cdot C$, ce qui donne le résultat.
  \item Soit $x\in C$. Soit $r> 0$ tel que $B(x, r)\subseteq C$.
    Alors il existe $\varepsilon >0$ tel que $(1+\varepsilon)x\in
    B(x, r)$. D'où $x\in \frac{1}{1+\varepsilon}\cdot C$ ce qui
    montre la première inclusion.

    Réciproquement, soit $x\in E$ tel que $j_C(x)<1$. Il existe
    donc $1 > \varepsilon > 0$ tel que $x\in \varepsilon \cdot C$.
    Par convexité de $C$, le segment joignant $0$ à
    $\frac{1}{\varepsilon}x$ est inclus à $C$; puisque
    ce dernier contient $x$, on a l'inclusion désirée.
  \item Soient $x$ un élément de $E$ et $\alpha$ un réel strictement
    positif. Pour tout $\beta > 0$, $\alpha x\in \beta\cdot C$, \ssi{}
    $x\in \alpha^{-1}\beta C$. D'où $j_C(\alpha x) = \alpha j_C(x)$ en
    passant à l'infimum sur $\beta$.
  \item Soient $x$, $y$  deux éléments de $E$. Soit
    $\varepsilon>0$. On a:%\in\left]0, 1 \right[$. Alors
    $\frac{1}{j_C(x)+\varepsilon}x\in C$ et
    $\frac{1}{j_C(y)+\varepsilon}y\in C$ (cf. exercice résolu \ref{hb:g:j2}).
    Par convexité de $C$, on a pour tout $t\in [0, 1]$:
    $$t\frac{1}{j_C(x)+\varepsilon}x +
    (1-t)\frac{1}{j_C(y)+\varepsilon}y\in C$$
    En prenant $\displaystyle{t = \frac{j_C(x)+\varepsilon}
      {j_C(x) +j_C(y)+2\varepsilon}}$, on obtient
    $$\frac{x+y}{j_C(x) +j_C(y)+2\varepsilon}\in C $$

    Par le point \ref{lem:jc2} et la positive homogénéité de la jauge d'un convexe,
    on obtient $j(x + y) < j_C(x)+j_C(y)+2\varepsilon$. On conclut
    en prenant $\varepsilon \to 0$.
  \end{enumerate}
\end{proof}

Nous pouvons donc procéder à la preuve de la proposition \ref{hb:g:l}
en utilisant ces propriétés.

\begin{proof}
Sans perte de généralité, on peut supposer $0\in C$; si ce n'est pas le cas,
on considère $y_0\in C$ et \og on translate \fg, c'est-à-dire on prouve la
propriété pour le convexe $C' = C - y_0$ et le point $x_0' = x_0 - y_0$ et on utilise
la linéarité de la forme obtenue pour avoir le résultat.

On pose $G = \mathbb{R}x_0$ et on considère la forme linéaire définie sur $G$
par $\forall t\in \mathbb{R}$, $f(tx_0) = t$. Alors pour tout $t$ réel, on a
$f(tx_0)\leq j_C(tx_0)$; si $t$ est négatif, c'est clair, et si $t$ est positif,
cela revient à montrer (puisque $j_C$ est positivement homogène)
que $1\leq j_C(x)$, ce qui est vrai car $x\notin C$.

Par la forme analytique du théorème de Hahn-Banach (théorème \ref{hb:a1}),
$f$ se prolonge en une forme linéaire $g$ sur $E$ vérifiant $\forall x \in E$,
$g(x)\leq j_C(x)$.
De plus, comme il existe une constante $M >0$ telle que tout élément
$x$ de $E$ vérifie $j_C(x) \leq M\|x\|$, la continuité de $g$ est assurée
sur $E$.

On considère l'hyperplan $H = g^{-1}(\{1\})$. Montrons qu'il sépare bien $C$ et
le singleton $x_0$. C'est immédiat car $g(x_0) = 1$ et tout élément $y$ de $C$
est tel que $g(y)\leq j_C(y) < 1$.
\end{proof}

Le corollaire suivant est laissé au lecteur à titre d'exercice. Vous êtes
invités à le montrer de plusieurs manières; en utilisant la proposition
précédente ou un des corollaires des formes analytiques.

\begin{exo}
  Montrer que $E^*$ sépare les points de $E$, c'est-à-dire si $x$, $y$
  sont des éléments de $E$, il existe $x^*$ une forme linéaire continue
  sur $E$ telle que $x^*(x)\neq x^*(y)$
\end{exo}

En particulier deux éléments de $E$ sont égaux \ssi{} leur image
par toute forme linéaire continue coïncide.

\subsection{Théorème de Hahn-Banach: Formes géométriques}
Tout au long de cette section, on considère un espace vectoriel normé
$(E, \|.\|)$ sur $\mathbb{R}$.
\begin{thm}[Théorème de Hahn-Banach -- Première forme géométrique] \label{hb:g1}
  Soient $A$ un ouvert convexe non vide de $E$, $B$ un convexe non vide
  de $E$ tels que $A$ et $B$ sont disjoints. Alors il existe un hyperplan
  fermé qui sépare $A$ et $B$ au sens large.
\end{thm}

\begin{proof}
  Soit $C = A - B $. Il s'agit d'un convexe ouvert de $E$. La convexité
  se vérifie par calcul et pour montrer que $C$ est ouvert,
  il suffit de remarquer qu'il s'agit d'une union d'ouverts:

  $$ C = \bigcup_{b\in B} A - b =
  \bigcup_{b\in B} \left\{a - b \mid a\in A\right\}$$

  \'{E}tant donné que $A$ et $B$ sont disjoints, $0\notin C$.
  Par la proposition \ref{hb:g:l}, il existe une forme linéaire
  continue $x^*$ telle que tout élément $x$ de $C$ a une image
  strictement négative. Il s'ensuit que pour tout élément $a$ de
  $A$, pour tout élément $b$ de $B$, $x^*(a - b) < 0$.

  Puisque $B$ est non vide, $\alpha = \sup_{x\in A}x^*(a)$ est fini. L'hyperplan
  $(x^*)^{-1}(\alpha)$ sépare $A$ et $B$ au sens large, ce qui conclut la preuve.
\end{proof}

Il existe une seconde forme géométrique du théorème de Hahn-Banach.

\begin{thm}[Théorème de Hahn-Banach -- Deuxième forme géométrique] \label{hb:g2}
  Soient $A$ un compact convexe non vide de $E$, $B$ un fermé convexe non vide
  de $E$ tels que $A$ et $B$ sont disjoints. Alors il existe un hyperplan
  fermé qui sépare $A$ et $B$ au sens strict.
\end{thm}

Introduisons un lemme utile dans la preuve du théorème

\begin{lem}\label{hb:g:l2}
  Soit $(F, \|.\|)$ un espace vectoriel normé sur $\mathbb{K}$ (où
  $\mathbb{K}= \mathbb{R}$ ou $\mathbb{C}$), $A$, $B$ deux convexes
  non vides disjoints de $F$, $A$ compact et $B$ fermé.

  Alors il existe $\varepsilon > 0$ tel que $A_\varepsilon =
  A + B(0, \varepsilon)$  et $B_\varepsilon = B
  + B(0, \varepsilon)$ sont disjoints.
\end{lem}
\begin{proof}
  Supposons par l'absurde que cela soit faux. Pour tout naturel $n$ non
  nul, il existe $x_n\in A_{1/n}\cap B_{1/n}$. Alors $x_n$ s'écrit comme
  $a_n + z_n$ pour certains $a_n\in A$, $z_n\in B(0, 1/n)$ et comme
  $b_n + w_n$ pour certains $b_n\in B$, $w_n\in B(0, 1/n)$.

  Par compacité séquentielle de $A$, il existe une sous-suite de
  $(a_n)_{n\in\mathbb{N}}$, notons-la $(a_{n_k})_{k\in\mathbb{N}}$, un élément
  $a$ dans $A$ tel que   $(a_{n_k})_{k\in\mathbb{N}}$ converge
  vers $a$. Puisque les suites $(z_n)_{n\in\mathbb{N}}$ et
  $(w_n)_{n\in\mathbb{N}}$ convergent vers $0$, on déduit que
  la sous-suite  $(b_{n_k})_{k\in\mathbb{N}}$ de
  $(b_{n})_{n\in\mathbb{N}}$ converge vers $a$, ce qui implique $a$
  dans $B$ (car $B$ est fermé), ce qui contredit $A$ et $B$ disjoints.
\end{proof}

Prouvons la seconde forme géométrique du théorème de Hahn-Banach à l'aide
du lemme ci-dessus.

\begin{proof}
  Soit $\varepsilon> 0 $ tel que $A_\varepsilon$ et $B_\epsilon$ (au sens du
  lemme \ref{hb:g:l2}) sont disjoints. Puisque ce sont des ensembles ouverts,
  convexes, non vides et disjoints, il existe $x^*$ forme linéaire continue,
  $\alpha$ un nombre réel tel que pour tout $a\in A_\varepsilon$,
  $x^*(a)\leq\alpha$ pour tout $b\in B_\varepsilon$, $x^*(b)\geq \alpha$.
  (par la première forme géométrique).

  On a pour tout $a$ dans $A$, $x^*(a)\leq \alpha - \varepsilon\|x^*\|$ et
  de manière analogue,
  pour tout $b$ dans $B$, $x^*(b)\geq \alpha +\varepsilon \|x^*\|$, ce qui
  conclut la preuve.

\end{proof}

\subsection{Généralisation aux espaces vectoriels complexes}
Soit $(E, \|.\|)$ un espace vectoriel normé sur $\mathbb{C}$. Il est
possible de généraliser les formes géométriques du théorème de Hahn-Banach
aux espaces vectoriels sur le corps des complexes.


Voici une première généralisation de la première forme géométrique.
\begin{thm}[Hahn-Banach -- Première forme géométrique, cas complexe]\label{hb:g:c1}
  Soient $A$ un ouvert convexe non vide de $E$, $B$ un
  convexe non vide de $E$ tels que $A$ et $B$ sont disjoints. Il existe
  $x^*$ une forme linéaire non nulle sur $E$ et $\alpha$ un nombre réel tels que:
  $$A\subseteq \left\{x\in E\mid \Re(x^*)(x) \leq \alpha\right\} \mbox{ et }
  B\subseteq \left\{x\in E\mid \Re(x^*)(x) \geq \alpha\right\} $$
\end{thm}

\begin{proof}
  Par la première forme géométrique (dans le cas réel), il existe une forme continue $f$
  $\mathbb{R}$-linéaire sur $E$ telle que
  $$A\subseteq \left\{x\in E\mid f(x) \leq \alpha\right\} \mbox{ et }
  B\subseteq \left\{x\in E\mid f(x) \geq \alpha\right\} $$

  On conclut en posant $x^*(x) = f(x) - if(ix)$ pour tous $x$ dans $E$.
  La linéarité est à vérifier à titre d'exercice et
  elle est bien continue car: $$|x^*(x)|\leq |f(x)|+|f(ix)|\leq
  \|f\|\|x\|+\|f\|\|ix\| = 2 \|f\|\|x\|$$
\end{proof}

Il est possible d'avoir un résultat similaire en considérant
des modules plutôt que les parties réelles de la forme linéaire,
au coût d'une hypothèse supplémentaire sur un des ensembles.
Introduisons d'abord une définition.

\begin{df}
  Soit $A$ un sous-ensemble de $E$. $A$ est dit équilibré
  (ou $\mathbb{C}$-symétrique) si quel que soit
  l'élément $v$ de $A$ et le complexe $\lambda$ de module 1 considérés, le vecteur
  $\lambda{}v$ est un élément de $A$.
\end{df}

Le résultat est le suivant:

\begin{cor}\label{hb:g:cc1}
  Soient $A$ un ouvert convexe non vide \emph{équilibré} de $E$, $B$ un
  convexe non vide de $E$ tels que $A$ et $B$ sont disjoints. Il existe
  $x^*$ une forme linéaire non nulle sur $E$ et $\alpha$ un nombre réel tels que:
  $$A\subseteq \left\{x\in E\mid |x^*(x)| \leq \alpha\right\} \mbox{ et }
  B\subseteq \left\{x\in E\mid |x^*(x)| \geq \alpha\right\} $$
\end{cor}


\begin{proof}

  Par le théorème \ref{hb:g:c1}, il existe $x^*$ linéaire continue telle que
  $$A\subseteq \left\{x\in E\mid \Re(x^*)(x) \leq \alpha\right\} \mbox{ et }
  B\subseteq \left\{x\in E\mid \Re(x^*)(x) \geq \alpha\right\} $$

  Alors pour tous $b$ dans $B$,
  on a: $|x^*(b)|\geq \Re(x^*)(b) \geq \alpha$.
  Soit $a\in A$. On a (par linéarité de $x^*$ et par $\mathbb{C}$-symétrie
  de $A$):
  $$|x^*(a)| = x^*(e^{-i\arg(x^*(a))}a) =
  \Re(x^*)(e^{-i\arg(x^*(a))}a)\leq \alpha$$
\end{proof}

Adaptons la seconde forme géométrique du théorème de Hahn-Banach.
\begin{thm}[Hahn-Banach -- Deuxième forme géométrique, cas complexe]
  Soient $A$ un compact convexe non vide de $E$, $B$ un fermé
  convexe non vide de $E$ tels que $A$ et $B$ sont disjoints. Il existe
  $x^*$ une forme linéaire non nulle sur $E$, $\alpha$ un nombre réel et
  $\varepsilon > 0$ tels que:
  $$A\subseteq \left\{x\in E\mid \Re(x^*)(x) \leq \alpha-\varepsilon\right\} \mbox{ et }
  B\subseteq \left\{x\in E\mid \Re(x^*)(x) \geq \alpha+\varepsilon\right\} $$
\end{thm}

\begin{proof}
  Par le lemme \ref{hb:g:l2}, il existe $\varepsilon > 0$ tel que $A_\varepsilon$
  et $B_\varepsilon$ (en reprenant les notations du lemme) sont disjoints.

  Alors par le théorème \ref{hb:g:c1}, il existe une forme linéaire continue $x^*$,
  un réel $\alpha$ tel que:
  $$A_\varepsilon\subseteq \left\{x\in E\mid \Re(x^*)(x) \leq \alpha\right\} \mbox{ et }
  B_\varepsilon \subseteq \left\{x\in E\mid \Re(x^*)(x) \geq \alpha\right\} $$

  Alors pour tout $a$ dans $A$, pour tout $z$ dans $B(0, \varepsilon)$,
  $\Re(x^*)(a + z) \leq \alpha$.

  Soit $a\in A$. Par linéarité de $x^*$, la dernière affirmation implique
  que pour tout $z$ dans la boule unité,
  $\Re(x^*)(a) + \varepsilon\Re(x^*)(z) \leq \alpha$.
  Puisque l'affirmation est vraie
  pour tous $z$ dans la boule unité, on peut
  passer au suprémum sur les $z$ et on obtient l'inégalité $\Re(x^*)(a)+
  \varepsilon \|x^*\|\leq \alpha$.
  Avec un calcul similaire, on peut conclure pour $B$.
\end{proof}

\textbf{Remarque}: il est possible d'écrire une version du résultat
utilisant des modules plutôt que des parties réelles dans les inégalités
à la manière du corollaire \ref{hb:g:cc1}. Vous êtes invités à écrire
le résultat et à le montrer à titre d'exercice.

\subsection{Applications}
Nous considérons à nouveau dans cette section un espace vectoriel
normé $(E, \|.\|)$ sur le corps des nombres réels. Nous allons
présenter plusieurs corollaires des théorèmes de Hahn-Banach.

Commencer par présenter une preuve alternative du corollaire \ref{hb:a:cor4},
qui utilise la forme géométrique du théorème de Hahn-Banach.
\begin{cor}
  Soit $F$ un sous-espace vectoriel de $E$ tel que $F$ n'est
  pas dense dans $E$. Alors il existe une forme linéaire non nulle
  continue s'annulant sur $F$.
\end{cor}
\begin{proof}
  Un sous-espace vectoriel est un convexe. Soit $x_0\in E\setminus\mathrm{adh}(F)$.
  Alors $\{x_0\}$ est un compact convexe de $E$ et $\mathrm{adh}(F)$
  est un sous-espace vectoriel (donc convexe) fermé ne contenant pas $x_0$.
  Par la deuxième forme géométrique de Hahn-Banach, il existe une forme
  linéaire $x^*$, un réel $\alpha$ et $\varepsilon >0$
  tels que $x^*(x_0)\geq \alpha + \varepsilon$  et
  pour tout $y\in F$, $x^*(y)\leq \alpha-\varepsilon$. En particulier, $x^*$ est majorée
  sur $F$ par une constante; il est facile de vérifier que $x^*$ s'annule
  sur $F$. On en déduit $\alpha \geq 0$ ce qui nous permet de conclure
  que $x^*(x_0)$ est non nul.
\end{proof}

\textbf{Remarque}: ce résultat a déjà été énoncé (cf. corollaire
\ref{hb:a:cor4}). Il est répété ici pour donner une preuve différente
qui utilise les formes géométriques de Hahn-Banach.

De manière similaire à l'annulateur d'un sous-espace vectoriel, on peut
définir un sous-espace vectoriel similaire du dual.
\begin{df}
  Soit $N$ un sous-espace vectoriel de $E^*$. On définit $N_\perp$
  comme le sous-ensemble de $E$ défini par:
  $$N_\perp = \left\{ x\in E\mid \forall x^*\in N, x^*(x) = 0\right\}$$
\end{df}

Il s'agit d'un sous-espace vectoriel fermé de $E$ : cela se peut montrer
en écrivant l'ensemble comme l'intersection des noyaux des éléments de $N$
qui sont fermés.

\begin{prop}
  Soit $F$ un sous-espace vectoriel de $E$. Montrer l'égalité
  $(F^\perp)_\perp = \mathrm{adh}(F)$.
\end{prop}

\begin{proof}
  On a clairement l'inclusion $F \subseteq (F^\perp)_\perp$, ce qui
  fournit $\mathrm{adh}(F) \subseteq (F^\perp)_\perp$.

  L'inclusion réciproque est fournie par le résultat de l'exercice
  \ref{hb:a:exo1} (la solution de l'exercice est donnée en annexe).
\end{proof}

Maintenant nous avons tous les outils nécessaires pour démontrer la
réciproque de la proposition \ref{lin:cpl:imp}.
\begin{thm}\label{lin:cpl:equ}
  Soient $(F, \|.\|_F)$, $(G, \|.\|_G)$ deux espaces vectoriels normés sur
  sur $\mathbb{K}$ ($\mathbb{R}$ ou $\mathbb{C}$).

  $\mathscr{L}(F, G)$ est un espace de Banach \ssi{} $G$ est un espace
  de Banach.
\end{thm}

\begin{proof}
  Il reste à prouver une implication. Supposons $\mathscr{L}(F, G)$ complet.
  Soit $(y_n)_{n\in\mathbb{N}}$ une suite de Cauchy dans $G$.

  Fixons un élément $x_0$ de la sphère unité de $F$, c'est-à-dire
  $\|x_0\|_F = 1$. Par le corollaire \ref{hb:a:cor2}, il existe
  donc un élément $x^*$ du dual de $F$ de norme 1 et tel que $x^*(x_0) = 1$.

  Pour tout naturel $n$, on définit l'application $f_n: F\to G$
  par $f_n(x) = x^*(x)y_n$ quel que soit $x$ dans $F$;
  il s'agit bien d'une application linéaire
  et continue. Montrons que la suite $(f_n)_{n\in\mathbb{N}}$ est une suite
  de Cauchy. Soient $n$, $m$ deux naturels, on a:
  $$\|f_n-f_m\| =
  \sup_{\substack{x\in F\\\|x\|_F = 1}}\|x^*(x)y_n - x^*(x)y_m\|_G =
  \|y_n - y_m\|_G \sup_{\substack{x\in F\\\|x\|_F = 1}} |x^*(x)| =
  \|y_n - y_m\|_G $$

  Puisque la suite $(y_n)_{n\in\mathbb{N}}$ est de Cauchy, il est immédiat
  par l'égalité précédente que la suite $(f_n)_{n\in\mathbb{N}}$ l'est
  également. Elle converge donc vers une application linéaire et continue
  $f$.

  La convergence au sens de la norme opérateur impliquant la
  convergence ponctuelle, on a donc
  $y_n = f_n(x_0)\xrightarrow[n\to+\infty]{} f(x_0)$, donc
  la suite $(y_n)_{n\in\mathbb{N}}$ est convergente.

\end{proof}

\begin{prop}\label{ker:sub:mult}
  Soient $\varphi$, $\psi$ deux formes linéaires sur $E$ non nulles telles
  que $\mathrm{Ker}(\varphi) \subseteq \mathrm{Ker}(\psi)$. Alors
  il existe $\lambda \in\mathbb{R}$ non nul tel que $\psi = \lambda \varphi$
\end{prop}

\begin{proof}
  Soit l'application $F: E \to \mathbb{R}^2: x\mapsto (\varphi(x), \psi(x))$.
  Alors le vecteur $(1, 0)$ n'est pas dans l'image de $F$. Il existe,
  par la seconde forme géométrique de Hahn-Banach, une forme linéaire
  et continue séparant l'image de $F$ (convexe fermé) et $\{(1, 0)\}$ (convexe
  compact) au sens strict, c'est-à-dire il existe $x^*\in (\mathbb{R}^2)^*$,
  $\alpha\in\mathbb{R}$ et $\varepsilon > 0$ tels que $x^*((1, 0))\geq
  \alpha +\varepsilon$ et tout vecteur $u$ de $E$ vérifie
  $x^*(F(u)) = x^*(\varphi(u), \psi(u))\leq \alpha-\varepsilon$.
  Notons
  $x^*((x, y)) = x_1 x + x_2 y$, pour tout $(x, y)\in\mathbb{R}^2$.

  Puisque $x^*\circ F$ est linéaire et que son image est majorée par une
  constante, on en déduit qu'elle est nulle, c'est-à-dire $x_1 \varphi(u) +
  x_2 \psi(u) = 0$, quel que soit $u$ dans $E$.
  Cela implique également que $x_1 > 0$ (puisque $x_1 \geq
  \alpha + \varepsilon > x^*(F(0_E)) = 0$).
  On en déduit également $x_2\neq 0$ car en considérant $u\in E\setminus
  \mathrm{Ker}(\psi)$, on a $x_2 \psi(u) = -x_1\varphi(u) \neq 0 $.
On conclut en posant $\lambda = -x_1/x_2$.
\end{proof}

\textbf{Remarque}: dans la proposition précédente, la norme sur $E$
n'intervient pas. De plus, nous ne précisons pas de norme sur $\mathbb{R}^2$
étant donné qu'elles sont toutes équivalentes (donc la continuité ne
dépend pas de la norme considérée), d'autant plus que la continuité de $x^*$
n'intervient pas dans l'argument.

\begin{prop}
  Tout hyperplan vectoriel $H$ de $E$ est soit fermé, soit dense.
\end{prop}
\begin{proof}
  Soit $f$ la forme linéaire (non nulle)
  sur $E$ telle que $H = \mathrm{Ker}(f)$.
  Si $H$ est dense on a le résultat (en particulier $f$ n'est pas continue
  car son noyau n'est pas fermé).

  Supposons maintenant que $H$ n'est pas dense dans $E$ et
  montrons que $H$ est fermé.
  Il existe par   le corollaire \ref{hb:a:cor4} une forme linéaire
  et continue $x^*$ non nulle s'annulant sur l'adhérence de $H$.
  Par la proposition \ref{ker:sub:mult}, puisque
  $H\subseteq \mathrm{Ker}(x^*)$, $x^*$ et $f$ sont multiples, donc
  $f$ est continue. On en conclut que $H$ est fermé.
\end{proof}


\begin{prop}
  Soient $x^*$, $y^*\in E^*$ telles que $\|x^*\| = \|y^*\| = 1$ et
  $\varepsilon >0$. Supposons que $|x^*(x)| < \varepsilon$ pour tout
  $x$ dans l'intersection du noyau de $y^*$ et de la boule unité de $E$.

  Alors on a $\| x^* + y^*\| < 2\varepsilon$ ou
  $\| x^* - y^*\| < 2\varepsilon$
\end{prop}

\begin{proof}
  \textbf{\'{E}tape 1}:
  montrons tout d'abord qu'il existe $z^*\in E^*$ et un réel $\alpha$ tels
  que $\|z^*\|\leq \varepsilon$ et $x^* - z^* = \alpha y^*$.

  Considérons la restriction de $x^*$ au noyau de $y^*$: par le
  théorème \ref{hb:a3}, cette restriction s'étend en une forme linéaire
  continue $z^*$ de même norme.
  Or puisque par hypothèse $|x^*(x)| < \varepsilon$ pour tout $x\in B(0, 1)
  \cap \mathrm{Ker}(y^*)$, on a $\|z^*\|\leq \varepsilon$. Par la proposition
  \ref{ker:sub:mult}, puisque $\mathrm{Ker}(x^* - z^*) \supseteq
  \mathrm{Ker}(y^*)$, il existe $\alpha \in\mathbb{R}$ tel que $x^*-z^* =
  \alpha y^*$.

  \textbf{\'{E}tape 2}: montrons ensuite que $|1- |\alpha||\leq \varepsilon$.

  % Quel que soit $x$ dans la boule unité de $E$, on a $|x^*(x) -
  % \alpha y^*(x)| = |z^*(x)|\leq \varepsilon$.

  Soit $x$ un élément de boule unité de $E$. On a:
  $$|z^*(x)| = |x^*(x)-\alpha y^*(x)| \geq |x^*(x)| - |\alpha||y^*(x)|
  \geq |x^*(x)| - |\alpha| \|y^*\| = |x^*(x)| - |\alpha|$$

  D'où $|x^*(x)| \leq |z^*(x)| + |\alpha|\leq \varepsilon + |\alpha|$.
  Comme cette inégalité est vraie pour tout élément de la boule unité de $E$,
  on peut passer au suprémum et on obtient $1 = \|x^*\|\leq \varepsilon
  + |\alpha|$, ou encore $ 1 - |\alpha| \leq \varepsilon$.

  On a également:
   $$|z^*(x)| = |x^*(x)-\alpha y^*(x)| \geq |\alpha||y^*(x)| -|x^*(x)|
   \geq |\alpha| |y^*(x)| - \|x^*\| = |\alpha||y^*(x)| - 1$$

   D'où $|\alpha| |y^*(x)| \leq 1 + |z^*(x)|\leq 1 + \varepsilon$. De manière
   analogue à ci-dessus, on en déduit $|\alpha| \leq 1 + \varepsilon$,
   c'est-à-dire $|\alpha| - 1\leq \varepsilon$.

   Ceci termine l'étape 2.

   \textbf{\'{E}tape 3}: conclusion.

   Soit $\eta \in\{-1, 1\}$. On a, par l'inégalité triangulaire:
   $$\|x^*+ \eta y^*\| \leq \|x^* - \alpha y^*\| + \|\alpha y^* + \eta y^*\|
   \leq \varepsilon + \|\alpha y^* + \eta y^*\|$$

   Pour conclure il reste à choisir $\eta$ (ce choix revient à se
   placer dans un des cas de l'alternative à montrer) tel que
   $|\alpha + \eta| \|y^*\|= |\alpha + \eta| \leq \varepsilon$.
   Si $\alpha$ est positif, on prend  $\eta = -1$, sinon, on choisit
   $\eta = 1$ et on conclut par l'étape 2.
\end{proof}

%%% Local Variables:
%%% mode: latex
%%% TeX-master: "../analyse3"
%%% End:


\chapter{Théorème de Baire}
\section{Un petit mot historique}
\textbf{Cantor} (1845-1918): théorie des ensembles de nombres réels;
tout intervalle de $\mathbb{R}$ est non-dénombrable.

\textbf{Borel} (1871-1956): théorie de la mesure; tout intervalle
de $\mathbb{R}$ est de mesure non nulle; notion de propriété vraie
presque partout: une propriété $P$ est vraie presque partout si
elle est vérifiée l'ensemble complet hormis un sous-ensemble de mesure
nulle.

\textbf{Baire} (1874-1932): point de vue topologique; tout intervalle
n'est pas n'est pas de première catégorie; une propriété $P$ est dite
vraie presque partout au sens de Baire si elle est vérifiée sur tout
l'ensemble hormis un sous-ensemble de première catégorie.

\section{Définitions}

Soit $(X, \mathcal{T})$ un espace topologique (fixé tout au long de cette
section).

\begin{df}
  Un sous-ensemble $A$ de $X$ est dit nulle part dense s'il vérifie
  $$\mathrm{int}\left(\mathrm{adh}(A)\right) = \varnothing$$

  De manière équivalente, $A$ est dit nulle part dense si le complémentaire
  de son adhérence est dense dans $X$.
\end{df}

Par exemple, dans $\mathbb{R}$ muni de sa topologie usuelle, l'ensemble
vide, les singletons, l'ensemble des nombres naturels et l'ensemble des
nombres entiers sont tous nulle part dense.

\begin{prop}
  L'ensemble des parties nulle part denses de $X$ est héréditaire et
  est stable par union finie  et par fermeture.
\end{prop}

\begin{rem}
  Montrer qu'un sous-ensemble $C$ de $X$ est dense revient à
  montrer que $\mathrm{adh}(C) = X$, c'est-à-dire
  $\forall x\in X$, $\forall O_x\in\mathcal{T}$ , $x\in O_x\implies O_x\cap C\neq
  \varnothing$.

  Cela revient au même que de montrer que tout ouvert $O$ de $X$ non vide,
  $O\cap C\neq \varnothing$.
\end{rem}

\begin{proof}
  Il est clair que cette classe est stable par fermeture. L'hérédité est
  facile à démontrer étant donné que le passage à l'intérieur et à l'adhérence
  préserve les inclusions.

  Il reste donc à montrer que cette classe est stable par union finie.
  Il suffit de le montrer pour deux ensembles (et d'itérer lorsque plus de
  deux ensembles interviennent)

  Soient
  $A$, $B$ deux sous-ensembles nulle part denses de $X$. Alors le complémentaire
  de l'adhérence de $A$ est dense dans $X$, c'est-à-dire pour tout ouvert $O$ non vide
  de $\mathcal{T}$, $O\cap (X\setminus \mathrm{adh(A)})$ est non vide. De même
  pour le complémentaire de l'adhérence de $B$.

  On doit montrer que le complémentaire de l'adhérence de $A\cup B$ est dense
  dans $X$. L'adhérence de $A\cup B$ correspondant à l'union des adhérences,
  on doit donc montrer que (en appliquant les lois de De Morgan):
  $$\forall O \in \mathcal{T}, O\neq \varnothing\implies
  O \cap (X\setminus \mathrm{adh}(A))\cap (X\setminus \mathrm{adh}(B))\neq\varnothing$$
  Soit $O$ un ouvert de $X$.
  Puisque le complémentaire de l'adhérence de $A$ est un ouvert dense,
  $O\cap (X\setminus \mathrm{adh}(A))$ est un ouvert non vide. Puique
  le complémentaire de l'adhérence de $B$ est dense, on a le résultat.
\end{proof}

Toutefois la classe des ensembles nulle part denses n'est pas stable par union
dénombrable. On introduit donc la notion d'ensemble de première catégorie:

\begin{df}
  Un sous-ensemble $A$ de $X$ est dit de première catégorie s'il s'agit d'une
  union dénombrable de parties nulle part denses.
\end{df}

Par exemple, dans $\mathbb{R}$ muni de sa topologie usuelle, l'ensemble
des nombres rationnels est de première catégorie (il s'agit d'une union
dénombrable de singletons) mais n'est pas nulle part dense car son
adhérence étant $\mathbb{R}$, l'intérieur de son adhérence est non vide.

En particulier, ceci montre que l'ensemble des parties de première
catégorie de $X$ n'est pas stable par fermeture en général, en plus de
donner un exemple de sous-ensemble de première catégorie qui n'est pas
nulle part dense.

\begin{prop}
  Le sous-ensemble des parties de première catégorie de $X$ est héréditaire
  et stable par union dénombrable.
\end{prop}

\begin{proof}
  L'hérédité est claire. Une union dénombrable d'unions dénombrables d'ensembles
  nulle part denses étant une union dénombrable d'ensembles nulle part denses,
  l'autre partie du résultat est claire.
\end{proof}

La dernière définition introduite ici est celle de la propriété de Baire.
\begin{df}[Propriété de Baire]
  On dit que $X$ a la propriété de Baire si toute intersection dénombrable
  d'ouverts denses de $X$ est dense dans $X$. Dans ce cas on dit que $X$
  est un espace de Baire.
\end{df}

\section{Formulations équivalentes}
Soit $(X, \mathcal{T})$ un espace topologique fixé.
 
La propriété de Baire peut être exprimée de manière équivalente
en termes de fermés:
\begin{prop}
  $X$ a la propriété de Baire \ssi{} toute union dénombrable de fermés
  d'intérieur vide est d'intérieur vide.
\end{prop}

\begin{proof}
  Supposons que $X$ a la propriété de Baire. Soit $(F_n)_{n\in\mathbb{N}}$ une
  famille de fermés de $X$ d'intérieur vide. Alors la famille
  $(X\setminus F_n)_{n\in\mathbb{N}}$ est une famille d'ouverts denses dans $X$
  (car l'adhérence du complémentaire d'un ensemble correspond au complémentaire
  de son intérieur). Par la propriété de Baire,
  $$X\setminus\mathrm{int} \left(\bigcup_{n\in\mathbb{N}} F_n \right) 
  =\mathrm{adh}\left(\bigcap_{n\in\mathbb{N}} X\setminus F_n\right) = X$$
  En passant au complémentaire, on a la première implication.

  L'autre implication se démontre de manière similaire et est donc laissée
  à titre d'exercice.
\end{proof}

On peut également énoncer la propriété de Baire en termes d'ensembles
résiduels. Introduisons les:
\begin{df}
  Un sous-ensemble $A$ de $X$ est dit résiduel si son complémentaire
  est de première catégorie.
\end{df}

\begin{prop}
  $X$ a la propriété de Baire \ssi{}  tout
  ensemble résiduel est dense dans $X$.
\end{prop}

\begin{proof}
  Supposons que $X$ est un espace de Baire. Soit $A$ un ensemble résiduel,
  c'est-à-dire $X\setminus A = \bigcup_{n\in\mathbb N} S_n$ où les $S_n$ sont
  des sous-ensembles de $X$ nulle part denses. Alors chaque fermeture de $S_n$
  est un sous-ensemble de $X$ d'intérieur vide. Par la propriété de Baire,
  $$\mathrm{int}(X\setminus A) =
  \mathrm{int}\left(\bigcup_{n\in\mathbb N}S_n\right)= \varnothing$$
  En passant au complémentaire, on déduit que l'adhérence de $A$ est $X$,
  c'est-à-dire $A$ est dense dans $X$.

  Réciproquement, supposons que tout ensemble résiduel est dense dans $X$.
  Soit $(F_n)_{n\in\mathbb N}$ une famille de fermés d'intérieur vide. Alors
  il s'agit d'une famille de sous-ensembles nulle part denses de $X$, donc
  le complémentaire de leur union est dense dans $X$ (car il s'agit d'un
  ensemble résiduel). Ceci implique que l'intérieur de l'union des $F_n$
  est vide, ce qu'on voulait montrer.
\end{proof}


\section{Théorème de Baire}
Le théorème de Baire affirme que les espaces métriques complets
sont des espaces de Baire, ce qui nous fournira des exemples
concrets d'espaces de Baire.

\begin{thm}[Théorème de Baire]
  Soit $(X, d)$ un espace métrique. Si $X$ est complet, alors
  il a la propriété de Baire.
\end{thm}

\begin{proof}
  Soit $(O_n)_{n\in\mathbb{N}}$ une famille dénombrable d'ouverts denses
  dans $X$. Montrons que tout ouvert de $X$ a une intersection non vide
  avec $\bigcap_{n\in\mathbb{N}}O_n$. Soit $U$ un ouvert de $X$.

  Puisque $O_0$ est un ouvert dense dans $X$, il existe $x_0$ dans
  $O_0\cap U$. Il existe $r_0 \in \left]0, 1\right]$ tel
  que $\bar{B}(x_0, r_0)\subseteq  O_0\cap U$\footnote{La notation
    $\bar{B}(y, r)$ est une notation adoptée ici pour dénoter la boule
    fermée de centre $y$ et de rayon $r$. Pour rappel, dans un espace
    métrique, l'adhérence de la boule ouverte ne correspond pas nécessairement
    à la boule fermée.}.
  De même, puisque $O_1$ est dense dans $X$, il existe $x_1$ dans
  $O_1\cap B(x_0, r_0)$, et $r_1\in\left]0, 2^{-1}\right]$ tels que
  $\bar{B}(x_1, r_1)\subseteq O_1\cap B(x_0, r_0)$.

  En itérant de la sorte, on construit deux suites $(x_n)_{n\in\mathbb{N}}$
  et $(r_n)_{n\in\mathbb{N}}$ telles que $x_n\in O_n\cap B(x_{n-1}, r_{n-1})$,
  $\bar{B}(x_n, r_n)\subseteq O_n$
  et $r_n\in\left]0, 2 ^{-n}\right]$. La suite $(x_n)_{n\in\mathbb{N}}$ ainsi
  construite est de Cauchy dans un espace métrique complet; elle converge
  donc vers un élément $x$ de $X$.

  Par la première étape, on a $(x_n)_{n\in\mathbb{N}}\subseteq \bar{B}(x_0, r_0)$
  qui est un sous-ensemble de $U$. Ceci montre que $x$ est élément de $U$
  (puisqu'une boule fermée est fermée).

  Pour tous naturels $n$ et $p$, on a par construction $x_{n+p}\in
  \bar{B}(x_n, r_n)$ et en faisant tendre $p$ vers l'infini, étant
  donné que $\bar{B}(x_n, r_n)$ est fermée, on en déduit que $x$
  est un élément de $\bar{B}(x_n, r_n)$ qui est contenue dans $O_n$,
  ce qui prouve que $x$ est bien dans l'intersection des $O_n$.
\end{proof}

Un autre résultat repris comme faisant partie du théorème de Baire concerne
la topologie induite.
\begin{thm}\label{baire:ind}
  Soient $(X, \mathcal{T})$ un espace de Baire et $O$ un ouvert de $X$.
  $O$ muni de la topologie induite est également un espace de Baire.
\end{thm}

\begin{rem}[Rappels]
  Soit $(X, \mathcal{T})$ un espace topologique.
  Rappelons que si $B$ est un sous-ensemble de $X$, la topologie
  induite sur $B$ est la topologie $\mathcal{T}_B$ suivante:
  $$\mathcal{T}_B=\left\{U\cap B\mid U\in\mathcal{T}\right\}$$
  De plus, si on note $\mathcal{F}$ l'ensemble des fermés de $X$,
  on a la caractérisation suivante des fermés de la topologie
  induite:
  $$\mathcal{F}_B=\left\{F\cap B\mid F\in\mathcal{F}\right\}$$

  En particulier, l'intérieur d'un sous-ensemble $A$ de $B$
  au sens de $B$ correspond à l'intersection de son intérieur
  au sens de $X$ et de $B$ (car on regarde au plus grand ouvert
  contenu dans $A$).
  De même, la fermeture de $A$ au sens de $B$ correspond
  l'intersection de $B$ et de la fermeture de $A$ au sens
  de $X$ (car cette fois on on regarde au plus petit fermé
  contenant $A$).
\end{rem}

Nous rappelons un résultat vu l'année passée qui servira dans la preuve
d'un lemme intermédiaire.
\begin{lem}\label{rap:top1}
  Soient $(X, \mathcal{T})$ un espace topologique, $U$ et $V$ deux ouverts
  de $X$. Si $U$ est $V$ sont disjoints, alors l'adhérence de $U$ est disjointe
  de $V$.
\end{lem}
\begin{proof}
  Supposons qu'il existe $x$ un élement de $\mathrm{adh}(U)\cap O$.
  Alors pour tout voisinage $V$ de $x$, $V$ intersecte $U$ puisque
  $x$ est dans l'adhérence de $U$. Or $O$ est un voisinage de $x$
  donc son intersection avec $U$ doit être non vide. Cela constituant
  une contradiction, que $\mathrm{adh}(U)\cap O$ est vide.
\end{proof}


Pour prouver le théorème \ref{baire:ind},
on introduit tout d'abord un lemme utile
à sa preuve.
\begin{lem}\label{baire:ind:help}
  Soient $(X, \mathcal{T})$ un espace de Baire et $O$ un ouvert de $X$.
  Un sous-ensemble $A$ nulle part dense de $O$ (au sens de la topologie
  induite) est nulle part dense dans $X$ (au sens de $\mathcal{T}$).
\end{lem}

\begin{proof}
  Montrer que $A$ est nulle part dense revient à montrer que
  tout ouvert compris dans l'adhérence de $A$ (au sens de $X$) est vide. Soit $U$
  un ouvert de $X$ tel que $U$ est contenu dans l'adhérence
  de $A$. Alors $U\cap O$ est contenu dans $\mathrm{adh}_O(A) =
  \mathrm{adh}_X(A)\cap O$, d'intérieur vide, ce qui implique que $U$ et $O$
  sont disjoints.

  Par le lemme \ref{rap:top1}, $U\cap \mathrm{adh}_X(O)$
  est également vide, or on a la chaîne d'inclusions suivante:
  $U\subseteq \mathrm{adh}_X(A)\subseteq \mathrm{adh}_X(O)$;
  ce qui implique que $U$ est vide.
\end{proof}

Nous pouvons donc prouver assez facilement le théorème \ref{baire:ind} en utilisant
le lemme \ref{baire:ind:help}

\begin{proof}[Preuve du théorème \ref{baire:ind}]
  Soit $S$ un sous-ensemble de première catégorie de $O$.
  Montrons que $\mathrm{adh}_O(O\setminus S) = O$.
  Par le lemme \ref{baire:ind:help}, $U$ est de première
  catégorie au sens de $\mathcal{T}$. Donc l'ensemble $X\setminus
  S$ est dense dans $X$. On a:
  $$\mathrm{adh}_O(O\setminus S) = \mathrm{adh}_X(O\setminus S)\cap O
  = \mathrm{adh}_X((X\setminus S)\cap O)\cap O\supseteq
  \mathrm{adh}_X((X\setminus S)) \cap \mathrm{adh}_X(O)\cap O = O$$

  L'inclusion réciproque étant toujours vraie, on a l'égalité.
\end{proof}

\section{Corollaires et applications}
Cette section présente plusieurs résultats et corollaires
découlant de la propriété de Baire.

\begin{prop}
  Soient $(X, \mathcal{T})$ un espace de Baire et $(F_n)_{n\in\mathbb N}$
  une famille de fermés d'intérieur vide. L'union des $F_n$ n'est pas
  $X$.
\end{prop}
\begin{proof}
  L'union étant d'intérieur vide, elle ne peut être $X$.
\end{proof}

\begin{prop}\label{baire:cor:intf}
  Soient $(X, \mathcal{T})$ un espace de Baire et $(F_n)_{n\in\mathbb N}$
  une famille de fermés de $X$. Si l'union des $F_n$ est $X$,
  alors il existe $n_0$ tel que $F_{n_0}$ est d'intérieur non vide.
\end{prop}
\begin{proof}
  S'ils étaient tous d'intérieur vide, alors leur union le serait également,
  et elle ne pourrait pas être $X$.
\end{proof}

On peut raffiner ce résultat:
\begin{prop}
  Soient $(X, \mathcal{T})$ un espace de Baire et $(F_n)_{n\in\mathbb N}$
  une famille de fermés de $X$. Si l'union des $F_n$ est $X$,
  alors l'union des intérieurs des $F_n$ est dense dans $X$
\end{prop}

\begin{proof}
  On pose $U = \bigcup_{n\in\mathbb N} \mathrm{int}F_n$, qui est
  ouvert. On pose également, pour tout naturel $n$,
  $F'_n = F_n\cap (X\setminus U)$, qui est fermé.

  Pour tout naturel $n$, $F'_n$ est d'intérieur vide; $x$ est dans
  l'intérieur de $F'_n$ \ssi{} il est à la fois dans l'intérieur de
  $F_n$ et de $X\setminus U$, c'est-à-dire il existe un ouvert $O_1$
  (resp. $O_2$) contenant $x$ inclus dans $F_n$ (resp. $X\setminus U$),
  ce qui implique que $x$ est dans $O_1\cap O_2$ qui est inclus dans
  l'intersection de l'intérieur de $F_n$ et de $E\setminus U$ qui est vide.

  Par la propriété de Baire, l'intérieur de l'union des $F'_n$ est vide,
  c'est-à-dire l'ensemble
  $$\bigcup_{n\in\mathbb N} \left(F_n \cap (X\setminus U)\right) =
  (X\setminus U)\cap \bigcup_{n\in\mathbb N} F_n$$
  est d'intérieur vide. C'est-à-dire $X\setminus U$ est d'intérieur vide
  par hypothèse sur les $F_n$. Cela montre que
  $U$ est dense dans $X$.
\end{proof}

Une autre propriété des espaces de Baire est qu'ils ne sont pas de première
catégorie.
\begin{prop}
  Soit $(X, \mathcal T)$ un espace de Baire. Alors $X$ n'est pas de
  première catégorie.
\end{prop}
\begin{proof}
  Si $X$ était de première catégorie, son complémentaire (le vide) serait
  dense dans $X$ par la propriété de Baire, ce qui est une contradiction.
\end{proof}

\begin{prop}
  Tout intervalle de $\mathbb R$ n'est pas de première catégorie.
\end{prop}
\begin{proof}
  Supposons par contradiction qu'il existe un intervalle $I$
  de première catégorie. Alors son complémentaire est dense dans
  $\mathbb R$. Or le complémentaire de intervalle a pour
  adhérence $\mathbb R\setminus \mathrm{int}(I)\neq \mathbb R$
\end{proof}

On introduit deux définitions pour les résultats à venir.
\begin{df}
  Soit $(X, \mathcal{T})$ un espace topologique. On appelle $G_\delta$ un
  sous-ensemble de $X$ qui est une intersection dénombrable d'ouverts et
  $F_\sigma$ un sous-ensemble de $X$ qui est une union dénombrable de fermés.
\end{df}

\begin{prop}
  Soient $(X, \mathcal{T})$ un espace de Baire et $A$ un sous-ensemble
  de $X$. $A$ est résiduel si et seulement s'il contient un $G_\delta$
  dense.
\end{prop}

\begin{proof}
  Supposons $A$ résiduel, c'est-à-dire le complémentaire de $A$
  s'écrit comme $\bigcup_{n\in\mathbb N} S_n$ pour des sous-ensembles
  $S_n$ de $X$ qui sont nulle part denses. Puisque $\bigcup_{n\in\mathbb N} S_n$
  est contenue dans $\bigcup_{n\in\mathbb N} \mathrm{adh}(S_n)$ (qui est d'intérieur
  vide car les $S_n$ sont nulle part denses et $X$ est un espace de Baire),
  on a que $X\setminus \bigcup_{n\in\mathbb N} \mathrm{adh}(S_n)$ est contenu
  dans $A$. Il est facile de vérifier qu'il s'agit d'un $G_\delta$ dense.

  Réciproquement supposons que $A$ contient un $G_\delta$ dense, c'est-à-dire
  qu'il existe une famille dénombrable d'ouverts $(O_n)_{n\in\mathbb N}$ telle
  que $\bigcap_{n\in\mathbb N}O_n$ est contenue dans $A$ et est dense. Chaque
  $O_n$ est dense dans $X$ car il contient un sous-ensemble dense dans $X$.
  Le complémentaire de $A$ est donc contenu dans le complémentaire du
  $G_\delta$; il s'agit d'une intersection de fermés d'intérieur vide, donc
  de première catégorie. Par hérédité
  (de la propriété \og être de première catégorie\fg), le complémentaire
  de $A$ est de première catégorie, c'est-à-dire $A$ est résiduel.
\end{proof}

\begin{prop}\label{baire:base}
  Soit $(E, \|.\|)$ un espace de Banach. Toute base algébrique
  (dite base de Hamel) est soit finie, soit non dénombrable.
\end{prop}

\begin{proof}
  Puisque $E$ est complet, il s'agit d'un espace de Baire.

  Si la dimension de $E$ est finie, alors on a le résultat.
  Supposons que la dimension de $E$ est infinie et supposons
  par l'absurde qu'il existe une base algébrique $(e_n)_{n\in\mathbb N}$
  dénombrable de $E$.

  On pose pour tout naturel $n$, $F_n = \langle e_0, \ldots, e_n\rangle$ le
  sous-espace vectoriel engendré par les vecteurs $e_0$ jusque $e_n$.
  Alors il est fermé car tout sous-espace vectoriel de dimension finie
  est fermé et il est de dimension $n+1$.

  On a $E = \bigcup_{n\in\mathbb N} F_n$. Donc par la proposition
  \ref{baire:cor:intf}, il existe $n_0$ tel que l'intérieur de
  $F_{n_0}$ est non vide. Soient $x\in \mathrm{int}(F_{n_0})$, $r > 0$
  tels que $B(x, r)\subseteq F_{n_0}$. Par symétrie, $-x$
  est également un élément de $F_{n_0}$. On en déduit que la boule
  $B(0, r)$ est contenue dans $F_{n_0}$. En divisant chaque $e_n$
  par une constante appropriée $c_n$ non nulle, on a $c_n e_n$
  dans $B(0, r)$ ce qui montre que chaque $e_n$ est dans $F_{n_0}$,
  d'où $E \subseteq F_{n_0}$, ce qui contredit que $E$ est de dimension infinie.
\end{proof}

\begin{prop}
  L'espace vectoriel normé $(\mathscr{C}[0, 1], \|.\|_1)$ n'est pas un
  espace de Baire
\end{prop}
Un corollaire de cette proposition est qu'il ne s'agit pas d'un espace de
Banach.

\begin{proof}
  Soit $B_\infty = \{ f\in \mathscr C [0, 1]\mid \|f\|_\infty\leq 1\}$.
  Montrons qu'il s'agit d'un fermé au sens de la norme 1.

  Soient $(f_n)_{n\in\mathbb N}$ une suite de fonctions dans $B_\infty$ et
  une fonction $f\in\mathscr C[0, 1]$ telles que $f_n\xrightarrow{\|.\|_1} f$.
  Supposons par contradiction que $f$ n'est pas dans $B_\infty$, c'est-à-dire
  $\|f\|_\infty > 1$. Il existe $a\in [0, 1]$ et $\delta > 0$ tel que
  $|f(a)| > 1+\delta$. Par continuité,  il existe $r>0$ tel que tout $x$
  dans l'intervalle $I := \left]a-r, a+r\right[\cap [0, 1]$ vérifie
  $|f(x)| > 1 + \frac \delta 2$. Pour tout naturel $n$, on a:
  \begin{IEEEeqnarray*}{rCl}
    \int_0^1 |f_n(x) - f(x)|\mathrm{d}x&\geq & \int_I |f_n(x) - f(x)|\mathrm{d}x \\
    & \geq & \int_I (|f_n(x)| - |f(x)|)\mathrm{d}x \\
    & \geq & \int_I \left(\left(1 +\frac \delta 2\right) - 1\right)\mathrm{d}x \\
    & \geq & \frac{\delta r}{2} > 0
  \end{IEEEeqnarray*}
  Ce qui contredit la convergence de la suite des $f_n$ vers $f$ au sens de la
  norme 1. Donc $f\in B_\infty$, ce qui montre que $B_\infty$ est fermée au sens
  de la norme 1.

  Supposons par contradiction que $\mathscr C [0, 1]$ est un espace de Baire.
  Puisque toute fonction dans $\mathscr C [0, 1]$ est bornée, on a
  $\mathscr C [0, 1] = \bigcup_{n\in\mathbb N}n B_\infty$. Il existe donc $n_0$
  tel que $n_0B_\infty$ est d'intérieur non vide. Ceci implique que $B_\infty$
  est d'intérieur non vide. Soient $g$ dans l'intérieur de $B_\infty$ et $r >0$
  tel que $B(g, r)\subseteq B_\infty$. Puisque $B_\infty$ est symétrique, on
  a que $-B(g, r)=B(-g, r)\subseteq B_\infty$, et par convexité de la boule $B_\infty$,
  on a $$B(0, r)\subseteq \frac{1}{2} \left( B(g, r) + B(-g, r)\right) \subseteq B_\infty$$

  Montrons maintenant qu'il existe une constante $K$ telle que toute fonction $f$
  continue sur $[0, 1]$ vérifie $\|f\|_\infty \leq K \|f\|_1$.
  Supposons $f$ non nulle, alors $\frac{r\cdot f}{\|f\|_1}$ est élément
  de $B(0, r)$, donc de $B_\infty$, et on a donc $\|f\|_\infty\leq
  \frac{1}{r}\|f\|_1$.

  Toutefois cette dernière affirmation est une contradiction! Il suffit
  de considérer la fonction $f_n(x) = x^n$ où $n$ est un nombre naturel;
  on a $\|f_n\|_\infty = 1$ et $\|f_n\|_1 = \frac{1}{n+1}$. Or l'affirmation
  qu'on a montré implique que pour tout $n$ naturel, $1\leq \frac{K}{n+1}$.
\end{proof}

\begin{thm}
  Les limites ponctuelles de fonctions définies sur un
  espace de Baire à image dans $\mathbb{R}$ sont
  continues sur un $G_\delta$ dense.
\end{thm}

\begin{proof}
  Plus tard.
\end{proof}


%%% Local Variables:
%%% mode: latex
%%% TeX-master: "../analyse3"
%%% End:


\chapter{Théorème de Banach-Steinhaus}
\section{Théorème}

Il est conseillé de se remémorer les équivalences pour
la continuité d'applications linéaires (le résultat
\ref{cont:lin}) avant de procéder à la lecture de ce
chapitre.


\begin{thm}[Théorème de Banach-Steinhaus (Principle of uniform boundedness)]
  Soient $(E, \|.\|_E)$ un espace de Banach
  et $(F, \|.\|_F)$ un espace vectoriel normé,
  $I$ un ensemble non vide et $(T_i)_{i\in I}$ une famille d'opérateurs
  linéaires continus (c'est-à-dire pour tout $i\in I$,
  $T_i\in \mathscr{L}(E, F)$).

  Supposons que pour tout $x\in E$, $\sup_{i\in I}\|T_i(x)\|_F$ est fini.

  Alors on a que $\sup_{i\in I}\|T_i\|$ est fini.
\end{thm}

L'hypothèse de ce théorème équivaut à dire que pour tout $x$ dans $E$,
l'ensemble $\{\|T_i(x)\|_F\mid i\in I\}$ est borné, ou encore
$$\forall x\in E, \exists C_x, \forall i\in I, \|T_i(x)\|_F \leq C_x$$

La conclusion, quant à elle, équivaut à dire que l'ensemble
$\{\|T_i\|\mid i\in I\}$ est borné. De manière équivalente:
$$\exists C, \forall x\in E, \forall i\in I, \|T_i(x)\|_F\leq C\|x\|_E$$

\begin{proof}
  Pour rappel, $(E, \|.\|_F)$ est un espace de Baire, car il est complet.

  Pour tout naturel $n$, on considère l'ensemble $E_n$ défini par:
  $$E_n = \left\{x\in E\mid \forall i\in I, \|T_i(x)\|_F\leq n\right\}
  =\bigcap_{i\in I}T_i^{-1}(B_F[0, n])$$

  On déduit de la seconde écriture que $E_n$ est
  fermé pour tout $n$ (par continuité des $T_i$). De plus,
  les $E_n$ sont non vides car ils contiennent tous $0$.

  On a l'égalité $E = \bigcup_{n\in\mathbb N}E_n$ car pour tout
  $x$ dans $E$, l'ensemble $\{\|T_i(x)\|_F\mid i\in I\}$ est borné.
  Il existe donc un naturel $n_0$ tel que $\mathrm{int}(E_{n_0})$
  est non vide (par la proposition \ref{baire:cor:intf}),
  c'est-à-dire il existe
  un élément $x_0$ de $E_{n_0}$ et un réel $r>0$ tels que $B_E(x_0, r)
  \subseteq E_{n_0}$.

  Ceci implique que pour tout $z$ de la boule unité de $E$ et
  pour tout $i$ dans $I$, on a l'inégalité
  $\|T_i(x_0 + r z)\|_F\leq n_0$. D'où les inégalités (en utilisant
  l'inégalité triangulaire renversée et la linéarité):
  $$\|T_i(z)\|_F \leq \frac 1 r \left( n_0 + \left\|T_i (x_0)\right\|_F \right) \leq
  \frac{1}{r} \left( n_0 + \sup_{i\in I}\|T_i(x_0)\|_F \right)$$

  Ceci implique que pour tout $i$ dans $I$, $\|T_i\|$ est
  majorée par une constante indépendante de $i$, ce qui
  fournit le résultat.

\end{proof}

Soit $ \mathbb K$ un corps (le corps des réels ou des complexes)
et $(E, \|.\|)$ un espace de Banach sur $\mathbb K$. Soit
$(x^*_n)_{n\in\mathbb N}$, telle que $x^*_n\in E^*$ pour tout naturel $n$.

En particulier, si on suppose que pour tout élément $x$ de $E$,
$\sup_{n\in\mathbb N}|x^*_n(x)|$ est fini, alors
$\sup_{n\in\mathbb N}\|x^*_n\|$  est fini.

Le résultat n'est plus vrai si on ne suppose pas la complétude de
$E$.
\begin{ex}
  Soit $E$ l'espace des fonctions polynomiales de $[0, 1]$
  dans $\mathbb R$. Pour $p\in E$, $p(t) =
  \sum_{n=0}^Na_n t^n$ pour tout $t\in [0, 1]$, on pose
  $\|p\| = \max_{0\leq n\leq N}|a_n|$.

  Alors $E$ n'est pas complet; on peut voir cela comme une conséquence
  du résultat qui affirme que tout espace de Banach n'admet pas
  de base algébrique dénombrable.

  % Alternativement, il s'agit d'une conséquence du théorème d'approximation
  % de Weierstrass\footnote{
  %   Il affirme que toute fonction
  %   continue sur l'intervalle $[0, 1]$ peut être approchée uniformément
  %   par une suite de polynômes.}.

  % Une troisième manière est de montrer
  % que la suite $\displaystyle{
  %   \left(x\mapsto \sum_{n=0}^N\frac{x^n}{n!}\right)_{N\in\mathbb N}}
  % $ est de
  % Cauchy au sens de la norme que nous nous sommes fixés, mais n'est
  % pas convergente car sa limite n'est pas un polynôme (exercice).

  Pour tout naturel $n$, on considère la forme linéaire et continue
  \begin{equation*}
  x^*_n: E\to \mathbb R: \sum_{k=0}^Na_kt^k\mapsto
  \begin{cases}
    n a_n \mbox{ si $n\leq N$} \\
    0 \mbox{ sinon}
  \end{cases}
  \end{equation*}

  La norme de $x^*_n$ est $n$ quel que soit $n$. Or, quel que
  soit le polynôme $p\in E$, l'ensemble $\{|x^*_n(p)|\mid n\in\mathbb N\}$
  est fini donc borné. Toutefois, $\sup_{n\in\mathbb N} \|x^*_n\|$
  n'est pas fini!
\end{ex}

\section{Conséquences du théorème}

\begin{cor}
  Soient $(E, \|.\|_E)$, $(F, \|.\|_F)$ des espaces de Banach et
  $(T_n)_{n\in\mathbb N}$ une suite d'éléments de $\mathscr{L}(E, F)$
  convergeant ponctuellement vers une fonction $T: E\to F$.

  La fonction $T$ est linéaire et on a les inégalités:
  $$\|T\| \leq \sup_{n\in\mathbb N}\|T_n\| < \infty$$
  En particulier, $T$ est continue.
\end{cor}

\begin{proof}
  Soient $a$ un scalaire et $x, y$ deux éléments de $E$. Pour tout
  naturel $n$, on a par linéarité de $T_n$ que
  $T_n(ax + y) = a T_n(x) + T_n(y)$. En passant à la limite sur $n$,
  on a par unicité de la limite que $T(ax + y) = aT(x) + T(y)$, ce qui
  montre la linéarité.

  Puisque $T_n$ converge ponctuellement vers $T$, on a que pour
  tout $x$ dans $E$, $\|T_n(x)\|$ converge vers $\|T(x)\|$. Ceci
  implique que pour tout $x$ dans $E$, $\sup_{n\in\mathbb N} \|T_n(x)\|$
  est fini (car toute suite convergente est bornée). Par le
  théorème de Banach-Steinhaus, $\sup_{n\in\mathbb N}\|T_n\|$ est fini.

  Il reste à montrer l'inégalité $\|T\|\leq \sup_{n\in\mathbb N}\|T_n\|$
  pour conclure. Soit $x$ un élément de $E$. Quel que soit le naturel
  $n$ considéré, on a:
  $$\|T_n(x)\|_F\leq \|T_n\|\|x\|_E\leq
  \left(\sup_{k\in\mathbb N}\|T_k\|\right)\|x\|_E$$

  On conclut en passant à la limite sur $n$.
\end{proof}

\begin{prop}
  Soient $(E, \|.\|_E)$, $(F, \|.\|_F)$ des espaces de Banach et
  $(T_n)_{n\in\mathbb N}$ une suite d'éléments de $\mathscr{L}(E, F)$
  convergeant ponctuellement vers une fonction $T: E\to F$.

  Pour tout compact $K$ de $E$, la suite des $T_n$ converge
  uniformément sur $K$ vers $T$.
\end{prop}

\begin{proof}
  On a $T\in\mathscr{L}(E, F)$ (par le résultat précédent).

  Soit $K$ un compact de $E$. Supposons par l'absurde que
  la suite des $T_n$ ne converge pas uniformément vers $T$
  sur $K$, c'est-à-dire qu'il existe $\varepsilon > 0$ tel
  que pour tout naturel $N$, il existe un naturel $n > N$
  et un élément $x$ de $K$ tels que $\|T_n(x)-T(x)\| > \varepsilon$.

  Il existe donc, pour $N=0$, un naturel $n_0>0$ et un élément $x_0$
  de $K$ tel que $\|T_{n_0}(x_0)-T(x_0)\|> \varepsilon$.
  En continuant
  ainsi, on construit une suite $(x_k)_{k\in\mathbb{N}}$ et une suite
  $(n_k)_{k\in\mathbb N}$ telles que pour tout naturel $k$,
  $\|T_{n_k}(x_k)-T(x_k)\|> \varepsilon$ et $n_k > k$.

  Par compacité séquentielle, il existe un élément $x$ de $K$
  et une sous-suite $(x_j)_{j\in J}$ de $(x_k)_{k\in \mathbb N}$
  convergeant vers $x$. On a, pour tout $j$ dans $J$:
  \begin{IEEEeqnarray*}{rCl}
    \varepsilon &\leq & \|T_{n_j}(x_j)-T(x_j)\| \\
    & \leq & \|T_{n_j}(x_j) - T_{n_j}(x)\| + \|T_{n_j}(x)-T(x)\| \\
    & \leq & \|T_{n_j}\|\cdot \|x_j-x\| + \|T_{n_j}(x)-T(x)\| \\
    & \leq & \sup_{n\in\mathbb N} \|T_n\|\cdot \|x_j-x\| + \|T_{n_j}(x)-T(x)\|
  \end{IEEEeqnarray*}

  Or puisque le dernier membre de cette inégalité converge vers $0$
  lorsque $j$ tend vers l'infini (par convergence ponctuelle des $T_n$
  et car $x_j\xrightarrow{j\in J}x$), il y a contradiction avec la
  stricte positivité de $\varepsilon$.
\end{proof}

Le résultat précédent n'est pas vrai si on considère un ensemble borné
non compact. Prenons $E = c_0$, $F=\mathbb R$ et pour tout
naturel $n$, $e_n^*: c_0\to\mathbb R: x\mapsto x_n$ qui est linéaire continue.
Alors $e_n^*$ converge ponctuellement vers $0$ (par définition de $c_0$),
mais sur la boule unité de $c_0$ (qui est bornée), la suite des $e_n^*$ ne
converge pas uniformément vers $0$. En effet, quel que soit $n$ le naturel
considéré, il existe un élément $x$ de la boule unité de $c_0$ tel que
$|e_n^*(x)| > 1/2$ (prendre $x = e_n$).

\begin{prop}\label{bs:borne}
  Soit $(E, \|.\|)$ un espace vectoriel normé et $B$ un sous-ensemble
  de $E$. Supposons que quel que soit $x^*\in E^*$ que l'on considère,
  $x^*(B)$ est borné. Alors $B$ est borné.
\end{prop}

\begin{proof}
  Soit pour $b\in B$, l'application linéaire $T_b: E^*\to\mathbb{K}:
  x^*\mapsto x^*(b)$. On a par le corollaire \ref{hb:a:cor2} que
  $\displaystyle{\|b\| =
    \max_{\substack{\|x^*\|\leq 1 \\ x^*\in E^*}}|x^*(b)| =\|T_b\|}$

  Par hypothèse, quel que soit $x^*\in E^*$,
  $\sup_{b\in B}|T_b(x^*)| = \sup_{b\in B}|x^*(b)|$
  est fini. Alors par le théorème de Banach-Steinhaus,
  $\sup_{b\in B}\|T_b\| = \sup_{b\in B}\|b\|$ est fini, ce qui
  montre que $B$ est borné.
\end{proof}

%%% Local Variables:
%%% mode: latex
%%% TeX-master: "../analyse3"
%%% End:


\part{Second quadrimestre}
\chapter{Théorèmes de l'application ouverte et du graphe fermé}
\section{Théorème de l'application ouverte}
On rappelle qu'une application $f(X, \tau_X)\to(Y, \tau_Y)$ est ouverte
si l'image de tout ouvert de $X$ par $f$ est un ouvert de $Y$.
\'{E}nonçons le théorème de l'application ouverte.

\begin{thm}[Théorème de l'application ouverte]
  Soient $E$, $F$ deux espaces de Banach et $T\in\mathscr{L}(E, F)$.

  Si $T$ est une surjection, alors $T$ est ouverte.
\end{thm}

Prouvons tout d'abord un résultat qui nous permet de montrer
qu'un opérateur est ouvert.
\begin{prop}
  Soit $T: E\to F$ une application linéaire. $T$ est ouverte si et seulement
  si il existe $r>0$ tel que $T\left(B_E(0, 1)\right)\supseteq B_F(0, r)$.
\end{prop}

Remarquez que de manière équivalente, $T$ linéaire est ouverte si et seulement
si il existe $s>0$ tel que $T\left(B_E(0, s)\right)\supseteq B_F(0, 1)$.

\begin{proof}
  Supposons $T$ ouverte. Alors l'image par $T$ de $B_E(0, 1)$ est un ouvert
  de $F$ contenant $0$, ce qui fournit le résultat.

  Réciproquement, supposons qu'il existe $r>0$ tel que
  $T\left(B_E(0, 1)\right)\supseteq B_F(0, r)$. Soit $O$ un ouvert de $E$.
  Soit $T(x)$ un élément de $f(O)$. Il existe $s>0$ tel que $B_E(x, s)%
  \subseteq O$. Puisque $B_E(x, s) = x + s B_E(0, 1)$, on a que $T(O)$ contient
  l'ensemble $T(x) + s T(B_E(0, 1))$ est un sous-ensemble de $f(O)$.
  Par hypothèse, ce sous-ensemble contient $T(x) + s B_F(0, r)$, ce qui
  implique $B_F(T(x), r\cdot{}s)\subseteq T(O)$.
\end{proof}

Nous pouvons désormais prouver le théorème. Nous utilisons les résultats
suivants, implicitement:
\begin{exo}
  Soient $A\subseteq E$ et $T\in\mathscr{L}(E, F)$.
  \begin{enumerate}
  \item On a que $A + A$ comprend $2A$;
  \item Si $A$ est convexe, $A+A = 2A$;
  \item Si $A$ est convexe, alors $T(A)$ est convexe;
  \item Si $A$ est convexe, alors son adhérence aussi.
  \end{enumerate}
\end{exo}
\begin{proof}
  Montrons premièrement qu'il existe $r>0$ tel que $\mathrm{adh}(T(B_E(0, 1)))%
  \supseteq B_F(0, r)$. Afin de faire cela, on utilise le théorème de Baire.

  Pour tout naturel $n\geq 1$, on pose $F_n = n\cdot\mathrm{adh}(T(B_E(0, 1)))$
  qui est un fermé de $F$. Dès lors, l'union des $F_n$ est $F$ est $F$
  par surjectivité de $T$. En effet, il suffit de constater que
  $F_n\supseteq n\cdot T(B_E(0, 1)) =T(B_E(0, n))$.

  Dès lors le théorème de Baire assure qu'il existe $n_0$ tel que
  l'intérieur de $F_{n_0}$ est non vide. On en déduit que l'intérieur
  de $F_1$ est non vide. Soit $x_0$ un élément de l'intérieur de $F_1$.
  Alors il existe $s>0$ tel que
  $B_F(x_0, s)\subseteq  \mathrm{adh}(T(B_E(0, 1)))$. Puisque
  $\mathrm{adh}(T(B_E(0, 1)))$ est symétrique ($A$ est symétrique
  si pour tout élément $a$ de $A$, $-a$ est dans $A$), $B_F(-x_0, s)$
  est également contenue dans $\mathrm{adh}(T(B_E(0, 1)))$. Par convexité
  de $\mathrm{adh}(T(B_E(0, 1)))$, on déduit que
  $B_F(0, s)\subseteq \mathrm{adh}(T(B_E(0, 1)))$.

  Maintenant, on a montré que $\exists r' > 0$, $\mathrm{adh}(T(B_E(0, 1))%
  \supseteq B_F(0, r')$. On en déduit que $\mathrm{adh}(T(B_E(0, r))%
  \supseteq B_F(0, 1)$ où $r= 1/r'$. Pour conclure, on doit montrer l'existence
  de $s>0$ tel que $B_F(0, 1)\subseteq T(B_E(0, s))$, par la proposition.

  Soit $y\in F$, $\|y\| < 1$. Alors, $y\in\mathrm{adh}(T(B_E(0, r)))$.

  Par définition d'adhérence, il existe $x_0\in B_E(0, r)$ tel que
  $\|y-T(x_0)\| < \frac{1}{2}$. Dès lors, $y-T(x_0)$ est élément
  de l'ensemble $\mathrm{adh}(T(B_E(0, r/2)))$. Il existe donc
  $x_1\in B_E(0, r/2)$ tel que $\|y-T(x_0 + x_1)\| < \frac{1}{2^2}$.

  \`{A} l'étape $n$, on a $y - T(x_0 + \cdots + x_{n-1}\in%
  \mathrm{adh}(T(B(0, \frac{r}{2^n})))$. Il existe donc
  $x_n\in B(0, \frac{r}{2 ^{n}})$ tel que
  $\|y - T(x_1 + \cdots + x_n)\| < \frac{1}{2^{n+1}}$.

  La série des $x_n$ converge car elle est absolument convergente (on a
  pour tout naturel $n$, $\|x_n\|<\frac{r}{2^{n}}$). Soit $x$ la limite
  de cette série. Alors,$\|x\|\leq 2r$ et on a $T(x) = y$.
  D'où $y\in T(B(0, 2r))$.

  Il suffit donc de prendre $s = 2r$ pour conclure.
\end{proof}
\begin{cor}[Théorème d'isomorphisme de Banach]
  Soient $E$, $F$ deux espaces de Banach, $T\in\mathscr L(E, F)$.
  Si $T$ est bijective, alors elle est bicontinue (c'est-à-dire sa
  réciproque est également continue).
\end{cor}

\begin{proof}
  Toute application bijective, continue et ouverte est un
  homéomorphisme. Le résultat est immédiat par le théorème de
  l'application ouverte.
\end{proof}

\begin{cor}
  Soient $\|.\|_1$, $\|.\|_2$ deux normes sur un même espace
  vectoriel $E$ telles que $(E, \|.\|_i)$ est un espace de Banach
  pour $i =1, 2$.

  S'il existe $c > 0$ tel que pour tout $x\in E$, $\|x\|_1\leq c\|x\|_2$ 
  (c'est-à-dire que la norme $\|.\|_2$ domine la norme $\|.\|_1$), alors elles
  sont équivalentes.
\end{cor}

\begin{proof}
  L'application $\mathrm{Id}: (E, \|.\|_2)\to (E, \|.\|_1)$ est linéaire,
  continue et bijective. Par le théorème d'isomorphisme de Banach, l'identité
  est bien un homéomorphisme.
\end{proof}

\begin{prop}
  Soient $E$, $F$ des espaces de Banach. L'ensemble des applications
  linéaires et continues ouvertes $O(E, F)$ est un ouvert de $\mathscr L(E, F)$.
\end{prop}

Avant de prouver la proposition, on introduit le lemme suivant:
\begin{lem}
  Soit $\varepsilon \in \left]0, 1\right[$ et $A\subseteq F$ un sous-ensemble
  borné tels que $A\subseteq T(B_E(0, 1)) + \varepsilon A$.
  Alors $A\subseteq \frac{1}{1-\varepsilon} T(B_E[0, 1])$.
\end{lem}
\begin{rem}
  Le lemme ne requiert pas la complétude de $F$.
\end{rem}

\begin{proof}
  Soit $a = a_0$ un élément de $A$. Il existe $x_0$ un élément
  de $B_E(0, 1)$ et $a_1\in A$ tels que $a_0 = T(x_0) + \varepsilon a_1$.
  De même, il existe $x_1$ un élément de $B_E(0, 1)$ et $a_2$ un élément
  de $A$ tels que $a_1 = T(x_1) + \varepsilon a_2$, ce qui implique
  que
  $$a_0 = T(x_0) + \varepsilon (T(x_1) +\varepsilon a_2).$$

  En continuant ainsi, pour $a_n\in A$, il existe $x_n\in B_E(0, 1)$
  et $a_{n+1}\in A$ tels que $a_n = T(x_n) + \varepsilon a_{n+1}$. On en déduit
  $$a_0 = T(x_0 + \varepsilon x_1 + \cdots + \varepsilon^nx_n)+%
  \varepsilon^{n+1}a_{n+1}.$$
  Puisque $A$ est borné et que $0 < \varepsilon < 1$, on a
  $\varepsilon^na_n \to 0$. De plus, on a:
  $$\sum_{k=0}^{\infty}\varepsilon^k\|x_k\| \leq%
  \sum_{k=0}^{\infty}\varepsilon^k = \frac{1}{1-\varepsilon}$$

  Puisque $E$ est complet, la série des $\varepsilon^kx_k$ converge
  vers un élément $x\in E$. Alors $a_0 = T(x)$ et $\|x\|\leq
  \frac{1}{1-\varepsilon}$ implique $(1-\varepsilon)x \in B_E[0, 1]$.
\end{proof}
Prouvons la proposition.
\begin{proof}
  Soit $T$ une application linéaire et continue ouverte de $E$ dans $F$.
  On recherche $r>0$ tel que $B_{\mathscr L(E, F)}(T, r)\subseteq O(E, F)$.
  Puisque $T$ est une application ouverte, il existe $s>0$ tel que
  $B_F(0, s)\subseteq T(B_E(0, 1))$. Soit $S\in B_{\mathscr L(E, F)}(T, s/2)$,
  montrons que $S$ est une application ouverte.

  Par hypothèse sur $S$, $\|T-S\| < s/2$. Cela implique que pour tout
  $x\in B_E(0, 1)$, $\|T(x) - S(x)\|< s/2$, d'où $\|T(x)\|\leq \| S(x)\| + s/2$.

  En termes d'inclusions d'ensembles, on obtient
  $$T(B_E(0, 1)) \subseteq S(B_E(0, 1))+\frac{1}{2}B_F(0, s).$$
  De l'hypothèse, on déduit,
  $$B_F(0, s) \subseteq S(B_E(0, 1))+\frac{1}{2}B_F(0, s).$$

  Par le lemme, on a $B_F(0, s)\subseteq\frac{1}{1-\frac{1}{2}}S(B_E(0, 1))$.
  On en déduit que $B_F(0, \frac{s}{2})\subseteq L(B_E(0, 1))$
\end{proof}
%%% Local Variables:
%%% mode: latex
%%% TeX-master: "../analyse3"
%%% End:



\chapter{Topologie faible}
\section{Rappels sur la notion de topologie initiale}
On se fixe $E$ un ensemble non vide, $(E_j, \mathcal{T}_j)_{j\in J}$ une famille
d'espaces topologiques et pour chaque $j$ dans $J$ une fonction
$f_j: E\to E_j$.
\begin{df}[Base d'ouverts pour la topologie initiale]\label{init:df}
  On prend comme base d'ouverts les \og rectangles
  ouverts \fg{} suivants:
  \begin{equation*}
    \bigcap_{i\in I}f^{-1}_i(O_i), \mbox{ où }I\subseteq J \mbox{ fini et }
    O_i\in\mathcal{T}_i, \forall i \in I
  \end{equation*}
  Pour rappel, tout ouvert est une union (quelconque) d'éléments
  de la base d'ouverts.
\end{df}

\begin{prop}
  La topologie initiale est la topologie la moins fine
  qui rend les $f_j$ continue.
\end{prop}

\begin{proof}
  On fixe $j\in J$. Soit $O_j$ un ouvert de $\mathcal{T}_j$.
  Alors son image réciproque par $f_j$ est un ouvert
  de la topologie initiale par définition. De plus,
  si une topologie rend continue les $f_j$, elle
  contiendra nécessairement la topologie initiale
  par construction (on vérifie facilement que les
  éléments de la base d'ouverts  de la topologie initiale
  sont ouverts pour la topologie considérée si elle
  rend les $f_j$ continues).
\end{proof}

\begin{prop}\label{init:cont}
  Soient $(F, \mathcal{T}_F)$ et $f: F\to E$ où $E$ est
  muni de la topologie initiale. Alors $f$ est
  continue si et seulement si pour tout $j\in J$,
  $f_j\circ f$ est continue.
\end{prop}

\begin{proof}
  Supposons que $f$ est continue. Alors le résultat est
  immédiat car la continuité est préservée par composition.

  Supposons maintenant que pour tout $j\in J$, $f_j\circ f$
  est continue. Soit $\bigcap_{i\in I}f^{-1}_i(O_i)$,
  où $I\subseteq J$ fini et $O_i\in\mathcal{T}_i$, $\forall i \in I$,
  un ouvert de la base d'ouverts de $E$. Alors
  $$f^{-1}\left(\bigcap_{i\in I}f^{-1}_i(O_i)\right) =
  \bigcap_{i\in I} \left(f_i\circ f\right)^{-1}(O_i)$$
  est, par hypothèse, une intersection finie d'ouverts
  de $F$ (par continuité des $f_i$, $i\in I$) donc est
  ouverte. Donc $f$ est continue.
\end{proof}
Notez que la preuve fonctionne aussi ponctuellement;
$f$ est continue en $x\in E$ si et seulement si
les $f_j\circ f$ sont continues en $x$ (\textbf{exercice}).
Pour montrer cela
on se rappelle que tout voisinage d'un point contient
un ouvert de la base d'ouverts contenant ce point.

\begin{cor}\label{init:lim}
  Soient $(x_n)_n\subseteq (E, \mathcal{T}_{\mathrm{init}})$ et $x\in E$.
  Alors $x_n\xrightarrow{\mathcal{T}_{\mathrm{init}}} x$ si et seulement
  si pour tout $j\in J$, $f_j(x_n)\xrightarrow{\mathcal{T}_{\mathrm{init}}} f_j(x)$.
\end{cor}

Afin de prouver le corollaire, on montre le résultat suivant (qui
est un rappel):
\begin{lem}\label{lim:topo}
  Soit $(X, \mathcal{T})$ un espace topologique, $(x_n)_n$ une
  suite d'éléments de $X$, et $x\in X$. On munit $\bar{\mathbb{N}}=%
  \mathbb{N}\cup\{+\infty\}$ de la topologie dont une base d'ouverts
  est les singletons et les ensembles de la forme
  $\{n\geq n_0\}\cup\{+\infty\}$, où $n_0\in\mathbb N$.

  Dès lors, $x_n\to x$ si et seulement si l'application
  $n\mapsto x_n$, $+\infty\mapsto x$ est continue en $+\infty$.
\end{lem}

\begin{proof}
  On traduit ce que signifient chacune de ces assertions.
  Dire que cette application est continue en $+\infty$ est,
  par définition:
  \begin{equation*}
    \forall V_x\mbox{ voisinage de $x$},
    \exists V_\infty \mbox{ voisinage de $+\infty$},
    \forall n\in V_\infty, x_n\in V_x
  \end{equation*}
  Dire que $x_n\to x$, revient à dire:
  \begin{equation*}
    \forall V_x \mbox{ voisinage de $x$}, \exists n_0,
    \forall n\geq n_0, x_n\in V_x.
  \end{equation*}
  Il est clair que ces assertions sont équivalentes.
\end{proof}

Le corollaire \ref{init:lim} découle immédiatement de la proposition
\ref{init:cont} et du lemme \ref{lim:topo}.

\subsection*{Exemples}
On rappelle deux exemples vus en deuxième année.
\begin{ex}
  Soit $(X, \mathcal{T})$ un espace topologique et $Y\subseteq X$ un
  sous-ensemble non vide de $X$. Alors, la topologie initiale sur $Y$
  associée à l'injection $Y\hookrightarrow X: y\mapsto y$ est la topologie
  induite.
\end{ex}

\begin{ex}
  Soient $((E_i, \mathcal{T}_i))_{i\in I}$ une famille d'espaces topologiques.
  Soit $E = \prod_{i\in I} E_i$. La topologie initiale sur $E$ associée
  aux projections $p_i: E\to E_i$ est la topologie produit.
\end{ex}


\section{Topologie faible}
Soit $(E, \|.\|)$ un espace vectoriel normé réel. La topologie
faible sur $E$ est une topologie initiale sur $E$.
\begin{df}
  La topologie initiale associée aux éléments de $E^*$
  s'appelle topologie faible sur $E$ et est notée $\sigma(E, E^*)$
  (ou encore $\omega$).
\end{df}

La topologie faible est la topologie la moins fine qui rend les
éléments de $E^*$ continus. En particulier, la topologie faible
sur $E$ vérifie $\sigma(E, E^*)\preceq \mathcal{T}_{\|.\|}$.

\begin{lem}
  Soit $f: E\to\mathbb R$ une application linéaire
  continue au sens de la topologie faible. Alors,
  $f$ est continue au sens de la norme $\|.\|$
\end{lem}

\begin{proof}
  Puisque $\sigma(E, E^*)\preceq \mathcal{T}_{\|.\|}$, et
  que tout ouvert $O$ de $\mathbb R$ vérifie, par
  hypothèse que $f^{-1}(O)\in\sigma(E, E^*)$, le
  résultat est immédiat.
\end{proof}

Les éléments de la base d'ouvert de $\sigma(E, E^*)$ sont de la forme
\begin{equation*}
  \bigcap_{j=1}^n (x_j^*)(O_j) =
  \left\{ x\in E\mid \forall 1\leq j\leq n,\quad x_j^*(x)\in O_j\right\}
\end{equation*}
où $n$ est un naturel non nul, $x_j^*$ sont des formes linéaires et continues
sur $E$ et $O_j$ sont des ouverts de $\mathbb R$.

\'{E}tudions les voisinages de $0$. Soit $V_0\subseteq E$ un voisinage
de $0$ au sens de la topologie faible. Alors, il existe $\varepsilon > 0$,
$n\geq 1$ un naturel et $x_j^*\in E^*$, pour $1\leq j\leq n$ tels que
\begin{equation*}
  V_0\supseteq V_{\varepsilon, x_1^*, \ldots, x_n^*}(0) :=
  \left\{ x\in E\mid \forall 1\leq j\leq n,\quad
    |x_j^*(x)| < \varepsilon\right\}
\end{equation*}

\textbf{Explication}: il existe un ouvert $U$ de la base
donnée par la définition de topologie initiale (\ref{init:df})
contenu dans $V_0$, c'est-à-dire
il existe donc un nombre fini (disons $n$) de $x_j^*\in E^*$ et d'ouverts de
$\mathbb R$ contenant $0$ tels que
$U=\bigcap_{j}(x_j^*)^{-1}(O_j)\subseteq V_0$.
Puisque chaque $O_j$ contient un intervalle ouvert centré
en $0$,
on prend pour $\varepsilon$ le minimum des rayons de ces intervalles.
Dès lors, on a:
$$V_{\varepsilon, x_1^*, \ldots, x_n^*}(0)\subseteq U \subseteq V_0.$$

Ceci montre que l'ensemble suivant est un système fondamental
de voisinages de $0$:
$$\left\{V_{\varepsilon, x_1^*, \ldots, x_n^*}(0)\mid \varepsilon > 0,
  n\geq 1, x_j\in E^*, \forall 1\leq j\leq n\right\}. $$

En raisonnant de la même manière, on montre que pour tout $a\in E$,
l'ensemble
$$\left\{V_{\varepsilon, x_1^*, \ldots, x_n^*}(a)\mid \varepsilon > 0,
  n\geq 1, x_j\in E^*, \forall 1\leq j\leq n\right\}$$
est un système fondamental de voisinages de $a\in E$
où $$V_{\varepsilon, x_1^*, \ldots, x_n^*}(a) =
   \left\{ x\in E\mid \forall 1\leq j\leq n,\quad
    |x_j^*(x-a)| < \varepsilon\right\}.$$


\begin{ex}
  Soit $E = \mathbb R^2$. On a vu $E \equiv E^*$.
  Soit $x^*=(x_1, x_2)$ une forme linéaire non nulle sur $E$. L'ensemble
  $\{y = (y_1, y_2)\in \mathbb R^2\mid x^*(y)=0\} = \mathrm{Ker}(x^*)$
  est une droite du plan. Si on fixe $\varepsilon > 0$, on trouve
  que l'ensemble $V_{\varepsilon, x^*}(0)$ est une \og bande\fg dans le plan
  (voir figure \ref{w:ill}).

  En particulier, un tel ouvert contient une droite passant par $0$ et
  tout ouvert de la base d'ouverts est une intersection de telles bandes.
\end{ex}
% Dessin dans le plan d'un ouvert V_epsilon, x^*(0)

\begin{figure}[!h]
  \begin{center}
    \caption{Exemple d'ouvert $V_{\varepsilon, x^*}(0)$ dans le plan}%
    \label{w:ill}
    \begin{tikzpicture}
      \draw [->] (0, -2) -- (0, 2);
      \draw [->] (-2, 0) -- (2, 0);
      \draw (-2, -1) -- (2, 1) node[right] {$\mathrm{Ker}(x^*)$};
      \fill [black, nearly transparent] %
            (-2, -2) -- (-2, 0) -- (2, 2) -- (2, 0) -- cycle;
      \draw [dotted, thick] (-2, 0)  -- (0, 1)%
            node[above left] {$V_{\varepsilon, x^*}(0)$}-- (2, 2);
      \draw [dotted, thick] (-2, -2) -- (2, 0);
    \end{tikzpicture}
\end{center}
\end{figure}
On peut généraliser cet exemple: soit $x^*\in E^*$
non nul, alors l'ensemble $V_{\varepsilon, x^*}(0)$ contient un hyperplan
contenant $0$ (car il contient le noyau de $x^*$).

\begin{df}
  Soient $(x_n)$ une suite d'éléments de $E$ et $x\in E$.
  On dit que $(x_n)$ converge faiblement vers $x$ si la suite
  $x_n\xrightarrow{\sigma(E, E^*)}x$ (\emph{ie}. converge au sens de la topologie
  faible).

  Les notations $x_n\xrightarrow{\omega}x$ et $x_n\rightharpoonup x$ sont
  également employées pour la convergence faible.
\end{df}
\begin{lem}
  Soient $(x_n)$ une suite d'éléments de $E$ et $x\in E$.
  Par le corollaire \ref{init:lim}, pour avoir
  $x_n\xrightarrow{\sigma(E, E^*)}x$,  il suffit de vérifier que pour
  toute forme linéaire $x^*$, on a $$x^*(x_n)\to x^*(x).$$
\end{lem}

Dans le cas d'un espace de Hilbert $H$ sur $\mathbb R$, puisque $H\equiv H^*$
(théorème de Riesz-Fréchet), on a:
$$x_n\xrightarrow{\sigma(E, E^*)}x \iff
\left(\forall y\in H, \langle x_n, y\rangle \to \langle x, y\rangle\right).$$

Il est tout à légitime de se demander s'il y a unicité de la limite
lorsqu'on considère la convergence faible. On a le résultat suivant, qui
assure l'unicité de la limite faible:
\begin{prop}
  $(E, \sigma(E, E^*))$ est séparé.
\end{prop}

\begin{proof}
  Soient $x$, $y$ deux éléments distincts de $E$. Alors $\{x\}$ et
  $\{y\}$ sont deux convexes compacts.
  Par la forme géométrique du théorème de Hahn-Banach, il existe
  un réel $\alpha$, une forme linéaire et continue $x^*$ sur $E$
  tels que $x^*(x) < \alpha < x^*(y)$.
  Alors les ouverts faibles $(x^*)^{-1}(\left]-\infty, a\right[)$
  et $(x^*)^{-1}(\left]a, +\infty\right[)$ contiennent $x$
  et $y$ respectivement et sont clairement disjoints.
\end{proof}

\begin{rem}
  La convergence d'une suite au sens de $\|.\|$ implique la converge faible
  de cette dernière vers la même limite. Toutefois, l'implication réciproque
  n'est pas vraie en général.

  Soit $\ell^2$ l'espace des suites de carré sommable
  (pour rappel il s'agit d'un espace
  de Hilbert), et la suite $(e_n)_{n\in\mathbb N}\in \ell^2$.

  Cette suite converge faiblement vers $0$, car pour tout élément
  $y$ de $\ell^2$, $\langle e_n, y\rangle = y_n \to 0$ (car
  les termes d'une série convergente convergent vers $0$).
  Toutefois, elle ne converge pas vers $0$ au sens de la norme
  car pour tout $n$, $\|e_n\|_2 = 1$.
\end{rem}
%%%Local Variables:
%%% mode: latex
%%% TeX-master: "../analyse3"
%%% End:


\chapter{Théorie ergodique et propriété de Blum-Hanson}
\section{Opérateurs adjoints}
\subsection{Adjoints dans les espaces de Hilbert }
\begin{prop}\label{adj:hilb}
  Soient $H$ un espace de Hilbert et $T\in\mathcal L(H)$. Il existe un opérateur
  linéaire $T^*$ tel que pour tous $x$, $y\in H$, on a
  $$\langle Tx, y\rangle = \langle x, T^*y\rangle$$
  et on a $\|T\| = \|T^*\|$.
\end{prop}

\begin{proof}
  Pour $y\in H$, on définit $y^*:H\to\mathbb K: x\mapsto \langle Tx, y\rangle$.
  Le théorème de Riesz-Fréchet assure dès lors qu'il existe un unique
  $z_y\in H$, tel que pour tout $x$ dans $H$, $\langle x, z_y\rangle =
  \langle Tx, y\rangle$. On pose $T^*y = z_y$.

  Montrons que $T^*$ est linéaire. Soient $y$, $z\in H$ et $a$,
  $b\in\mathbb K$. Pour tout $x\in H$,
  \begin{IEEEeqnarray*}{rCl}
    \langle x, T^*(ay + bz)\rangle & = & \langle Tx, ay + bz\rangle \\
    & = & \bar a \langle Tx, y\rangle + \bar b \langle Tx, z\rangle \\
    & = & \bar a \langle x, T^*y\rangle + \bar b \langle x, T^*z\rangle \\
    & = & \langle x, a T^*(y) + bT^*(z)\rangle.
  \end{IEEEeqnarray*}
  Par non-dégénérescence du produit scalaire, on a
  $T^*(ay + bz) = a T^*(y) + bT^*(z)$, ce qui montre la linéarité.

  Montrons maintenant la continuité de $T^*$. Pour tous $x$, $y\in H$,
  l'inégalité de Cauchy-Schwarz assure que
  $|\langle x, T^*y\rangle| = |\langle Tx, y\rangle| \leq
  \|Tx\|\cdot \|y\| \leq \|T\|\cdot \|x\|\cdot\|y\|$. Rappelons que
  $\|T^*y\| = \sup_{\|x\|\leq 1}|\langle x, T^*y\rangle|$, et on déduit de
  ces inégalités que $\|T^*y\| \leq \|T\| \cdot\|y\|$. Ceci montre la continuité
  de $T^*$.

  Il reste à démontrer la deuxième inégalité pour clôturer la preuve de la
  proposition. Or, pour tous $x\in H$, on a
  $$\|Tx\| = \sup_{\|y\|\leq 1}\langle Tx, y\rangle =
  \sup_{\|y\|\leq 1}\langle x, T^*y\rangle \leq \|x\|\cdot \|T^*\|$$
  ce qui conclut la preuve.
\end{proof}

\begin{df}
  Soient $H$ un espace de Hilbert et $T\in\mathcal L(H)$. On appelle l'opérateur
  $T^*$ donné par la proposition \ref{adj:hilb} l'adjoint de $T$.
\end{df}

\begin{rem}
  L'adjoint d'un opérateur linéaire $T$ est unique. L'application
  $\mathcal L(H)\to\mathcal L(H): T\mapsto T^*$ est bijective, car elle
  est son propre inverse, c'est-à-dire $T^{**} = T$.
\end{rem}

\begin{exo}
  Soit $T: \ell^2\to \ell^2: (x_1, x_2, \ldots)\mapsto (0, x_1, x_2, \ldots)$.
  Déterminez $T^*$.
\end{exo}
\subsection{Adjoints dans les espaces de Banach}
\begin{df}
  Soient $X$, $Y$ des espaces de Banach et $T\in \mathcal L(X, Y)$.
  On définit l'adjointe de $T$ (également appelée transposée ou application
  duale) par $$T^*: Y^*\to X^*: y^*\mapsto y^*\circ T.$$
\end{df}

\begin{prop}
  Soient $X$, $Y$ des espaces de Banach et $T\in \mathcal L(X, Y)$.
  L'application $T^*$ est linéaire et continue et on a l'égalité
  $\|T^*\| = \|T\|$.
\end{prop}
\begin{proof}
  La preuve de la linéarité est laissée en exercice.

  Pour tout $y^*\in Y^*$, on a $\|T^*(y^*)\| \leq \|y^*\| \cdot \|T\|$ (par
  sous-multiplicativité de la norme opérateur), ce qui montre que
  $\|T^*\| \leq \|T\|$. En particulier, $T^*$ est continue.

  L'inégalité réciproque se déduit du corollaire \ref{hb:a:norme} du théorème
  de Hahn-Banach; pour tout $x\in X$, on a:
  $$\|Tx\| = \max_{\substack{\|y^*\|\leq 1 \\ y^*\in Y^*}}|y^*(T(x))|
  =\max_{\substack{\|y^*\|\leq 1 \\ y^*\in Y^*}}|T^*(y^*)(x)|
  \leq \|T^*\|\cdot \|x\|.$$
\end{proof}
\section{Théorie ergodique}
\subsection{Espaces de Hilbert}
\begin{thm}[Théorème ergodique de Von Neumann]
  Soient $H$ un espace de Hilbert et $T\in\mathcal L(H)$ tel que $\|T\|\leq 1$
  (on dit dans ce contexte que $T$ est une contraction).
  Soit
  $$S_n = \frac{1}{n}\sum_{k=0}^{n-1}T^n.$$
  La suite $(S_n)_{n\in\IN}$ converge dans $\mathcal L(H)$ et l'opérateur
  limite est la projection orthogonale sur
  $\mathrm{Fix}(T) = \{x\in H\mid T(x) = x\}$.
\end{thm}

\begin{proof}
  Posons $F = \mathrm{adh}( (\mathrm{Id}-T)(H) )$. Tout élément $x\in H$
  s'écrit de manière unique $x = y +z$ avec $y\in F$ et $z\in F^\perp$ (théorème
  du supplémentaire orthogonal d'un sous-espace vectoriel fermé).
  On considère trois cas avant de les combiner pour conclure.

  \textbf{Cas 1}: $y\in (\mathrm{Id}-T)(H)$. Alors il existe $w\in H$
  tel que $y = w - T(w)$. Alors, pour tout naturel $k$,
  $T^k(y) = T^k(w) - T^{k+1}(w)$. On en déduit que $S_n(y) = 1/n (w - T^n(w))$.
  Comme $T$ est une contraction, on a l'inégalité
  $$\|S_n(y)\| \leq \frac{1}{n}(\|w\| + \|T^n(w)\|) \leq \frac{2\|w\|}{n}$$
  qui implique $S_n(y) \xrightarrow[n\to+\infty]{}0$.

  \textbf{Cas 2}: $y\in \mathrm{adh}((\mathrm{Id}-T)(H))$. Alors pour tout
  $\epsilon >0$, il existe $z\in (\mathrm{Id}-T)(H)$ tel que
  $\|y - z\|\leq \epsilon$. Du cas précédent, on déduit qu'il existe
  $N$ tel que pour tout $n\geq N$, $\|S_n(z)\| \leq \epsilon$.
  Comme $T$ est une contraction et $S_n$ est linéaire, on peut affirmer que
  \begin{IEEEeqnarray*}{rCl}
    \|S_n(y)\| & \leq & \|S_n(y -z)\| + \|S_n(z)\| \\
    & \leq & \frac{1}{n}\left\|\sum_{k=0}^{n-1}T^n(y - z)\right\| + \epsilon \\
    & \leq & \frac{1}{n}\sum_{k=0}^{n-1}\|y - z\| + \epsilon \\
    & \leq & 2\epsilon.
  \end{IEEEeqnarray*}
  On a donc montré que $S_n(y) \xrightarrow[n\to+\infty]{}0$.

  \textbf{Cas 3}: $z\in \mathrm{adh}((\mathrm{Id}-T)(H))^\perp =
  \mathrm{Ker}(\mathrm{Id} - T^*)$. Alors $z$ est un point fixe de $T^*$.
  Montrons que $z$ est un point fixe de $T$. On a:
  \begin{IEEEeqnarray*}{rCl}
    \|z-T(z)\|^2 & = & \|z\|^2 + \|T(z)\|^2- 2 \Re\langle z, T(z)\rangle \\
    & = & \|z\|^2 + \|T(z)\|^2- 2 \Re\langle T^*(z), z\rangle \\
    & = & \|T(z)\|^2 - \|x\|^2 \leq 0.
  \end{IEEEeqnarray*}
  Cette inégalité implique que $\|z-T(z)\| = 0$, c'est-à-dire $z = T(z)$.
  Il en découle que $S_n(z) = z$.

  On combine maintenant les cas présentés. Soit $x\in H$, alors il existe
  d'uniques $y\in F$, $z\in F^{\perp}$ tels que $x = y + z$. On en déduit que
  $$\lim_{n\to+\infty}S_n(x) =
  \lim_{n\to+\infty}S_n(y) + \lim_{n\to+\infty}S_n(z)=
  0 + z = z.$$
  Ceci montre que $S_n$ converge ponctuellement vers l'opérateur annoncé.

  % L'application $\sigma: F \times F^\perp\to H: (y, x) \mapsto y + z$
  % est linéaire,  continue. Comme $F$ et $F^\perp$
  % sont fermés dans $H$, ils sont complets. Par le théorème d'isomorphisme
  % de Banach, $sigma^{-1}$ est continue. Sur $F$, on a par continuité de $S_n$
  % que $S_n = (1/n) (\mathrm{Id} - T^n)$. Sur $F^\perp$, $S_n$ est l'identité.
  % Il est clair que sur chacun de ces sous-espaces, $S_n$ converge en norme
  % (respectivement vers l'application nulle et l'identité).

  Soit $x\in B(0, 1)$, $x = y + z$ avec $(y, z)\in F\times F^\perp$.
  Alors $$\|S_n(x) - z \| = \frac{1}{n}\|y- T^n(y)\| \leq \frac{2}{n}$$
  car $T$ est une contraction et
  $\|y\|^2\leq \|y\|^2 + \|z\|^2 = \|x\|^2 \leq 1$. On déduit de cette
  dernière inégalité que $S_n$ converge uniformément en $x$ vers sa limite
  ponctuelle sur la boule
  unité, ce qui implique que $S_n$ converge en norme vers sa limite ponctuelle.
\end{proof}


\section{Propriété de Blum-Hanson (1960)}
\begin{df}
  Soit $X$ un espace de Banach réel et $(x_n)_{n\in\IN}$ une suite d'éléments de
  $X$.
\end{df}
%%% Local Variables:
%%% mode: latex
%%% TeX-master: "../analyse3"
%%% End:


\chapter{Séparabilité}
\section{Définitions et propriétés}
\begin{df}
  Soit $(E, \mathcal T)$ un espace topologique. On dit que $E$ est séparable
  s'il existe un sous-ensemble $D$ de $E$ dénombrable tel que
  $\bar D = E$.
\end{df}

Un exemple d'espace séparable est $\mathbb R$ (avec sa topologie usuelle).
En effet, $\mathbb Q$ est un sous-ensemble dénombrable dense de $\mathbb R$.

\begin{prop}\label{sep:ind}
  Soient $(E, d)$ un espace métrique et $F\subseteq E$. Si $E$ est séparable,
  alors $F$ est séparable.
\end{prop}
\begin{proof}
  Soit $D = \{x_n \in E\mid {n\in\mathbb N}\}$ tel que $\bar D = E$.
  Pour tous $n$, $k$ naturels, $k\neq 0$, si
  $B\left(x_n, 1/k\right)\cap F$ est non
  vide, on choisit $y_{n, k}$ un élément de cette intersection. Posons
  $$L = \left\{y_{n, k} \in F\mid n, k\in\mathbb N, k \neq 0,
    B\left(x_n, \frac{1}{k}\right)\cap F\neq \varnothing\right\}$$
  et montrons que $\mathrm{adh}_F L = F$.

  Soient $y\in F$ et $k$ un naturel non nul.
  Comme $\bar D = E$, il existe $n$ tel que $d(x_n, y) < 1/k$. Ceci implique
  que $B(x_n, 1/k)\cap F$ est non vide, et donc  $y_{n, k}$ est bien défini.
  On a, par l'inégalité triangulaire que $d(y, y_{n, k}) \leq d(y, x_n) +
  d(x_n, y_{n, k}) < 2/k$.

  Ceci montre que $\mathrm{adh}_F L = F$. Comme $L$ est au plus dénombrable,
  la preuve est terminée.
\end{proof}

\begin{prop}
  Soit $(E, \|.\|)$ un espace vectoriel normé. Les assertions suivantes sont
  équivalentes:
  \begin{enumerate}
  \item $E$ est séparable;
  \item $B(E) = \{x\in E\mid \|x\|\leq 1\}$ est séparable;
  \item $S(E) = \{x\in E\mid \|x\| =  1\}$ est séparable.
  \end{enumerate}
\end{prop}
\begin{proof}
  Au vu de la proposition \ref{sep:ind}, la première assertion implique
  la deuxième et la deuxième la troisième. On montre que la première
  est impliquée par la troisième.

  Soit $D = \{x_n \in S(E) | n\in\mathbb N\}$ tel que $\bar D = S(E)$.
  Posons $L = \{qx_n\mid q\in\mathbb Q^{>0}, n\in\mathbb N\}$. Alors $L$ est
  dense dans $E$ car pour tout vecteur $x\in E$,
  $$x = \|x\| \cdot \frac{x}{\|x\|}\in \left[0, +\infty \right[ \cdot S(E)
  = \mathrm{adh}_{\mathbb R}\mathbb Q^{>0}\cdot \mathrm{adh}_{E}D = \bar L.$$
  {L'argument sous-jacent à ces notations est que la multiplication
    scalaire $\mathbb R\times E\to E$ est continue et qu'en conséquence
    si $\lambda_n\to\lambda$ et $v_n\to v$ sont deux suites convergentes,
    respectivement une suite de réels et de vecteurs, alors la limite de
  la suite $\lambda_nv_n$ existe et vaut $\lambda v$}.
\end{proof}
\section{Exemples d'espaces séparables}
\textbf{Vous êtes invité à travailler ces exemples par vous-même avant de lire
les solutions.}
\begin{ex}
  L'espace de suites $(c_0, \|.\|_\infty)$ est séparable.
\end{ex}
\begin{proof}
  Soit $D = \{(q_0, \ldots, q_N, 0, \ldots)\mid
  N\in\mathbb N, q_j \in\mathbb Q\}$ l'ensemble des suites à coefficients
  rationnels ultimement nulles. Il s'agit d'un ensemble dénombrable
  (se plonge naturellement dans l'union dénombrable $\bigcup_{n}\IQ^n$).
  Montrons que $\bar D = c_0$.

  Soit $x = (x_n)_n\in c_0$.
  Soit $\epsilon > 0$. Il existe $N$ tel que tout $n\geq N$
  vérifie $|x_n|<\epsilon$. Pour $j = 0 \to N$, il existe $q_j\in \IQ$ tel
  que $|x_j-q_j|<\epsilon$ (car $\IQ$ est dense dans $\IR$). Dès lors,
  $(q_0, \ldots, q_N, \ldots)$ est élément de $D$ $\epsilon$-proche de $x$.

  Ceci termine la preuve.
\end{proof}

\begin{ex}
  Soit $p \in \left[1, +\infty\right[$.
  L'espace de suites $(\ell^p, \|.\|_p)$ est séparable.
\end{ex}
\begin{proof}
  Soit $D = \{(q_0, \ldots, q_N, 0, \ldots)\mid
  N\in\mathbb N, q_j \in\mathbb Q\}$ l'ensemble des suites à coefficients
  rationnels ultimement nulles. Il s'agit d'un ensemble dénombrable.

  Soient $x = (x_n)_n\in \ell ^p$ et $\epsilon >0$. Il existe $N$ tel que
  pour tout $n \geq N$,
  $$\sum_{k=n}^{+\infty}|x_k|^p < \epsilon.$$
  Pour $j = 0 \to N$, il existe $q_j$ rationnel tel que
  $|x_j-q_j|^p<\epsilon/N$. Posons $u=(q_0, \ldots, q_N, 0, \ldots)\in D$, alors
  $\|u-x\|^p_p<2\varepsilon$.

  Ceci termine la preuve.
\end{proof}

\begin{rem}
  Dans les deux exemples précédents, on a considéré $c_0$ et $\ell^p$
  comme $\IR$-espaces vectoriels normés. Si on considère leur analogue
  complexe, il est nécessaire de considérer l'ensemble des suites à
  coefficients dans $\IQ(i)$ (ou un autre sous-ensemble dénombrable dense
  de $\IC$) ultimement nulles, plutôt que $\IQ$.
\end{rem}

\begin{ex}
  $\mathcal C([0, 1]; \IR)$ est séparable.
\end{ex}
\begin{proof}
  Le théorème d'approximation de Weierstrass assure que l'espace des polynômes
  est dense dans $\mathcal C([0, 1]; \IR)$. Il suffit donc de montrer qu'il
  existe un sous-ensemble de l'espace des polynômes dénombrable dont l'adhérence
  contient l'espace des polynômes\footnote{Rappel: Soient $(X, \mathcal T)$ un
    espace topologique, $D$ dense dans $X$ et $E\subseteq D$ tel
    que $D\subseteq \bar E$. Par monotonicité de l'adhérence, on a
    $X = \bar D \subseteq \bar{\bar E} =\bar E$, donc $E$ est également dense.}.

  Il suffit de considérer l'ensemble des polynômes à coefficients
  rationnels. Il est bien dense dans l'ensemble des polynômes (car $\IQ$
  est dense dans $\IR$).
\end{proof}
\section{Exemples d'espaces non-séparables}
\textbf{Rappel}: $|\mathcal P (\mathbb N)| = |\mathbb R| = 2^{\aleph_0}$.

Avant de présenter des exemples, on introduit un résultat utile pour montrer
qu'un espace métrique n'est pas séparable.

\begin{prop}
  Soit $(E, d)$ un espace métrique. S'il existe $B\subseteq E$ non dénombrable
  tel que tous $b_1$, $b_2\in B$ distincts on a $d(b_1, b_2)\geq 1$ alors
  tout sous-ensemble $A$ dense de $E$ est non dénombrable.
\end{prop}
\begin{proof}
  Soient $b_1$, $b_2$ deux éléments de $B$. \`A l'aide de l'inégalité
  triangulaire, on obtient que $B(b_1, 1/4)\cap B(b_2, 1/4)=\varnothing$
  (sinon $b_1$ et $b_2$ sont $1/2$-proches).

  Soit $A$ un sous-ensemble dense de $E$. Alors $A$ intersecte chacune des
  boules $B(b, 1/4)$, $b\in B$. Comme elles sont disjointes et que
  $B$ est non dénombrable, cela assure que $A$ est non dénombrable.
\end{proof}

\begin{cor}\label{sep:neg}
  Soit $(E, d)$ un espace métrique. S'il existe $B\subseteq E$ non dénombrable
  tel que tous $b_1$, $b_2\in B$ distincts on a $d(b_1, b_2)\geq 1$ alors
  $E$ n'est pas séparable.
\end{cor}

\begin{ex}
  L'espace de suites $\ell^\infty$ n'est pas séparable.
\end{ex}
\begin{proof}
  On applique le corollaire \ref{sep:neg}. On prend le sous-ensemble
  de $\ell^\infty$ suivant:
  $$B = \{x = (x_n)_n\in\ell^\infty\mid
  \forall n, x_n= \ind_A(n), A\subseteq \IN\}.$$
  L'ensemble $B$ est bien non dénombrable (on peut y plonger les parties
  de $\IN$) et deux éléments distincts seront distants de 1; étant données
  deux parties différentes de $\IN$, il existe un naturel dans une partie
  et pas l'autre, et les suites associées vaudront respectivement 0 et 1
  en la composante correspondante, ce qui implique l'affirmation.
\end{proof}

\begin{ex}
  $(\mathcal C([0, 1]; \IR))^*$ n'est pas séparable.
\end{ex}
\begin{proof}
  Soit pour $t\in [0, 1]$ la forme linéaire $\ev_{t}: C([0, 1]; \IR) \to \IR:
  f\mapsto f(t)$. Cette forme est continue car pour $f\in C([0, 1]; \IR)$,
  $|\ev_t(f)|\leq \|f\|_\infty$.

  On montre que la famille $(\ev_t)_{t\in[0, 1]}$ satisfait les hypothèses
  du corollaire \ref{sep:neg}. Il s'agit bien d'une famille non dénombrable
  d'éléments de $(\mathcal C([0, 1]; \IR))^*$. Considérons maintenant $t$, $s
  \in[0, 1]$, $t\neq s$. Pour montrer que $\ev_t$ et $\ev_s$ vérifient
  $\|\ev_t-\ev_s\|\geq 1$, il suffit de prendre une fonction de norme au
  plus $1$ similaire à celle illustrée à la figure \ref{sepa:ill}.
  % Figure pour illustrer les fonctions à choisir pour montrer que le
% dual de C[0, 1] n'est pas séparable.
\begin{figure}[!h]
  \begin{center}
    \caption{Fonction $f$ illustrant $\|\ev_t-\ev_s\|\geq 1$}%
    \label{sepa:ill}
    \begin{tikzpicture}[scale=0.5]
      \draw [->] (0, -3.4) -- (0, 3.4);
      \draw [->] (0, 0) -- (3.4, 0);
      \draw (1, 0.1) -- (1, -0.1) node [below] {$s$};
      \draw (2, 0.1) -- (2, -0.1) node [below] {$t$};
      \draw (3, 0.1) -- (3, -0.1) node [below] {$1$};
      \draw (0.1, 3) -- (-0.1, 3) node [left] {$1$};
      \draw (0.1, -3) -- (-0.1, -3) node [left] {$-1$};
      \draw [blue] (0, -3) -- (1 , -3) -- (2, 3) -- (3, 3);
    \end{tikzpicture}
\end{center}
\end{figure}


%%% Local Variables:
%%% mode: latex
%%% TeX-master: "../analyse3"
%%% End:

\end{proof}

\begin{ex}
  $\mathcal L(\ell^2)$ n'est pas séparable.
\end{ex}
\begin{proof}
  Notons chaque élément de $\ell^2$ comme suit:
  $$(x_n)_n = \sum_{n=0}^\infty x_n e_n.$$

  Soit $A\subseteq \IN$, on pose
  $$T_A: \ell^2 \to\ell^2: \sum_{n=0}^\infty x_n e_n \mapsto
  \sum_{n\in A} x_n e_n.$$

  On vérifie facilement que $T_A$ est bien définie, linéaire et continue
  (on a $\|T_A\|= 1$ si $A$ est non vide, sinon $T_A$ est l'application nulle).
  L'ensemble $B = \{T_A\mid A\subseteq \IN\}$ vérifie les hypothèses du
  corollaire  \ref{sep:neg}; il s'agit bien d'un ensemble non dénombrable
  de $\mathcal L(\ell^2)$, et étant donnés $A, A'\subseteq \IN$, $A\neq A'$,
  alors il existe (sans perte de généralité, quitte à échanger $A$ et $A'$)
  $n\in A\setminus A'$, d'où $T_A(e_n) = 1$ et $T_{A'}(e_n) = 0$ ce qui
  implique  $\|T_A - T_{A'}\|\geq 1$.
\end{proof}
%%% Local Variables:
%%% mode: latex
%%% TeX-master: "../analyse3"
%%% End:

\appendix
\chapter{Solutions des exercices}
Tous les exercices ne seront pas corrigés.

\section{Hahn-Banach (Formes analytiques)}
\textbf{Solution de l'exercice \ref{hb:a:exo1}}

\begin{enumerate}
\item Supposons que $x_0\in \mathrm{adh(F)}$.
  Etant donné que le noyau de toute forme linéaire
  continue est fermé (le noyau est l'image réciproque du singleton
  0 qui est fermé, donc est fermé par continuité), et que le passage
  à l'adhérence conserve les inclusions, il est aisé de conclure.

  Réciproquement, supposons que $x_0\notin \mathrm{adh(F)}$.
  Par le corollaire \ref{hb:a:cor4} des formes analytiques de
  Hahn-Banach, il existe une forme s'annulant sur $F$ et pas
  en $x_0$, ce que l'on voulait montrer.
\item Supposons $F$ dense dans $E$. Soit $x^*\in E^*$ tel
  que $F\subseteq\mathrm{Ker}(x^*)$. Etant donné que ce
  noyau est fermé, il contient la fermeture de $F$ qui
  est $E$. Ceci implique que $x^*$ est l'application
  constante nulle.

  Réciproquement, supposons $F$ non dense dans $E$.
  Il suffit de considérer un élément du complémentaire
  de l'adhérence de $F$ et d'appliquer le résultat \ref{hb:a:cor4}
  pour obtenir une forme linéaire dont le noyau contient
  $F$ mais non identiquement nulle.
\end{enumerate}

\section{Hahn-Banach (Formes géométriques)}
\textbf{Solution de l'exercice \ref{hb:g:j2}}

On vérifie facilement que si $x\in \alpha C$, alors pour tout
$\beta >\alpha$, $x\in \beta C$;
avoir que $\alpha^{-1}x\in C$ implique que le segment joignant $0$ et ce
point est contenu dans $C$, et $\beta^{-1}x$ est dans le segment car
$\beta^{-1} <\alpha^{-1}$.

Par définition de la jauge, $j_C(x)= \inf\{\alpha > 0 \mid x\in \alpha C\}$.
Soit $\varepsilon >0$, alors il existe $\alpha > 0$ tel que $j_C(x)\leq
\alpha \leq j_C(x) + \varepsilon$ et $x\in\alpha C$. Par ce qui précède,
on peut conclure.

%%% Local Variables:
%%% mode: latex
%%% TeX-master: "analyse3"
%%% End:


\bibliography{references}
\bibliographystyle{ieeetr}
\end{document}

%%% Local Variables:
%%% mode: latex
%%% TeX-master: t
%%% End:
